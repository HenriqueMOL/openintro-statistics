


%_______________
\newpage\subsection*{Exercises} % Sampling from a small population

% 1

\eoce{\qt{Marbles in an urn\label{marbles_in_urn}} Imagine you have an urn 
containing 5 red, 3 blue, and 2 orange marbles in it. 
\begin{parts}
\item What is the probability that the first marble you draw is blue?
\item Suppose you drew a blue marble in the first draw. If drawing with 
replacement, what is the probability of drawing a blue marble in the second draw?
\item Suppose you instead drew an orange marble in the first draw. If drawing 
with replacement, what is the probability of drawing a blue marble in the second 
draw?
\item If drawing with replacement, what is the probability of drawing two blue 
marbles in a row?
\item When drawing with replacement, are the draws independent? Explain.
\end{parts}
}{}

% 2

\eoce{\qt{Socks in a drawer\label{socks_in_drawer}} In your sock drawer you have 
4 blue, 5 gray, and 3 black socks. Half asleep one morning you grab 2 socks at 
random and put them on. Find the probability you end up wearing
\begin{parts}
\item 2 blue socks
\item no gray socks
\item at least 1 black sock
\item a green sock
\item matching socks
\end{parts}
}{}

% 3

\eoce{\qt{Chips in a bag\label{chips_in_bag}} Imagine you have a bag 
containing 5 red, 3 blue, and 2 orange chips.
\begin{parts}
\item Suppose you draw a chip and it is blue. If drawing without replacement, 
what is the probability the next is also blue?
\item Suppose you draw a chip and it is orange, and then you draw a second chip 
without replacement. What is the probability this second chip is blue?
\item If drawing without replacement, what is the probability of drawing two blue 
chips in a row?
\item When drawing without replacement, are the draws independent? Explain.
\end{parts}
}{}

% 4

\eoce{\qt{Books on a bookshelf\label{books_on_shelf}} The table below shows the 
distribution of books on a bookcase based on whether they are nonfiction or 
fiction and hardcover or paperback.
\begin{center}
\begin{tabular}{ll  cc c} 
                                &           & \multicolumn{2}{c}{\textit{Format}} \\
\cline{3-4}
                                &           & Hardcover     & Paperback     & Total \\
\cline{2-5}
\multirow{2}{*}{\textit{Type}}  & Fiction   & 13            & 59            & 72 \\
                                & Nonfiction& 15            & 8             & 23 \\
\cline{2-5} 
                                & Total     & 28            & 67            & 95 \\
\cline{2-5}
\end{tabular}
\end{center}
\begin{parts}
\item Find the probability of drawing a hardcover book first then a paperback 
fiction book second when drawing without replacement.
\item Determine the probability of drawing a fiction book first and then a 
hardcover book second, when drawing without replacement.
\item Calculate the probability of the scenario in part~(b), except this time 
complete the calculations under the scenario where the first book is placed back 
on the bookcase before randomly drawing the second book.
\item The final answers to parts~(b) and~(c) are very similar. Explain why this 
is the case.
\end{parts}
}{}

% 5

\eoce{\qt{Student outfits\label{student_outfits}} In a classroom with 24 
students, 7 students are wearing jeans, 4 are wearing shorts, 8 are wearing 
skirts, and the rest are wearing leggings. If we randomly select 3 students 
without replacement, what is the probability that one of the selected students is 
wearing leggings and the other two are wearing jeans? Note that these are 
mutually exclusive clothing options.
}{}

% 6

\eoce{\qt{The birthday problem\label{birthday_problem}} Suppose we pick three 
people at random. For each of the following questions, ignore the special case 
where someone might be born on February 29th, and assume that births are evenly 
distributed throughout the year.
\begin{parts}
\item What is the probability that the first two people share a birthday? 
\item What is the probability that at least two people share a birthday?
\end{parts}
}{}
