\exercisesheader{}

% 1

\eoce{\qt{Quitters\label{quitters_chisq_independence}} Does being part of a 
support group affect the ability of people to quit smoking? A county 
health department enrolled 300 smokers in a randomized experiment. 150 
participants were assigned to a group that used a nicotine patch and 
met weekly with a support group; the other 150 received the patch and 
did not meet with a support group. At the end of the study, 40 of the 
participants in the patch plus support group had quit smoking while 
only 30 smokers had  quit in the other group.
\begin{parts}
\item Create a two-way table presenting the results of this study.
\item Answer each of the following questions under the null hypothesis 
that being part of a support group does not affect the ability of 
people to quit smoking, and indicate whether the expected values are 
higher or lower than the observed values.
\begin{subparts}
\item How many subjects in the ``patch + support" group would you 
expect to quit?
\item How many subjects in the ``patch only" group would you expect to 
not quit?
\end{subparts}
\end{parts}
}{}

% 2

\eoce{\qt{Full body scan, Part II\label{full_body_scan_chisq_indep}} The 
table below summarizes a data set we first encountered in 
Exercise~\ref{full_body_scan_HT_Error} regarding views on full-body 
scans and political affiliation. The differences in each political 
group may be due to chance. Complete the following computations under 
the null hypothesis of independence between an individual's party 
affiliation and his support of full-body scans. It may be useful to 
first add on an extra column for row totals before proceeding with the 
computations.
\begin{center}
\begin{tabular}{ll  cc c} 
            &   & \multicolumn{3}{c}{\textit{Party Affiliation}} \\
\cline{3-5}
                                &           & Republican & Democrat & Independent   \\
\cline{2-5}
\multirow{3}{*}{\textit{Answer}}& Should    & 264        & 299      & 351 \\
                                & Should not& 38         & 55       & 77 \\
                                & Don't know/No answer & 16 & 15    & 22 \\
\cline{2-5}
                                & Total      & 318       & 369      & 450
\end{tabular}
\end{center}
\begin{parts}
\item How many Republicans would you expect to not support the use of 
full-body scans?
\item How many Democrats would you expect to support the use of full-
body scans?
\item How many Independents would you expect to not know or not answer?
\end{parts}
}{}

% 3

\eoce{\qt{Offshore drilling, Part III\label{offshore_drilling_chisq_indep}} 
The table below summarizes a data set we first encountered in 
Exercise~\ref{offshore_drill_edu_dontknow_HT} that examines the 
responses of a random sample of college graduates and non-graduates on 
the topic of oil drilling. Complete a chi-square test for these data to 
check whether there is a statistically significant difference in 
responses from college graduates and non-graduates.
\begin{center}
\begin{tabular}{l c c}
			& \multicolumn{2}{c}{\textit{College Grad}} \\
\cline{2-3}
			& Yes		& No				\\
\cline{1-3}
Support		& 154		& 132			\\
Oppose		& 180		& 126			\\
Do not know	& 104		& 131			\\
\cline{1-3}
 Total		& 438		& 389		
\end{tabular}
\end{center}
}{}

% 4

\eoce{\qtq{TITLE\label{ID}}
\textbf{\color{red}REPLACE THIS EXERCISE}
The American 
National Election Studies (ANES) collects data on voter attitudes and intentions 
as well as demographic information. In this question we will focus on two 
variables from the ANES dataset:\footfullcite{data:anes:2012} 
\begin{itemize}
\item region (levels: Northeast, North Central, South, and West), and
\item whether the respondent feels things in this country are generally going 
in the right direction or things have pretty seriously gotten off on the wrong track.
\end{itemize}
To keep calculations simple we will work with a random sample of 500 
respondents from the ANES dataset. The distribution of responses are as follows:
\begin{center}
\begin{tabular}{rrr|r}
  \hline
                & Right     & Wrong     &  \\ 
                & Direction &  Track    & Total \\ 
  \hline
Northeast       & 29        & 54        & 83 \\ 
North Central   & 44        & 77        & 121 \\ 
South           & 62        & 131       & 193 \\ 
West            & 36        & 67        & 103 \\ 
  \hline
Total           & 171       & 329       & 500 \\ 
  \hline
\end{tabular}
\end{center}
\begin{parts}
\item Region: According to the 2010 Census, 18\% of US residents live in 
the Northeast, 22\% live in the North Central region, 37\% live in the South, 
and 23\% live in the West. Evaluate whether the ANES sample is representative 
of the population distribution of US residents. Make sure to clearly state 
the hypotheses, check conditions, calculate the appropriate test statistic 
and the p-value, and make your conclusion in context of the data. Also comment 
on what your conclusion says about whether or not this sample can be 
considered to be representative.
\item Region and direction:
\begin{enumerate}[(i)]
\item We would like to evaluate the relationship between region and feeling 
about the country's direction. What is the response variable and what is the 
explanatory variable?
\item What are the hypotheses for evaluating this relationship?
\item Complete the hypothesis test and interpret your results in context of 
the data and the research question.
\end{enumerate}
\end{parts}
}{}
