\exercisesheader{}

% 31

\eoce{\qt{True or false, Part I\label{tf_chisq_1}} Determine if the statements below 
are true or false. For each false statement, suggest an alternative wording to 
make it a true statement.
\begin{parts}
\item The chi-square distribution, just like the normal distribution, has two 
parameters, mean and standard deviation.
\item The chi-square distribution is always right skewed, regardless of the 
value of the degrees of freedom parameter.
\item The chi-square statistic is always positive.
\item As the degrees of freedom increases, the shape of the chi-square 
distribution becomes more skewed.
\end{parts}
}{}

% 32

\eoce{\qt{True or false, Part II\label{tf_chisq_2}} Determine if the statements below 
are true or false. For each false statement, suggest an alternative wording to 
make it a true statement.
\begin{parts}
\item As the degrees of freedom increases, the mean of the chi-square 
distribution increases.
\item If you found $\chi^2 = 10$ with $df = 5$ you would fail to reject $H_0$ 
at the 5\% significance level.
\item When finding the p-value of a chi-square test, we always shade the tail 
areas in both tails.
\item As the degrees of freedom increases, the variability of the chi-square 
distribution decreases.
\end{parts}
}{}

% 33

\eoce{\qt{Open source textbook\label{opensource_text_chisq_GOF}} A professor using 
an open source introductory statistics book predicts that 60\% of the 
students will purchase a hard copy of the book, 25\% will print it out from 
the web, and 15\% will read it online. At the end of the semester he asks his 
students to complete a survey where they indicate what format of the book 
they used. Of the 126 students, 71 said they bought a hard copy of the book, 
30 said they printed it out from the web, and 25 said they read it online.
\begin{parts}
\item State the hypotheses for testing if the professor's predictions were 
inaccurate.
\item How many students did the professor expect to buy the book, print the 
book, and read the book exclusively online?
\item This is an appropriate setting for a chi-square test. List the 
conditions required for a test and verify they are satisfied.
\item Calculate the chi-squared statistic, the degrees of freedom associated 
with it, and the p-value.
\item Based on the p-value calculated in part (d), what is the conclusion of 
the hypothesis test? Interpret your conclusion in this context.
\end{parts}
}{}

% 34

\eoce{\qt{Barking deer\label{barking_deer_chisq_GOF}}
Microhabitat factors associated with forage and bed sites
of barking deer in Hainan Island, China were examined.
In this region woods make up 4.8\% of the land,
cultivated grass plot makes up 14.7\%, and deciduous forests
make up 39.6\%.
Of the 426 sites where the deer forage, 4 were categorized
as woods, 16 as cultivated grassplot, and 61 as deciduous forests.
The table below summarizes these data.\footfullcite{Teng:2004}
\begin{center}
\begin{tabular}{c c c c c}
Woods	& Cultivated grassplot	& Deciduous forests	 & Other & Total \\
\hline 
4		& 16					& 61			     & 345	 & 426 \\
\end{tabular}
\end{center}

\noindent \begin{minipage}[c]{0.7\textwidth}
\begin{parts}
\item Write the hypotheses for testing if barking deer prefer to forage in 
certain habitats over others.
\item What type of test can we use to answer this research question?
\item Check if the assumptions and conditions required for this test are 
satisfied.
\item Do these data provide convincing evidence that barking deer prefer to 
forage in certain habitats over others? Conduct an appropriate hypothesis 
test to answer this research question.
\end{parts}
\end{minipage}
\begin{minipage}[c]{0.03\textwidth}
$\:$ \\
\end{minipage}
\begin{minipage}[c]{0.28\textwidth}
\begin{center}
\includegraphics[width=0.7\textwidth]{ch_inference_for_props/figures/eoce/barking_deer_chisq_GOF/barking_deer.jpg} \\
{\footnotesize Photo by Shrikant Rao (\oiRedirect{textbook-flickr_shrikant_rao_barking_deer}{http://flic.kr/p/4Xjdkk}) \oiRedirect{textbook-CC_BY_2}{CC~BY~2.0~license}}
\end{center}
\end{minipage}
}{}
