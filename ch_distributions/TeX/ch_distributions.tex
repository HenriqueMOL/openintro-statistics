\begin{chapterpage}{Distributions of random variables}
  \chaptertitle[30]{Distributions of random \titlebreak{} variables}
  \label{ch_distributions}
  \chaptersection{normalDist}
  %\chaptersection{assessingNormal}
  \chaptersection{geomDist}
  \chaptersection{binomialModel}
  \chaptersection{negativeBinomial}
  \chaptersection{poisson}
\end{chapterpage}
\renewcommand{\chapterfolder}{ch_distributions}


\chapterintro{In this chapter,
  we discuss statistical distributions that frequently
  arise in the context of data analysis or statistical
  inference.
  We start with the normal distribution in the first section,
  which is used frequently in later chapters of this book.
  The remaining sections will occasionally be referenced
  but may be considered optional for the content in this
  book.}

%_________________
\section{Normal distribution}
\label{normalDist}

\index{distribution!normal|(}
\index{normal distribution|(}

Among all the distributions we see in practice,
one is overwhelmingly the most common.
The symmetric, unimodal, bell curve is ubiquitous
throughout statistics.
Indeed it is so common, that people often know it as the
\termsub{normal curve}{normal distribution} or
\term{normal distribution}\index{distribution!normal|textbf}%
,\footnote{It
  is also introduced as the Gaussian distribution after Frederic
  Gauss, the first person to formalize its mathematical
  expression.}
shown in Figure~\ref{simpleNormal}.
Variables such as SAT scores and heights of US adult males
closely follow the normal distribution.

\begin{figure}[h]
  \centering
  \Figure{0.5}{simpleNormal}
  \caption{A normal curve.}
  \label{simpleNormal}
\end{figure}

\begin{onebox}{Normal distribution facts}
  Many variables are nearly normal, but none are exactly normal.
  Thus the normal distribution, while not perfect for any single
  problem, is very useful for a variety of problems.
  We will use it in data exploration and to solve important
  problems in statistics.
\end{onebox}


\subsection{Normal distribution model}

The \term{normal distribution} always describes a symmetric,
unimodal, bell-shaped curve.
However, these curves can look different depending on the
details of the model.
Specifically, the normal distribution model can be adjusted
using two parameters: mean and standard deviation.
As you can probably guess, changing the mean shifts the bell
curve to the left or right, while changing the standard deviation
stretches or constricts the curve.
Figure~\ref{twoSampleNormals} shows the normal distribution
with mean $0$ and standard deviation $1$ in the left panel
and the normal distributions with mean $19$ and standard
deviation $4$ in the right panel.
Figure~\ref{twoSampleNormalsStacked} shows these distributions
on the same axis.

\begin{figure}[h]
  \centering
  \Figure{0.7}{twoSampleNormals}
  \caption{Both curves represent the normal distribution.
      However, they differ in their center and spread.}
  \label{twoSampleNormals}
\end{figure}

\begin{figure}[h]
  \centering
  \Figure{0.6}{twoSampleNormalsStacked}
  \caption{The normal distributions shown in
      Figure~\ref{twoSampleNormals} but plotted together
      and on the same scale.}
  \label{twoSampleNormalsStacked}
\end{figure}

If a normal distribution has mean $\mu$ and standard deviation
$\sigma$, we may write the distribution as $N(\mu, \sigma)$.
The two distributions in Figure~\ref{twoSampleNormalsStacked}
may be written as
\begin{align*}
N(\mu=0,\sigma=1)
  \quad \text{and} \quad
  N(\mu=19,\sigma=4)
\end{align*}
Because the mean and standard deviation describe a normal
distribution exactly, they are called the distribution's
\termsub{parameters}{parameter}.
The normal distribution with mean $\mu = 0$ and
standard deviation $\sigma = 1$ is called the
\term{standard normal distribution}%
\index{normal distribution!standard|textbf}.

\begin{exercisewrap}
\begin{nexercise}
Write down the short-hand for a normal distribution
with\footnotemark{} \\
%\begin{enumerate}[(a)]
%\setlength{\itemsep}{0mm}
%\item
(a)
    mean~5 and standard deviation~3, \\
%\item
(b)
    mean~-100 and standard deviation~10, and \\
%\item
(c)
    mean~2 and standard deviation~9.
%\end{enumerate}
\end{nexercise}
\end{exercisewrap}
\footnotetext{(a)~$N(\mu=5,\sigma=3)$.
  (b)~$N(\mu=-100, \sigma=10)$.
  (c)~$N(\mu=2, \sigma=9)$.}


\subsection{Standardizing with Z-scores}

\noindent%
We often want to put data onto a standardized scale,
which can make comparisons more reasonable.

\newcommand{\satmean}{1100}
\newcommand{\satsd}{200}
\newcommand{\actmean}{21}
\newcommand{\actsd}{6}
\newcommand{\annsatscore}{1300}
\newcommand{\annsatzscore}{1}
\newcommand{\tomsatscore}{24}
\newcommand{\tomsatzscore}{0.5}

\begin{examplewrap}
\begin{nexample}{Table~\vref{satACTstats} shows the mean
    and standard deviation for total scores on the SAT and ACT.
    The distribution of SAT and ACT scores are both nearly normal.
    Suppose Ann scored \annsatscore{} on her SAT and Tom scored
    \tomsatscore{} on his ACT.
    Who performed better?}
  \label{actSAT}%
  We use the standard deviation as a guide.
  Ann is \annsatzscore{} standard deviation above average
  on the SAT: $\satmean{} + \satsd{} = \annsatscore{}$.
  Tom is \tomsatzscore{} standard deviations above the mean
  on the ACT:
  $\actmean{} + \tomsatzscore{} \times \actsd{} = \tomsatscore{}$.
  In Figure~\ref{satActNormals}, we can see that Ann tends
  to do better with respect to everyone else than Tom did,
  so her score was better.
\end{nexample}
\end{examplewrap}

\begin{figure}[h]
\centering
\begin{tabular}{l r r}
  \hline
  & SAT & ACT \\
  \hline
  Mean \hspace{0.3cm} & \satmean{} & \actmean{} \\
  SD & \satsd{} & \actsd{} \\
  \hline
\end{tabular}
\caption{Mean and standard deviation for the SAT and ACT.}
\label{satACTstats}
\end{figure}

\begin{figure}
  \centering
  \Figure{0.6}{satActNormals}
  \caption{Ann's and Tom's scores shown against
      the SAT and ACT distributions.}
  \label{satActNormals}
\end{figure}

Example~\ref{actSAT} used a standardization technique called
a Z-score, a method most commonly employed for nearly normal
observations but that may be used with any distribution.
The \term{Z-score}\index{Z@$Z$} of an observation is defined
as the number of standard deviations it falls above or below
the mean.
If the observation is one standard deviation above the mean,
its Z-score is~1.
If it is 1.5 standard deviations \emph{below} the mean,
then its Z-score is -1.5.
If $x$ is an observation from a distribution $N(\mu, \sigma)$,
we define the Z-score mathematically as
\begin{align*}
Z = \frac{x - \mu}{\sigma}
\end{align*}
Using $\mu_{SAT} = \satmean{}$, $\sigma_{SAT} = \satsd{}$,
and $x_{_{\text{Ann}}} = \annsatscore{}$, we find Ann's Z-score:
\begin{align*}
Z_{_{\text{Ann}}}
  = \frac{x_{_{\text{Ann}}} - \mu_{_{\text{SAT}}}}
      {\sigma_{_{\text{SAT}}}}
  = \frac{\annsatscore{} - \satmean{}}{\satsd{}}
  = \annsatzscore{}
\end{align*}

\begin{onebox}{The Z-score}
  The Z-score of an observation is the number of standard
  deviations it falls above or below the mean.
  We compute the Z-score for an observation $x$ that follows
  a distribution with mean $\mu$ and standard deviation
  $\sigma$ using
  \begin{align*}
  Z = \frac{x - \mu}{\sigma}
  \end{align*}
\end{onebox}

\begin{exercisewrap}
\begin{nexercise}
Use Tom's ACT score, \tomsatscore{}, along with the ACT mean and
standard deviation to find his Z-score.\footnotemark{}
\end{nexercise}
\end{exercisewrap}
\footnotetext{$Z_{Tom}
  = \frac{x_{\text{Tom}} - \mu_{\text{ACT}}}
      {\sigma_{\text{ACT}}}
  = \frac{\tomsatscore{} - \actmean{}}{\actsd{}}
  = \tomsatzscore{}$}

Observations above the mean always have positive Z-scores,
while those below the mean always have negative Z-scores.
If an observation is equal to the mean,
such as an SAT score of \satmean{}, then the Z-score is $0$.

\begin{exercisewrap}
\begin{nexercise}
Let $X$ represent a random variable from $N(\mu=3, \sigma=2)$,
and suppose we observe $x=5.19$. \\
%\begin{enumerate}[(a)]
%\setlength{\itemsep}{0mm}
%\item
(a)
    Find the Z-score of $x$. \\
%\item
(b)
    Use the Z-score to determine how many standard deviations
    above or below the mean $x$ falls.\footnotemark{}
%\end{enumerate}
\end{nexercise}
\end{exercisewrap}
\footnotetext{(a) Its Z-score is given by
    $Z
      = \frac{x-\mu}{\sigma}
      = \frac{5.19 - 3}{2}
      = 2.19/2
      = 1.095$.
    (b)~The observation $x$ is 1.095 standard deviations
    \emph{above} the mean.
    We know it must be above the mean since $Z$ is positive.}

\begin{exercisewrap}
\begin{nexercise} \label{headLZScore}
Head lengths of brushtail possums follow a normal
distribution with mean 92.6 mm and standard deviation 3.6 mm.
Compute the Z-scores for possums with head lengths of 95.4 mm
and 85.8~mm.\footnotemark{}
\end{nexercise}
\end{exercisewrap}
\footnotetext{For $x_1=95.4$ mm:
    $Z_1
      = \frac{x_1 - \mu}{\sigma}
      = \frac{95.4 - 92.6}{3.6}
      = 0.78$.
    For $x_2=85.8$ mm:
    $Z_2 = \frac{85.8 - 92.6}{3.6} = -1.89$.}

We can use Z-scores to roughly identify which observations
are more unusual than others.
An observation $x_1$ is said to be more unusual than another
observation $x_2$ if the absolute value of its Z-score is larger
than the absolute value of the other observation's Z-score:
$|Z_1| > |Z_2|$.
This technique is especially insightful when a distribution
is symmetric.

%\D{\newpage}

\begin{exercisewrap}
\begin{nexercise}
Which of the observations in Guided Practice~\ref{headLZScore}
is more unusual?\footnotemark{}
\end{nexercise}
\end{exercisewrap}
\footnotetext{Because the \emph{absolute value} of Z-score
  for the second observation is larger than that of the first,
  the second observation has a more unusual head length.}


\subsection{Finding tail areas}

It's very useful in statistics to be able to identify tail areas
of distributions.
For instance, how many people have an SAT score below
Ann's score of 1300?
This is the same as Ann's \term{percentile}, which is
the fraction of cases that have lower scores than Ann.
We can visualize such a tail area like the curve and shading
shown in Figure~\ref{satBelow1300}.


\begin{figure}[h]
  \centering
  \Figure{0.45}{satBelow1300}
  \caption{The area to the left of $Z$ represents the
      percentile of the observation.}
  \label{satBelow1300}
\end{figure}

There are many techniques for doing this, and we'll discuss
three of the options.
\begin{enumerate}
\item
    The most common approach in practice is to use
    statistical software.
    For example, in the program \R{}, we could find the area
    shown in Figure~\ref{satBelow1300} using the
    following command, which takes in the Z-score
    and returns the lower tail area: \\
    {\color{white}.....}%
        \texttt{> pnorm(1)} \\
    {\color{white}.....}%
        \texttt{[1] 0.8413447} \\
    According to this calculation,
    the region shaded that is below 1300
    represents the proportion 0.841 (84.1\%) of SAT test
    takers who had Z-scores below $Z = 1$.
    More generally, we can also specify the cutoff explicitly
    if we also note the mean and standard deviation: \\
    {\color{white}.....}%
        \texttt{> pnorm(1300, mean = 1100, sd = 200)} \\
    {\color{white}.....}%
        \texttt{[1] 0.8413447} %\\
    %\Add{More examples for using \R{} are provided
    %  at the end of the section.}


    There are many other software options, such as Python or SAS;
    even spreadsheet programs such as
    Excel and Google Sheets support these calculations.
\item
    A common strategy in classrooms is to use a graphing
    calculator, such as a TI or Casio calculator.
    These calculators require a series of button presses
    that are less concisely described.
    You can find instructions on using these calculators
    for finding tail areas of a normal distribution in the
    OpenIntro video library:
    \begin{center}
    \oiRedirect{textbook-openintro_videos}
        {www.openintro.org/videos}
    \end{center}
\item
    The last option for finding tail areas is to use
    what's called a \term{probability table};
    these are occasionally used in classrooms
    but rarely in practice.
    Appendix~\ref{normalProbabilityTable}
    contains such a table and a guide for how to use it.
\end{enumerate}
We will solve normal distribution problems in this section
by always first finding the Z-score.
The reason is that we will encounter close parallels
called \indexthis{test statistics}{test statistic}
beginning in Chapter~\ref{ch_foundations_for_inf};
these are, in many instances, an equivalent of a Z-score.

%No matter the approach you choose,
%try the Guided Practice exercises in this section
%using your preferred method.


\D{\newpage}

\subsection{Normal probability examples}
\label{normal_probability_examples}

\noindent%
Cumulative SAT scores are approximated well by a normal model,
$N(\mu = \satmean{}, \sigma = \satsd{})$.

\newcommand{\shannonsat}{1190}
\newcommand{\shannonsatz}{0.45}
\begin{examplewrap}
\begin{nexample}{Shannon is a randomly selected SAT taker,
    and nothing is known about Shannon's SAT aptitude.
    What is the probability Shannon scores at least
    \shannonsat{} on her SATs?}
  \label{satAbove1190Exam}%
  First, always draw and label a picture of the normal
  distribution.
  (Drawings need not be exact to be useful.)
  We are interested in the chance she scores above
  \shannonsat{}, so we shade this upper tail:
  \begin{center}
  \Figure{0.4}{satAbove1190}
  \end{center}
  The picture shows the mean and the values at
  2~standard deviations above and below the mean.
  The simplest way to find the shaded area under
  the curve makes use of the Z-score of the cutoff value.
  With $\mu = \satmean{}$, $\sigma = \satsd{}$,
  and the cutoff value $x = \shannonsat{}$,
  the Z-score is computed as
  \begin{align*}
  Z = \frac{x - \mu}{\sigma}
    = \frac{\shannonsat{} - \satmean{}}{\satsd{}}
    = \frac{90}{\satsd{}}
    = \shannonsatz{}
  \end{align*}
  Using statistical software (or another preferred method),
  we can area left of $Z = \shannonsatz{}$ as 0.6736.
  %This is Shannon's \term{percentile},
  %which is the fraction of folks who scored below her score
  %of \shannonsat{}.
  To find the area \emph{above} $Z = \shannonsatz{}$,
  we compute one minus the area of the lower tail:
  \begin{center}
  \Figure{0.4}{subtractingArea}
  \end{center}
  The probability Shannon scores at least 1190 on the SAT
  is 0.3264.
\end{nexample}
\end{examplewrap}

\begin{onebox}{Always draw a picture first,
    and find the Z-score second}
  For any normal probability situation,
  \emph{always always always} draw and label the
  normal curve and shade the area of interest first.
  The picture will provide an estimate of the probability.
  After drawing a figure to represent the situation,
  identify the Z-score for the value of interest.
\end{onebox}

\begin{exercisewrap}
\begin{nexercise}
If the probability of Shannon scoring at least \shannonsat{}
is 0.3264, then what is the probability she scores less than
\shannonsat{}?
Draw the normal curve representing this exercise,
shading the lower region instead of the upper one.\footnotemark{}
\end{nexercise}
\end{exercisewrap}
\footnotetext{We found this probability in
  Example~\ref{satAbove1190Exam}: 0.6736. \\
  \Figures{0.35}{subtractingArea}{subtracted}}

\D{\newpage}

\newcommand{\edwardsat}{1030}
\newcommand{\edwardsatz}{-0.35}
\newcommand{\edwardsatlower}{0.3632}
\begin{examplewrap}
\begin{nexample}{Edward earned a \edwardsat{} on his SAT.
    What is his percentile?}
  \label{edwardSatBelow\edwardsat{}}%
  First, a picture is needed.
  Edward's \hiddenterm{percentile} is the proportion of people
  who do not get as high as a \edwardsat{}.
  These are the scores to the left of \edwardsat{}.
\begin{center}
\Figure{0.3}{satBelow1030}
\end{center}
Identifying the mean $\mu=\satmean{}$, the standard
deviation $\sigma=\satsd{}$, and the cutoff for the tail
area $x=\edwardsat{}$ makes it easy to compute the Z-score:
\begin{align*}
Z
  = \frac{x - \mu}{\sigma}
  = \frac{\edwardsat{} - \satmean{}}{\satsd{}}
  = \edwardsatz{}
\end{align*}
Using statistical software, we get a tail area of 0.3632.
Edward is at the $36^{th}$ percentile.
\end{nexample}
\end{examplewrap}

\begin{exercisewrap}
\begin{nexercise}
Use the results of Example~\ref{edwardSatBelow\edwardsat{}}
to compute the proportion of SAT takers who did better than
Edward.
Also draw a new picture.\footnotemark{}
\end{nexercise}
\end{exercisewrap}
\footnotetext{If Edward did better than 36\% of SAT takers,
  then about 64\% must have done better than him. \\
  \Figures{0.25}{satBelow1030}{satAbove1030}}

\begin{onebox}{Finding areas to the right}
  Many software programs return the area to the left
  when given a Z-score.
  If you would like the area to the right, first find the
  area to the left and then subtract this amount from~one.
\end{onebox}

\newcommand{\stuartsat}{1500}
\newcommand{\stuarsatz}{2}
\begin{exercisewrap}
\begin{nexercise}
Stuart earned an SAT score of \stuartsat{}.
Draw a picture for each part. \\
(a)~What is his percentile? \\
(b)~What percent of SAT takers did better than
  Stuart?\footnotemark{}
\end{nexercise}
\end{exercisewrap}
\footnotetext{We leave the drawings to you.
  (a) $Z = \frac{\stuartsat{} - \satmean{}}{\satsd{}}
         = \stuarsatz{}
         \to 0.9772$.
  (b) $1 - 0.9772 = 0.0228$.}

Based on a sample of 100 men, the heights of male adults
in the US is nearly normal with mean 70.0''
and standard deviation 3.3''.

\begin{exercisewrap}
\begin{nexercise}
Mike is 5'7'' and Jose is 6'4'', and they both live in the US. \\
(a) What is Mike's height percentile? \\
(b) What is Jose's height percentile? \\
Also draw one picture for each~part.\footnotemark{}
\end{nexercise}
\end{exercisewrap}
\footnotetext{First put the heights into inches:
  67 and 76 inches.
  Figures are shown below. \\
  (a) $Z_{\text{Mike}} = \frac{67 - 70}{3.3} = -0.91\ \to\ 0.1814$.
  (b) $Z_{\text{Jose}} = \frac{76 - 70}{3.3} = 1.82\ \to\ 0.9656$.
  \\
  \Figure{0.45}{mikeAndJosePercentiles}}

\D{\newpage}

The last several problems have focused on finding the
percentile (or upper tail) for a particular observation.
What if you would like to know the observation corresponding
to a particular percentile?

\begin{examplewrap}
\begin{nexample}{Erik's height is at the $40^{th}$ percentile.
    How tall is he?}\label{normalExam40Perc}
  As always, first draw the picture.\vspace{-4mm}
  \begin{center}
  \Figure{0.3}{height40Perc}\vspace{-1mm}
  \end{center}
  In this case, the lower tail probability is known (0.40),
  which can be shaded on the diagram.
  We want to find the observation that corresponds to this value.
  As a first step in this direction, we determine the Z-score
  associated with the $40^{th}$ percentile.
  Using software, we can obtain the corresponding Z-score
  of about -0.25.

  Knowing $Z_{Erik} = -0.25$ and the population parameters
  $\mu = 70$ and $\sigma = 3.3$ inches, the Z-score formula can be
  set up to determine Erik's unknown height, labeled
  $x_{_{\text{Erik}}}$:
  \begin{align*}
  -0.25
    = Z_{_{\text{Erik}}}
    = \frac{x_{_{\text{Erik}}} - \mu}{\sigma}
    = \frac{x_{_{\text{Erik}}} - 70}{3.3}
  \end{align*}
  Solving for $x_{_{\text{Erik}}}$ yields a height of 69.18 inches.
  That is, Erik is about 5'9''.
\end{nexample}
\end{examplewrap}

\begin{examplewrap}
\begin{nexample}{What is the adult male height at the
    $82^{nd}$ percentile?}
  Again, we draw the figure first.\vspace{-3mm}
  \begin{center}
  \Figure{0.28}{height82Perc}\vspace{-1mm}
  \end{center}
  Next, we want to find the Z-score at the $82^{nd}$ percentile,
  which will be a positive value and can be found using software
  as $Z = 0.92$.
  Finally, the height $x$ is found using the Z-score formula
  with the known mean $\mu$, standard deviation $\sigma$,
  and Z-score $Z = 0.92$:
  \begin{align*}
  0.92 = Z = \frac{x-\mu}{\sigma} = \frac{x - 70}{3.3}
  \end{align*}
  This yields 73.04 inches or about 6'1'' as the height
  at the $82^{nd}$ percentile.
\end{nexample}
\end{examplewrap}

\begin{exercisewrap}
\begin{nexercise}
The SAT scores follow $N(\satmean{}, \satsd{})$.\footnotemark{} \\
(a) What is the $95^{th}$ percentile for SAT scores? \\
(b) What is the $97.5^{th}$ percentile for SAT scores?
\end{nexercise}
\end{exercisewrap}
\footnotetext{Short answers:
  (a) $Z_{95} = 1.65 \to 1430$ SAT score.
  (b) $Z_{97.5} = 1.96 \to 1492$ SAT score.}

\D{\newpage}

\begin{exercisewrap}
\begin{nexercise}\label{more74Less69}
Adult male heights follow $N(70.0$''$, 3.3$''$)$.\footnotemark{} \\
(a)~What is the probability that a randomly selected male
    adult is at least 6'2'' (74 inches)? \\
(b)~What is the probability that a male adult is shorter
    than 5'9'' (69 inches)?
\end{nexercise}
\end{exercisewrap}
\footnotetext{Short answers:
  (a) $Z = 1.21 \to 0.8869$, then subtract this value
      from 1 to get 0.1131.
  (b) $Z = -0.30 \to 0.3821$.}

\begin{examplewrap}
\begin{nexample}{What is the probability that a random adult
    male is between 5'9'' and 6'2''?}
  These heights correspond to 69 inches and 74 inches.
  First, draw the figure.
  The area of interest is no longer an upper or lower
  tail.\vspace{-2mm}
  \begin{center}
  \Figure{0.35}{between59And62}\vspace{-2mm}
  \end{center}
  The total area under the curve is~1.
  If we find the area of the two tails that are not shaded
  (from Guided Practice~\ref{more74Less69}, these areas are
  $0.3821$ and $0.1131$), then we can find the middle
  area:\vspace{-2mm}
  \begin{center}
  \Figure{0.55}{subtracting2Areas}\vspace{-2mm}
  \end{center}
  That is, the probability of being between 5'9'' and 6'2''
  is 0.5048.
\end{nexample}
\end{examplewrap}

\begin{exercisewrap}
\begin{nexercise}
SAT scores follow $N(\satmean{}, \satsd{})$.
What percent of SAT takers get between \satmean{} and
1400?\footnotemark
\end{nexercise}
\end{exercisewrap}
\footnotetext{This is an abbreviated solution.
  (Be sure to draw a figure!)
  First find the percent who get below \satmean{}
  and the percent that get above 1400:
  $Z_{\satmean{}} = 0.00 \to 0.5000$ (area below),
  $Z_{1400} = 1.5 \to 0.0668$ (area above).
  Final answer: $1.0000 - 0.5000 - 0.0668 = 0.4332$.}

\begin{exercisewrap}
\begin{nexercise}
Adult male heights follow $N(70.0$''$, 3.3$''$)$.
What percent of adult males are between 5'5''
and 5'7''?\footnotemark{}
\end{nexercise}
\end{exercisewrap}
\footnotetext{5'5'' is 65 inches ($Z = -1.52$).
  5'7'' is 67 inches ($Z = -0.91$).
  Numerical solution: $1.000 - 0.0643 - 0.8186 = 0.1171$,
  i.e. 11.71\%.}


\D{\newpage}

\subsection{68-95-99.7 rule}

Here, we present a useful rule of thumb for the probability of falling within 1, 2, and 3 standard deviations of the mean in the normal distribution. This will be useful in a wide range of practical settings, especially when trying to make a quick estimate without a calculator or Z-table.

\begin{figure}[hht]
\centering
\includegraphics[height=1.9in]{ch_distributions/figures/6895997/6895997}
\caption{Probabilities for falling within 1, 2, and 3 standard deviations of the mean in a normal distribution.}
\label{6895997}
\end{figure}

\begin{exercisewrap}
\begin{nexercise}
Use software, a calculator, or a probability table
to confirm that about 68\%, 95\%, and 99.7\%
of observations fall within 1, 2, and 3, standard deviations
of the mean in the normal distribution, respectively.
For instance, first find the area that falls between $Z=-1$
and $Z=1$, which should have an area of about 0.68.
Similarly there should be an area of about 0.95 between
$Z=-2$ and $Z=2$.\footnotemark{}
\end{nexercise}
\end{exercisewrap}
\footnotetext{First draw the pictures.
  Using software, we get 0.6827 within 1~standard deviation,
  0.9545 within 2~standard deviations,
  and 0.9973 within 3~standard deviations.}

It is possible for a normal random variable to fall 4,~5,
or~even more standard deviations from the mean.
However, these occurrences are very rare if the data are
nearly normal.
The probability of being further than 4 standard deviations
from the mean is about 1-in-15,000.
For 5 and 6 standard deviations, it is about 1-in-2 million
and 1-in-500 million, respectively.

\begin{exercisewrap}
\begin{nexercise}
SAT scores closely follow the normal model with mean
$\mu = \satmean{}$ and standard deviation
$\sigma = \satsd{}$.\footnotemark{} \\
(a) About what percent of test takers score 700 to 1500? \\
(b) What percent score between \satmean{} and 1500?
\end{nexercise}
\end{exercisewrap}
\footnotetext{(a) 700 and 1500 represent two standard deviations
  below and above the mean, which means about 95\% of test takers
  will score between 700 and 1500.
  (b)~We found that 700 to 1500 represents about 95\% of test
  takers.
  These test takers would be evenly split by the center of
  the distribution, \satmean{},
  so $\frac{95\%}{2} = 47.5\%$ of all test takers
  score between \satmean{} and 1500.}


{


%_______________
\newpage\subsection*{Exercises} % Normal distribution

% 1

\eoce{\qt{Area under the curve, Part I\label{area_under_curve_1}} What percent of a 
standard normal distribution $N(\mu=0, \sigma=1)$ is found in each region? 
Be sure to draw a graph. \vspace{-3mm}
\begin{multicols}{4}
\begin{parts}
\item $Z < -1.35$
\item $Z > 1.48$
\item $-0.4 < Z < 1.5$
\item $|Z| > 2$
\end{parts}
\end{multicols}
}{}

% 2

\eoce{\qt{Area under the curve, Part II\label{area_under_curve_2}} What percent of 
a standard normal distribution $N(\mu=0, \sigma=1)$ is found in each region? 
Be sure to draw a graph. \vspace{-3mm}
\begin{multicols}{4}
\begin{parts}
\item $Z > -1.13$
\item $Z < 0.18$
\item $Z > 8$
\item $|Z| < 0.5$
\end{parts}
\end{multicols}
}{}

% 3

\eoce{\qt{GRE scores, Part I\label{GRE_intro}} Sophia who took the Graduate Record 
Examination (GRE) scored 160 on the Verbal Reasoning section and 157 on the 
Quantitative Reasoning section. The mean score for Verbal Reasoning section 
for all test takers was 151 with a standard deviation of 7, and the mean 
score for the Quantitative Reasoning was 153 with a standard deviation of 
7.67. Suppose that both distributions are nearly normal. 
\begin{parts}
\item Write down the short-hand for these two normal distributions.
\item What is  Sophia's Z-score on the Verbal Reasoning section? On the 
Quantitative Reasoning section? Draw a standard normal distribution curve and 
mark these two Z-scores.
\item What do these Z-scores tell you?
\item Relative to others, which section did she do better on?
\item Find her percentile scores for the two exams.
\item What percent of the test takers did better than her on the Verbal 
Reasoning section? On the Quantitative Reasoning section?
\item Explain why simply comparing raw scores from the two sections could lead 
to an incorrect conclusion as to which section a student did better on.
\item If the distributions of the scores on these exams are not nearly 
normal, would your answers to parts (b) - (f) change? Explain your reasoning.
\end{parts}
}{}

% 4

\eoce{\qt{Triathlon times, Part I\label{triathlon_times_intro}} In triathlons, it 
is common for racers to be placed into age and gender groups. Friends Leo and 
Mary both completed the Hermosa Beach Triathlon, where Leo competed in the 
\textit{Men, Ages 30 - 34} group while Mary competed in the \textit{Women, 
Ages 25 - 29} group. Leo completed the race in 1:22:28 (4948 seconds), while 
Mary completed the race in 1:31:53 (5513 seconds). Obviously Leo finished 
faster, but they are curious about how they did within their respective 
groups. Can you help them? Here is some information on the performance of 
their groups:
\begin{itemize}
\setlength{\itemsep}{0mm}
\item The finishing times of the \textit{Men, Ages 30 - 34} group has a mean 
of 4313 seconds with a standard deviation of 583 seconds.
\item The finishing times of the \textit{Women, Ages 25 - 29} group has a 
mean of 5261 seconds with a standard deviation of 807 seconds.
\item The distributions of finishing times for both groups are approximately 
Normal.
\end{itemize}
Remember: a better performance corresponds to a faster finish.
\begin{parts}
\item Write down the short-hand for these two normal distributions.
\item What are the Z-scores for Leo's and Mary's finishing times? What do 
these Z-scores tell you?
\item Did Leo or Mary rank better in their respective groups? Explain your 
reasoning.
\item What percent of the triathletes did Leo finish faster than in his group?
\item What percent of the triathletes did Mary finish faster than in her 
group?
\item If the distributions of finishing times are not nearly normal, would 
your answers to parts (b)~-~(e) change? Explain your reasoning.
\end{parts}
}{}

% 5

\eoce{\qt{GRE scores, Part II\label{GRE_cutoffs}} In Exercise~\ref{GRE_intro} we 
saw two distributions for GRE scores: $N(\mu=151, \sigma=7)$ for the verbal 
part of the exam and $N(\mu=153, \sigma=7.67)$ for the quantitative part. Use 
this information to compute each of the following:
\begin{parts}
\item The score of a student who scored in the $80^{th}$ percentile on the 
Quantitative Reasoning section.
\item The score of a student who scored worse than 70\% of the test takers in 
the Verbal Reasoning section.
\end{parts}
}{}

% 6

\eoce{\qt{Triathlon times, Part II\label{triathlon_times_cutoffs}} In 
Exercise~\ref{triathlon_times_intro} we saw two distributions for triathlon 
times: $N(\mu=4313, \sigma=583)$ for \emph{Men, Ages 30 - 34} and 
$N(\mu=5261, \sigma=807)$ for the \emph{Women, Ages 25 - 29} group. Times are 
listed in seconds. Use this information to compute each of the following:
\begin{parts}
\item The cutoff time for the fastest 5\% of athletes in the men's group, i.e. those 
who took the shortest 5\% of time to finish. 
\item The cutoff time for the slowest 10\% of athletes in the women's group. 
\end{parts}
}{}

% 7

\eoce{\qt{LA weather, Part I\label{la_weather_intro}} The average daily high 
temperature in June in LA is 77\degree F with a standard deviation of 
5\degree F. Suppose that the temperatures in June closely follow a normal 
distribution. 
\begin{parts}
\item What is the probability of observing an 83\degree F temperature or 
higher in LA during a randomly chosen day in June?
\item How cool are the coldest 10\% of the days (days with lowest average 
high temperature) during June in LA?
\end{parts}
}{}

% 8

\eoce{\qt{CAPM\label{CAPM}} The Capital Asset Pricing Model (CAPM) is a financial 
model that assumes returns on a portfolio are normally distributed. Suppose a 
portfolio has an average annual return of 14.7\% (i.e. an average gain of 
14.7\%) with a standard deviation of 33\%. A return of 0\% means the value of 
the portfolio doesn't change, a negative return means that the portfolio 
loses money, and a positive return means that the portfolio gains money.
\begin{parts}
\item What percent of years does this portfolio lose money, i.e. have a 
return less than 0\%?
\item What is the cutoff for the highest 15\% of annual returns with this 
portfolio?
\end{parts}
}{}

% 9

\eoce{\qt{LA weather, Part II\label{la_weather_unit_change}} 
Exercise~\ref{la_weather_intro} states that average daily high temperature in 
June in LA is 77\degree F with a standard deviation of 5\degree F, and it can 
be assumed that they to follow a normal distribution. We use the following 
equation to convert \degree F (Fahrenheit) to \degree C (Celsius):
\[ C = (F - 32) \times \frac{5}{9}. \]
\begin{parts}
\item Write the probability model for the distribution of temperature in 
\degree C in June in LA.
\item What is the probability of observing a 28\degree C (which roughly 
corresponds to 83\degree F) temperature or higher in June in LA? Calculate 
using the \degree C model from part (a).
\item Did you get the same answer or different answers in part (b) of this 
question and part (a) of Exercise~\ref{la_weather_intro}? Are you surprised? Explain.
\item Estimate the IQR of the temperatures (in \degree C) in June in LA.
\end{parts}
}{}

% 10

\eoce{\qt{Find the SD\label{find_sd}} Find the standard deviation of the 
distribution in the following situations.
\begin{parts}
\item  MENSA is an organization whose members have IQs in the top 2\% of the 
population. IQs are normally distributed with mean 100, and the minimum IQ 
score required for admission to MENSA is 132.
\item Cholesterol levels for women aged 20 to 34 follow an approximately 
normal distribution with mean 185 milligrams per deciliter (mg/dl). Women 
with cholesterol levels above 220 mg/dl are considered to have high 
cholesterol and about 18.5\% of women fall into this category.
\end{parts}
}{}
}




%%_________________
%\section{Evaluating the normal approximation}
%\label{assessingNormal}
%
%Many processes can be well approximated by the normal distribution.
%We have already seen two good examples:
%SAT scores and the heights of US adult males.
%While using a normal model can be extremely convenient
%and helpful, it is important to remember normality is
%always an approximation.
%Evaluating the appropriateness of the normal assumption
%is a key step in many data analyses.
%
%\index{normal probability plot|(}
%
%Example~\ref{normalExam40Perc} in Section~\ref{normalDist}
%suggested the distribution of heights of US males is well
%approximated by the normal model.
%We are interested in proceeding under the assumption that
%the data are normally distributed, but first we must check
%to see if this is reasonable.
%
%There are two visual methods for checking the assumption of
%normality, which can be implemented and interpreted quickly.
%The first is a simple histogram with the best fitting normal
%curve overlaid on the plot, as shown in the left panel of
%Figure~\ref{fcidMHeights}.
%The sample mean $\bar{x}$ and standard deviation $s$ are used
%as the parameters of the best fitting normal curve.
%The closer this curve fits the histogram, the more reasonable
%the normal model assumption.
%Another common method is examining a
%\term{normal probability plot},\footnote{Also commonly
%  called a \term{quantile-quantile plot}.}
%shown in the right panel of Figure~\ref{fcidMHeights}.
%The closer the points are to a perfect straight line,
%the more confident we can be that the data follow the
%normal model.
%
%\begin{figure}[h]
%  \centering
%  \Figure{0.7}{fcidMHeights}
%  \caption{A sample of 100 male heights.
%      The observations are rounded to the nearest whole inch,
%      explaining why the points appear to jump in increments
%      in the normal probability plot.}
%  \label{fcidMHeights}
%\end{figure}
%
%\begin{examplewrap}
%\begin{nexample}{Three data sets of 40, 100, and 400
%    samples were simulated from a normal distribution,
%    and the histograms and normal probability plots
%    of the data sets are shown in Figure~\ref{normalExamples}.
%    These will provide a benchmark for what to look for
%    in plots of real data.}
%  \label{normalExamplesExample}%
%  The left panels show the histogram (top) and normal
%  probability plot (bottom) for the simulated data set
%  with 40 observations.
%  The data set is too small to really see clear structure
%  in the histogram.
%  The normal probability plot also reflects this,
%  where there are some deviations from the line.
%  We should expect deviations of this amount for
%  such a small data set.
%
%  The middle panels show diagnostic plots for the
%  data set with 100 simulated observations.
%  The histogram shows more normality and the normal
%  probability plot shows a better fit.
%  While there are a few observations that deviate
%  noticeably from the line, they are not particularly
%  extreme.
%
%  The data set with 400 observations has a histogram
%  that greatly resembles the normal distribution,
%  while the normal probability plot is nearly a perfect
%  straight line.
%  Again in the normal probability plot there is one
%  observation (the largest) that deviates slightly from
%  the line.
%  If that observation had deviated 3 times further from
%  the line, it would be of greater importance in a real
%  data set.
%  Apparent outliers can occur in normally distributed
%  data but they are rare.
%
%  Notice the histograms look more normal as the sample
%  size increases, and the normal probability plot becomes
%  straighter and more stable.
%\end{nexample}
%\end{examplewrap}
%
%\begin{figure}
%  \centering
%  \Figure{0.9}{normalExamples}
%  \caption{Histograms and normal probability plots for
%      three simulated normal data sets; $n=40$ (left),
%      $n=100$ (middle), $n=400$ (right).}
%  \label{normalExamples}
%\end{figure}
%
%\begin{examplewrap}
%\begin{nexample}{Are NBA player heights normally distributed?
%    Consider all 494 NBA players presented in
%    Figure~\ref{nbaNormal}.}
%  We first create a histogram and normal probability plot
%  of the NBA player heights.
%  The histogram in the left panel appears to have too few
%  observations at the upper end since the curve is notably
%  above the histogram.
%  The points in the normal probability plot
%  follow a straight line for much of the center of the
%  distribution, and then deviates more at the upper values.
%  We can compare these characteristics to the sample of
%  400 normally distributed observations in
%  Example~\ref{normalExamplesExample} and see that they
%  represent much stronger deviations from the normal model.
%  NBA player heights do not appear to come from a normal
%  distribution.
%\end{nexample}
%\end{examplewrap}
%
%\begin{examplewrap}
%\begin{nexample}{Can we approximate poker winnings by a normal distribution? We consider the poker winnings of an individual over 50 days. A histogram and normal probability plot of these data are shown in Figure~\ref{pokerNormal}.}
%The data are very strongly right skewed\index{skew!example: very strong} in the histogram, which corresponds to the very strong deviations on the upper right component of the normal probability plot. If we compare these results to the sample of 40 normal observations in Example~\ref{normalExamplesExample}, it is apparent that these data show very strong deviations from the normal model.
%\end{nexample}
%\end{examplewrap}
%
%\begin{figure}
%  \centering
%  \Figure{0.8}{nbaNormal}
%  \caption{Histogram and normal probability plot
%      for the NBA heights from the 2008-9 season.}
%  \label{nbaNormal}
%\end{figure}
%
%\begin{figure}
%  \centering
%  \Figure{0.9}{pokerNormal}
%  \caption{A histogram of poker data with the best
%      fitting normal plot and a normal probability plot.}
%  \label{pokerNormal}
%\end{figure}
%
%\begin{exercisewrap}
%\begin{nexercise}\label{normalQuantileExercise}%
%Determine which data sets represented in
%Figure~\ref{normalQuantileExer} plausibly come from
%a nearly normal distribution.
%Are you confident in all of your conclusions?
%There are 100 (top left), 50 (top right), 500 (bottom left),
%and 15 points (bottom right) in the four plots.\footnotemark{}
%\end{nexercise}
%\end{exercisewrap}
%\footnotetext{Answers may vary a little.
%  The top-left plot shows some deviations in the smallest values
%  in the data set;
%  specifically, the left tail of the data set has some outliers
%  we should be wary of.
%  The top-right and bottom-left plots do not show any obvious
%  or extreme deviations from the lines for their respective
%  sample sizes, so a normal model would be reasonable for these
%  data sets.
%  The bottom-right plot has a consistent curvature that suggests
%  it is not from the normal distribution.
%  If we examine just the vertical coordinates of these
%  observations, we see that there is a lot of data between
%  -20 and 0, and then about five observations scattered
%  between 0 and 70.
%  This describes a distribution that has a strong right skew.}
%
%\begin{figure}
%  \centering
%  \Figure{0.7}{normalQuantileExer}
%  \caption{Four normal probability plots for
%      Guided Practice~\ref{normalQuantileExercise}.}
%  \label{normalQuantileExer}
%\end{figure}
%
%\begin{exercisewrap}
%\begin{nexercise}
%\label{normalQuantileExerciseAdditional}%
%Figure~\ref{normalQuantileExerAdditional} shows normal
%probability plots for two distributions that are skewed.
%One distribution is skewed to the low end (left skewed)
%and the other to the high end (right skewed).
%Which is which?\footnotemark{}
%\end{nexercise}
%\end{exercisewrap}
%\footnotetext{Examine where the points fall along the
%  vertical axis.
%  In the first plot, most points are near the low end
%  with fewer observations scattered along the high end;
%  this describes a distribution that is skewed to the
%  high end.
%  The second plot shows the opposite features,
%  and this distribution is skewed to the low end.}
%
%\begin{figure}[h]
%  \centering
%  \Figures{0.8}{normalQuantileExer}{normalQuantileExerAdditional}
%  \caption{Normal probability plots for
%      Guided Practice~\ref{normalQuantileExerciseAdditional}.}
%  \label{normalQuantileExerAdditional}
%\end{figure}
%
%\index{normal probability plot|)}
\index{normal distribution|)}
\index{distribution!normal|)}




%_________________
\section{Geometric distribution}
\label{geomDist}

How long should we expect to flip a coin until it turns up \resp{heads}? Or how many times should we expect to roll a die until we get a \resp{1}? These questions can be answered using the geometric distribution. We first formalize each trial -- such as a single coin flip or die toss -- using the Bernoulli distribution, and then we combine these with our tools from probability (Chapter~\ref{probability}) to construct the geometric distribution.

\subsection{Bernoulli distribution}
\label{bernoulli}

\newcommand{\insureSprob}{0.7}
\newcommand{\insureSperc}{70\%}
\newcommand{\insureFprob}{0.3}
\newcommand{\insureFperc}{30\%}
\newcommand{\insureDistA}{0.7}
\newcommand{\insureDistB}{0.21}
\newcommand{\insureDistC}{0.063}
\newcommand{\insureDistD}{0.019}
\newcommand{\insureDistE}{0.006}
\newcommand{\insureCDistA}{0.7}
\newcommand{\insureCDistB}{0.91}
\newcommand{\insureCDistC}{0.973}
\newcommand{\insureCDistCComplement}{0.027}
\newcommand{\insureCDistD}{0.992}
\newcommand{\insureCDistE}{0.998}
\newcommand{\insureGeomMean}{1.43}

\index{distribution!Bernoulli|(}

Many health insurance plans in the United States have
a deductible, where the insured individual is responsible
for costs up to the deductible, and then the costs above
the deductible are shared between the individual and
insurance company for the remainder of the year.

Suppose a health insurance company found that \insureSperc{} of the
people they insure stay below their deductible in any given year.
Each of these people can be thought of as a \term{trial}.
We label a person a \term{success} if her healthcare costs
do not exceed the deductible.
We label a person a \term{failure} if she does exceed her
deductible in the year.
Because 70\% of the individuals will not hit their deductible,
we denote the \term{probability of a success} as
$p = \insureSprob{}$.
The probability of a failure is sometimes denoted with
$q = 1 - p$, which would be \insureFprob{} in for the insurance
example.

When an individual trial only has two possible outcomes, often
labeled as \resp{success} or \resp{failure}, it is called a
\termsub{Bernoulli random variable}{distribution!Bernoulli}.
We chose to label a person who does not hit her deductible
as a ``success'' and all others as ``failures''.
However, we could just as easily have reversed these labels.
The mathematical framework we will build does not depend
on which outcome is labeled a success and which a failure,
as long as we are consistent.

Bernoulli random variables are often denoted as \resp{1}
for a success and \resp{0} for a failure.
In addition to being convenient in entering data,
it is also mathematically handy.
Suppose we observe ten trials:
\begin{center}
\resp{1} \resp{1} \resp{1} \resp{0} \resp{1} \resp{0} \resp{0} \resp{1} \resp{1} \resp{0}
\end{center}
Then the \term{sample proportion}, $\hat{p}$, is the
sample mean of these observations:
\begin{align*}
\hat{p} = \frac{\text{\# of successes}}{\text{\# of trials}}
    = \frac{1+1+1+0+1+0+0+1+1+0}{10} = 0.6
\end{align*}%
This mathematical inquiry of Bernoulli random variables can
be extended even further.
%\Comment{Maybe the next footnote should instead be an EOCE?}
Because \resp{0} and \resp{1} are numerical outcomes,
we can define the {mean} and {standard deviation}
of a Bernoulli random variable.
(See Exercises~\ref{bernoulli_mean_derivation}
and~\ref{bernoulli_sd_derivation}.)

\begin{onebox}{Bernoulli random variable}
%  A Bernoulli random variable has exactly two possible
%  outcomes, often labeled \resp{1} for the ``success''
%  outcome and \resp{0} for the ``failure'' outcome.\vspace{3mm}
  If $X$ is a random variable that takes value 1 with
  probability of success $p$ and 0 with probability $1-p$,
  then $X$ is a Bernoulli random variable with mean
  and standard deviation
  \begin{align*}
  \mu &= p
      &\sigma&= \sqrt{p(1-p)}
  \end{align*}
\end{onebox}

In general, it is useful to think about a Bernoulli random variable as a random process with only two outcomes: a success or failure. Then we build our mathematical framework using the numerical labels \resp{1} and \resp{0} for successes and failures, respectively.

\index{distribution!Bernoulli|)}


\D{\newpage}

\subsection{Geometric distribution}

\index{distribution!geometric|(}

The \termsub{geometric distribution}{distribution!geometric}
is used to describe how
many trials it takes to observe a success.
Let's first look at an example.

\begin{examplewrap}
\begin{nexample}{Suppose we are working at the insurance
    company and need to find a case where the person did
    not exceed her (or his) deductible as a case study.
    If the probability a person will not exceed her
    deductible is \insureSprob{} and we are drawing people
    at random, what are the chances that the first person
    will not have exceeded her deductible, i.e. be a success?
    The second person?
    The third?
    What about we pull $n - 1$ cases before we find
    the first success, i.e. the first success is the
    $n^{th}$ person?
    (If the first success is the fifth person, then we say $n=5$.)}
  \label{waitForDeductible}%
  The probability of stopping after the first person is just
  the chance the first person will not hit her (or his)
  deductible:~\insureSprob{}.
  The probability the second person is the first to hit
  her deductible is
  \begin{align*}
  &P(\text{second person is the first to hit deductible}) \\
  &\quad
    = P(\text{the first won't, the second will})
    = (\insureFprob{})(\insureSprob{})
    = \insureDistB{}
  \end{align*}
  Likewise, the probability it will be the third case is
  $(\insureFprob{})(\insureFprob{})(\insureSprob{})
    = \insureDistC$.

  If the first success is on the $n^{th}$ person,
  then there are $n-1$ failures and finally 1 success,
  which corresponds to the probability
  $(\insureFprob{})^{n-1}(\insureSprob{})$.
  This is the same as
  $(1-\insureSprob{})^{n-1}(\insureSprob{})$.
\end{nexample}
\end{examplewrap}

Example~\ref{waitForDeductible} illustrates what the
\termsub{geometric distribution}{distribution!geometric},
which describes the waiting
time until a success for
\term{independent and identically distributed (iid)}
Bernoulli random variables.
In this case, the \emph{independence} aspect just means
the individuals in the example don't affect each other,
and \emph{identical} means they each have the same probability
of success.

The geometric distribution from Example~\ref{waitForDeductible} is shown in Figure~\ref{geometricDist70}. In general, the probabilities for a geometric distribution decrease \term{exponentially} fast.

\begin{figure}[h]
  \centering
  \Figure{0.8}{geometricDist70}
  \caption{The geometric distribution when the probability
      of success is $p = \insureSprob{}$.}
  \label{geometricDist70}
\end{figure}

While this text will not derive the formulas for the mean (expected) number of trials needed to find the first success or the standard deviation or variance of this distribution, we present general formulas for each.

\begin{onebox}{Geometric Distribution}
  \index{distribution!geometric|textbf}%
  If the probability of a success in one trial is $p$
  and the probability of a failure is $1-p$, then the
  probability of finding the first success in the
  $n^{th}$ trial is given by\vspace{-1.5mm}
  \begin{align*}
  (1-p)^{n-1}p
  \end{align*}
  The mean (i.e. expected value), variance,
  and standard deviation of this wait time are given by
  \begin{align*}
  \mu &= \frac{1}{p}
      &\sigma^2 &=\frac{1-p}{p^2}
      &\sigma &= \sqrt{\frac{1-p}{p^2}}
  \end{align*}
\end{onebox}

It is no accident that we use the symbol $\mu$ for both the mean and expected value. The mean and the expected value are one and the same.

It takes, on average, $1/p$ trials to get a success under the geometric distribution. This mathematical result is consistent with what we would expect intuitively. If the probability of a success is high (e.g. 0.8), then we don't usually wait very long for a success: $1/0.8 = 1.25$ trials on average. If the probability of a success is low (e.g. 0.1), then we would expect to view many trials before we see a success: $1/0.1 = 10$ trials.

\begin{exercisewrap}
\begin{nexercise}
The probability that a particular case would not exceed their
deductible is said to be \insureSprob{}.
If we were to examine cases until we found one that where
the person did not hit her deductible, how many cases should
we expect to check?\footnotemark{}
\end{nexercise}
\end{exercisewrap}
\footnotetext{We would expect to see about
    $1 / \insureSprob{} \approx \insureGeomMean{}$
    individuals to find the first success.}

\begin{examplewrap}
\begin{nexample}{What is the chance that we would find
    the first success within the first 3 cases?}
  \label{insureFirstSuccessInLT4}%
  This is the chance it is the first ($n=1$), second ($n=2$),
  or third ($n=3$) case is the first success, which are three
  disjoint outcomes.
  Because the individuals in the sample are randomly sampled
  from a large population, they are independent.
  We compute the probability of each case and add the separate
  results:
  \begin{align*}
  &P(n=1, 2, \text{ or }3) \\
    & \quad = P(n=1)+P(n=2)+P(n=3) \\
    & \quad = (\insureFprob{})^{1-1}(\insureSprob{})
        + (\insureFprob{})^{2-1}(\insureSprob{})
        + (\insureFprob{})^{3-1}(\insureSprob{}) \\
    & \quad = \insureCDistC{}
  \end{align*}
  There is a probability of \insureCDistC{} that we would
  find a successful case within 3 cases.
\end{nexample}
\end{examplewrap}

\begin{exercisewrap}
\begin{nexercise}
Determine a more clever way to solve Example~\ref{insureFirstSuccessInLT4}.
Show that you get the same result.\footnotemark{}
\end{nexercise}
\end{exercisewrap}
\footnotetext{First find the probability of the complement:
  $P($no success in first 3~trials$)
      = \insureFprob{}^3 = \insureCDistCComplement{}$.
  Next, compute one minus this probability:
  $1 - P($no success in 3 trials$)
      = 1 - \insureCDistCComplement{}
      = \insureCDistC{}$.}

\D{\newpage}

\begin{examplewrap}
\begin{nexample}{Suppose a car insurer has determined
    that 88\% of its drivers will not exceed their deductible
    in a given year.
    If someone at the company were to randomly draw
    driver files until they found one that had not exceeded
    their deductible, what is the expected number of drivers
    the insurance employee must check?
    What is the standard deviation of the number of driver files
    that must be drawn?}
  \label{carInsure08DrawOne}%
  In this example, a success is again when someone will not
  exceed the insurance deductible, which has probability
  $p = 0.88$.
  The expected number of people to be checked is
  $1 / p = 1 / 0.88 = 1.14$ and the standard deviation is
  $\sqrt{(1-p)/p^2} = 0.39$.
\end{nexample}
\end{examplewrap}

\begin{exercisewrap}
\begin{nexercise}
Using the results from Example~\ref{carInsure08DrawOne},
$\mu = 1.14$ and $\sigma = 0.39$, would it be appropriate
to use the normal model to find what proportion
of experiments would end in 3 or fewer trials?\footnotemark{}
\end{nexercise}
\end{exercisewrap}
\footnotetext{No. The geometric distribution is always
  right skewed and can never be well-approximated by the
  normal model.}

The independence assumption is crucial to the geometric
distribution's accurate description of a scenario.
Mathematically, we can see that to construct the probability
of the success on the $n^{th}$ trial, we had to use the
Multiplication Rule for Independent Processes.
It is no simple task to generalize the geometric model
for dependent trials.

\index{distribution!geometric|)}


{


%_______________
\newpage\subsection*{Exercises} % Geometric distribution

% 1

\eoce{\qtq{Is it Bernoulli\label{is_it_bernouilli}} Determine if each trial can be 
considered an independent Bernoulli trial for the following situations.
\begin{parts}
\item Cards dealt in a hand of poker.
\item Outcome of each roll of a die.
\end{parts}
}{}

% 2

\eoce{\qt{With and without replacement\label{with_without_replacement}} In the 
following situations assume that half of the specified population is male and 
the other half is female.
\begin{parts}
\item Suppose you're sampling from a room with 10 people. What is the 
probability of sampling two females in a row when sampling with replacement? 
What is the probability when sampling without replacement?
\item Now suppose you're sampling from a stadium with 10,000 people. What is 
the probability of sampling two females in a row when sampling with 
replacement? What is the probability when sampling without replacement?
\item We often treat individuals who are sampled from a large population as 
independent. Using your findings from parts~(a) and~(b), explain whether or 
not this assumption is reasonable.
\end{parts}
}{}

% 3

\eoce{\qt{Eye color, Part I\label{eye_color_geometric}} A husband and wife both 
have brown eyes but carry genes that make it possible for their children to 
have brown eyes (probability 0.75), blue eyes (0.125), or green eyes (0.125).
\begin{parts}
\item What is the probability the first blue-eyed child they have is their 
third child? Assume that the eye colors of the children are independent of 
each other.
\item On average, how many children would such a pair of parents have before 
having a blue-eyed child? What is the standard deviation of the number of 
children they would expect to have until the first blue-eyed child?
\end{parts}
}{}

% 4

\eoce{\qt{Defective rate\label{defective_rate}} A machine that produces a special 
type of transistor (a component of computers) has a 2\% defective rate. The 
production is considered a random process where each transistor is 
independent of the others.
\begin{parts}
\item What is the probability that the $10^{th}$ transistor produced is the 
first with a defect?
\item What is the probability that the machine produces no defective 
transistors in a batch of 100?
\item On average, how many transistors would you expect to be produced before 
the first with a defect? What is the standard deviation?
\item Another machine that also produces transistors has a 5\% defective rate 
where each transistor is produced independent of the others. On average how 
many transistors would you expect to be produced with this machine before the 
first with a defect? What is the standard deviation?
\item Based on your answers to parts (c) and (d), how does increasing the 
probability of an event affect the mean and standard deviation of the wait 
time until success?
\end{parts}
}{}
}





\section{Binomial distribution}
\label{binomialModel}

\index{distribution!binomial|(}

The \termsub{binomial distribution}{distribution!binomial}
is used to describe
the number of successes in a fixed number of trials.
%,
%and this distribution is occasionally used in statistics,
%especially when doing more careful analysis of samples
%of data where simpler tools are not helpful.
This is different from the geometric distribution,
which described the number of trials we must wait before
we observe a success.


\subsection{The binomial distribution}

%\newcommand{\insureSprob}{0.7}
%\newcommand{\insureSperc}{70\%}
%\newcommand{\insureFprob}{0.3}
%\newcommand{\insureFperc}{30\%}
%\newcommand{\insureDistA}{0.7}
%\newcommand{\insureDistB}{0.21}
%\newcommand{\insureDistC}{0.063}
%\newcommand{\insureDistD}{0.019}
%\newcommand{\insureDistE}{0.006}
%\newcommand{\insureCDistA}{0.7}
%\newcommand{\insureCDistB}{0.91}
%\newcommand{\insureCDistC}{0.973}
%\newcommand{\insureCDistCComplement}{0.027}
%\newcommand{\insureCDistD}{0.992}
%\newcommand{\insureCDistE}{0.998}
%\newcommand{\insureGeomMean}{1.43}
\newcommand{\insureS}{\resp{not}}
\newcommand{\insureF}{\resp{exceed}}
% Doesn't consider binomial coefficient in next calculated value.
\newcommand{\insureBinomCinDSingleScenario}{0.103}
\newcommand{\insureBinomCinD}{0.412}
\newcommand{\insureBinomEinHSingleScenario}{0.00454}
\newcommand{\insureBinomEinH}{0.254}
\newcommand{\insureBinomFourtyExpValue}{28}
\newcommand{\insureBinomFourtySD}{2.9}
\newcommand{\insureBinomFourtyLower}{22}
\newcommand{\insureBinomFourtyUpper}{34}

\noindent%
Let's again imagine ourselves back at the insurance agency
where \insureSperc{} of individuals do not exceed their
deductible.

\begin{examplewrap}
\begin{nexample}{Suppose the insurance agency is considering
    a random sample of four individuals they insure.
    What is the chance exactly one of them will exceed
    the deductible and the other four will not?
    Let's call the four people
    Ariana ($A$),
    Brittany ($B$),
    Carlton ($C$),
    and Damian ($D$)
    for convenience.}
  \label{insureOneOfFourExceedsDeductible}%
  Let's consider a scenario where one person exceeds
  the deductible:
  \begin{align*}
  &P(A=\text{\insureF{}},
      \text{ }B=\text{\insureS{}},
      \text{ }C=\text{\insureS{}},
      \text{ }D=\text{\insureS{}}) \\
    &\quad = P(A=\text{\insureF{}})\ 
        P(B=\text{\insureS{}})\ 
        P(C=\text{\insureS{}})\ 
        P(D=\text{\insureS{}}) \\
    &\quad =  (\insureFprob{})
        (\insureSprob{})
        (\insureSprob{})
        (\insureSprob{}) \\
    &\quad = (\insureSprob{})^3 (\insureFprob{})^1 \\
    &\quad = \insureBinomCinDSingleScenario{}
  \end{align*}
  But there are three other scenarios: Brittany, Carlton,
  or Damian could have been the one to exceed the deductible.
  In each of these cases, the probability is again
  $(\insureSprob{})^3 (\insureFprob{})^1$.
  These four scenarios exhaust all the possible ways that
  exactly one of these four people could have exceeded
  the deductible, so the total probability is
  $4 \times (\insureSprob{})^3 (\insureFprob{})^1
      = \insureBinomCinD{}$.
\end{nexample}
\end{examplewrap}

\begin{exercisewrap}
\begin{nexercise}
Verify that the scenario where Brittany is the only one
exceed the deductible has probability
$(\insureSprob{})^3 (\insureFprob{})^1$.~\footnotemark{}
\end{nexercise}
\end{exercisewrap}
\footnotetext{
  $P(A=\text{\insureS{}},
      \text{ }B=\text{\insureF{}},
      \text{ }C=\text{\insureS{}},
      \text{ }D=\text{\insureS{}})
    = (\insureSprob{})(\insureFprob{})
        (\insureSprob{})(\insureSprob{})
    = (\insureSprob{})^3 (\insureFprob{})^1$.}


The scenario outlined in Example~\ref{insureOneOfFourExceedsDeductible} is an
example of a binomial distribution scenario.
The \termsub{binomial distribution}{distribution!binomial}
describes the probability of having exactly $k$ successes
in $n$ independent Bernoulli trials with probability
of a success $p$
(in Example~\ref{insureOneOfFourExceedsDeductible},
$n=4$, $k=3$, $p=\insureSprob{}$).
We would like to determine the probabilities associated
with the binomial distribution more generally,
i.e. we want a formula where we can use $n$, $k$, and $p$
to obtain the probability.
To do this, we reexamine each part of
Example~\ref{insureOneOfFourExceedsDeductible}.

There were four individuals who could have been the one
to exceed the deductible, and each of these four scenarios
had the same probability.
Thus, we could identify the final probability as
\begin{align*}
[\text{\# of scenarios}] \times P(\text{single scenario})
\end{align*}
The first component of this equation is the number of ways
to arrange the $k=3$ successes among the $n=4$ trials.
The second component is the probability of any of the four
(equally probable) scenarios.

\D{\newpage}

Consider $P($single scenario$)$ under the general case of
$k$ successes and $n-k$ failures in the $n$ trials.
In any such scenario, we apply the Multiplication Rule
for independent events:
\begin{align*}
p^k (1 - p)^{n - k}
\end{align*}
This is our general formula for $P($single scenario$)$.

Secondly, we introduce a general formula for the number
of ways to choose $k$ successes in $n$ trials,
i.e. arrange $k$ successes and $n - k$ failures:
\begin{align*}
{n\choose k} = \frac{n!}{k! (n - k)!}
\end{align*}
The quantity ${n\choose k}$ is read
\term{n choose k}.\footnote{Other notation for
  $n$ choose $k$ includes $_nC_k$, $C_n^k$, and $C(n,k)$.}
The exclamation point notation (e.g. $k!$) denotes
a \term{factorial} expression.\label{factorial_defined}
\begin{align*}
& 0! = 1 \\
& 1! = 1 \\
& 2! = 2\times1 = 2 \\
& 3! = 3\times2\times1 = 6 \\
& 4! = 4\times3\times2\times1 = 24 \\
& \vdots \\
& n! = n\times(n-1)\times...\times3\times2\times1
\end{align*}
Using the formula, we can compute the number of ways
to choose $k = 3$ successes in $n = 4$ trials:
\begin{align*}
{4 \choose 3} = \frac{4!}{3!(4-3)!}
  = \frac{4!}{3!1!} 
  = \frac{4\times3\times2\times1}{(3\times2\times1) (1)}
  = 4
\end{align*}
This result is exactly what we found by carefully thinking
of each possible scenario in
Example~\ref{insureOneOfFourExceedsDeductible}.

Substituting $n$ choose $k$ for the number of scenarios
and $p^k(1-p)^{n-k}$ for the single scenario probability
yields the general binomial formula.

\begin{onebox}{Binomial distribution}
  Suppose the probability of a single trial being
  a success is $p$.
  Then the probability of observing exactly $k$ successes
  in $n$ independent trials is given by\vspace{-1mm}
  \begin{align*}
  {n\choose k}p^k(1-p)^{n-k} = \frac{n!}{k!(n-k)!}p^k(1-p)^{n-k}
  \end{align*}
  The mean, variance, and standard deviation
  of the number of observed successes are\vspace{-2mm}
  \begin{align*}
  \mu &= np
  &\sigma^2 &= np(1-p)
  &\sigma&= \sqrt{np(1-p)}
  \end{align*}
\end{onebox}

\begin{onebox}{Is it binomial? Four conditions to check.}
  \label{isItBinomialTipBox}%
  (1) The trials are independent. \\
  (2) The number of trials, $n$, is fixed. \\
  (3) Each trial outcome can be classified as a \emph{success}
      or \emph{failure}. \\
  (4) The probability of a success, $p$, is the same for
      each trial.
\end{onebox}

\D{\newpage}

\begin{examplewrap}
\begin{nexample}{What is the probability that 3 of 8 randomly
    selected individuals will have exceeded the insurance
    deductible, i.e. that 5 of 8 will not exceed the deductible?
    Recall that 70\% of individuals will not exceed the
    deductible.}
  We would like to apply the binomial model,
  so we check the conditions.
  The number of trials is fixed ($n = 8$) (condition 2)
  and each trial outcome can be classified as a success
  or failure (condition 3).
  Because the sample is random, the trials are independent
  (condition~1) and the probability of a success is the same
  for each trial (condition~4).

  In the outcome of interest, there are $k = 5$ successes
  in $n = 8$ trials (recall that a success is an individual
  who does \emph{not} exceed the deductible, and the
  probability of a success is $p = \insureSprob{}$.
  So the probability that 5 of 8 will not exceed the
  deductible and 3 will exceed the deductible is given by
  \begin{align*}
  { 8 \choose 5}(\insureSprob{})^5
  (1-\insureSprob{})^{8-5}
    &= \frac{8!}{5!(5-3)!}
        (\insureSprob{})^5(1-\insureSprob{})^{8-5} \\
    &= \frac{8!}{5!3!}
        (\insureSprob{})^5(\insureFprob{})^3
  \end{align*}
  Dealing with the factorial part:
  \begin{align*}
  \frac{8!}{5!3!}
    = \frac{8\times7\times6\times5\times4\times3\times2\times1}
        {(5\times4\times3\times2\times1)(3\times2\times1)}
    = \frac{8\times7\times6}{3\times2\times1}
    = 56
  \end{align*}
  Using $(\insureSprob{})^5(\insureFprob{})^3
    \approx \insureBinomEinHSingleScenario{}$,
  the final probability is about
  $56 \times \insureBinomEinHSingleScenario{}
    \approx \insureBinomEinH{}$.
\end{nexample}
\end{examplewrap}

\begin{onebox}{Computing binomial probabilities}
  The first step in using the binomial model is to check
  that the model is appropriate.
  The second step is to identify $n$, $p$, and $k$.
  As the last stage use software or the formulas
  to determine the probability, then interpret the results.%
  \vspace{3mm}

  If you must do calculations by hand, it's often useful
  to cancel out as many terms as possible in the top and
  bottom of the binomial coefficient.
\end{onebox}

\begin{exercisewrap}
\begin{nexercise}
If we randomly sampled 40 case files from the insurance agency
discussed earlier, how many of the cases would you expect to not
have exceeded the deductible in a given year?
What is the standard deviation of the number that would not
have exceeded the deductible?\footnotemark{}
\end{nexercise}
\end{exercisewrap}
\footnotetext{We are asked to determine the expected number
  (the mean) and the standard deviation, both of which can
  be directly computed from the formulas:
  $\mu = np = 40 \times \insureSprob{}
    = \insureBinomFourtyExpValue$
  and $\sigma = \sqrt{np(1-p)}
    = \sqrt{40\times \insureSprob{}\times \insureFprob{}}
    = \insureBinomFourtySD{}$.
  Because very roughly 95\% of observations fall within
  2~standard deviations of the mean
  (see Section~\ref{variability}), we would probably observe
  at least \insureBinomFourtyLower{}
  but fewer than \insureBinomFourtyUpper{} individuals
  in our sample who would not exceed the deductible.}

\begin{exercisewrap}
\begin{nexercise}
The probability that a random smoker will develop a severe
lung condition in his or her lifetime is about $0.3$.
If you have 4 friends who smoke, are the conditions for the
binomial model satisfied?\footnotemark{}
\end{nexercise}
\end{exercisewrap}
\footnotetext{One possible answer:
  if the friends know each other, then the independence
  assumption is probably not satisfied.
  For example, acquaintances may have similar smoking habits,
  or those friends might make a pact to quit together.}

\D{\newpage}

\begin{exercisewrap}
\begin{nexercise}
\label{noMoreThanOneFriendWSevereLungCondition}%
Suppose these four friends do not know each other
and we can treat them as if they were a random sample
from the population.
Is the binomial model appropriate?
What is the probability that\footnotemark{}
\begin{enumerate}[(a)]
\setlength{\itemsep}{0mm}
\item
    None of them will develop a severe lung condition?
\item
    One will develop a severe lung condition?
\item
    That no more than one will develop a severe lung condition?
\end{enumerate}
\end{nexercise}
\end{exercisewrap}
\footnotetext{To check if the binomial model is appropriate,
  we must verify the conditions.
  (i)~Since we are supposing we can treat the friends
  as a random sample, they are independent.
  (ii)~We have a fixed number of trials ($n=4$).
  (iii)~Each outcome is a success or failure.
  (iv)~The probability of a success is the same for each
  trials since the individuals are like a random sample
  ($p=0.3$ if we say a ``success'' is someone getting
  a lung condition, a morbid choice).
  Compute parts~(a) and~(b) using the binomial formula:
  $P(0)
    = {4 \choose 0} (0.3)^0 (0.7)^4
    = 1\times1\times0.7^4
    = 0.2401$,
  $P(1)
    = {4 \choose 1} (0.3)^1(0.7)^{3}
    = 0.4116$.
  Note: $0!=1$.
  Part~(c) can be computed as the sum of parts~(a) and~(b):
  $P(0) + P(1) = 0.2401 + 0.4116 = 0.6517$.
  That is, there is about a 65\% chance that no more than
  one of your four smoking friends will develop a severe
  lung condition.}

\begin{exercisewrap}
\begin{nexercise}
What is the probability that at least 2 of your 4 smoking
friends will develop a severe lung condition in their
lifetimes?\footnotemark{}
\end{nexercise}
\end{exercisewrap}
\footnotetext{The complement (no more than one will develop
  a severe lung condition) as computed in Guided
  Practice~\ref{noMoreThanOneFriendWSevereLungCondition}
  as 0.6517, so we compute one minus this value:~0.3483.}

\begin{exercisewrap}
\begin{nexercise}
Suppose you have 7 friends who are smokers and they can
be treated as a random sample of smokers.\footnotemark{}
\begin{enumerate}[(a)]
\setlength{\itemsep}{0mm}
\item
    How many would you expect to develop a severe lung
    condition, i.e. what is the mean?
\item
    What is the probability that at most 2 of your 7
    friends will develop a severe lung condition.
\end{enumerate}
\end{nexercise}
\end{exercisewrap}
\footnotetext{(a)~$\mu=0.3\times7 = 2.1$.
  (b)~$P($0, 1, or 2 develop severe lung condition$)
      = P(k=0) + P(k=1)+P(k=2) = 0.6471$.}

Next we consider the first term in the binomial probability,
$n$ choose $k$ under some special scenarios.

\begin{exercisewrap}
\begin{nexercise}
Why is it true that ${n \choose 0}=1$ and ${n \choose n}=1$
for any number $n$?\footnotemark{}
\end{nexercise}
\end{exercisewrap}
\footnotetext{Frame these expressions into words.
  How many different ways are there to arrange 0 successes
  and $n$ failures in $n$ trials?
  (1 way.)
  How many different ways are there to arrange $n$ successes
  and 0 failures in $n$ trials?
  (1 way.)}

\begin{exercisewrap}
\begin{nexercise}
How many ways can you arrange one success and $n-1$ failures
in $n$ trials?
How many ways can you arrange $n-1$ successes and one failure
in $n$ trials?\footnotemark{}
\end{nexercise}
\end{exercisewrap}
\footnotetext{One success and $n-1$ failures:
  there are exactly $n$ unique places we can put
  the success, so there are $n$ ways to arrange one
  success and $n-1$ failures.
  A~similar argument is used for the second question.
  Mathematically, we show these results by verifying
  the following two equations:
  \begin{align*}
  {n \choose 1} = n,
    \qquad {n \choose n-1} = n
  \end{align*}}


\newpage


\subsection{Normal approximation to the binomial distribution}
\label{normalApproxBinomialDistSubsection}

\index{distribution!binomial!normal approximation|(}

The binomial formula is cumbersome when the sample size ($n$) is large, particularly when we consider a range of observations. In some cases we may use the normal distribution as an easier and faster way to estimate binomial probabilities.

\newcommand{\smokeprop}{0.15}
\newcommand{\smokeperc}{15\%}
\newcommand{\smokepropcomp}{0.85}
\newcommand{\smokeperccomp}{85\%}
\newcommand{\smokex}{42}
\newcommand{\smokexplusone}{43}
\newcommand{\smoken}{400}
\newcommand{\smokelowertailbinom}{0.0054}
\newcommand{\smokemean}{60}
\newcommand{\smokemeancomp}{340}
\newcommand{\smokesd}{7.14}
\newcommand{\smokez}{-2.52}
\newcommand{\smokelowertailnormal}{0.0059}

\begin{examplewrap}
\begin{nexample}{Approximately \smokeperc{} of the
    US population smokes cigarettes.
    A local government believed their community had
    a lower smoker rate and commissioned a survey of
    400 randomly selected individuals.
    The survey found that only \smokex{} of the
    \smoken{} participants smoke cigarettes.
    If the true proportion of smokers in the community
    was really \smokeperc{}, what is the probability
    of observing \smokex{} or fewer smokers in a sample
    of \smoken{} people?}
  \label{exactBinomSmokerExSetup}%
  We leave the usual verification that the four conditions
  for the binomial model are valid as an exercise.

  The question posed is equivalent to asking,
  what is the probability of observing
  $k=0$, 1, 2, ..., or \smokex{} smokers in a sample of
  $n = \smoken{}$ when $p=\smokeprop{}$?
  We can compute these \smokexplusone{} different
  probabilities and add them together to find the answer:
  \begin{align*}
  &P(k=0\text{ or }k=1\text{ or }\cdots\text{ or } k=\smokex{}) \\
	&\qquad = P(k=0) + P(k=1) + \cdots + P(k=\smokex{}) \\
	&\qquad = \smokelowertailbinom{}
  \end{align*}
  If the true proportion of smokers in the community
  is $p=\smokeprop{}$, then the probability of observing
  \smokex{} or fewer smokers in a sample of $n=\smoken{}$
  is \smokelowertailbinom{}.
\end{nexample}
\end{examplewrap}

The computations in Example~\ref{exactBinomSmokerExSetup}
are tedious and long.
In general, we should avoid such work if an alternative method
exists that is faster, easier, and still accurate.
Recall that calculating probabilities of a range of values
is much easier in the normal model.
We might wonder, is it reasonable to use the normal model
in place of the binomial distribution?
Surprisingly, yes, if certain conditions are met.

\begin{exercisewrap}
\begin{nexercise}
Here we consider the binomial model when the probability
of a success is $p = 0.10$.
Figure~\ref{fourBinomialModelsShowingApproxToNormal}
shows four hollow histograms for simulated samples from
the binomial distribution using four different sample sizes:
$n = 10, 30, 100, 300$.
What happens to the shape of the distributions as the sample
size increases?
What distribution does the last hollow histogram
resemble?\footnotemark{}
\end{nexercise}
\end{exercisewrap}
\footnotetext{The distribution is transformed from a blocky
  and skewed distribution into one that rather resembles the
  normal distribution in last hollow histogram.}

\begin{figure}[h]
  \centering
  \Figure{0.92}{fourBinomialModelsShowingApproxToNormal}
  \caption{Hollow histograms of samples from the binomial
      model when $p = 0.10$.
      The sample sizes for the four plots are
      $n = 10$, 30, 100, and 300, respectively.}
  \label{fourBinomialModelsShowingApproxToNormal}
\end{figure}

\begin{onebox}{Normal approximation of the binomial distribution}
  The binomial distribution with probability of success
  $p$ is nearly normal when the sample size $n$ is
  sufficiently large that $np$ and $n(1-p)$ are both
  at least 10.
  The approximate normal distribution has parameters
  corresponding to the mean and standard deviation of
  the binomial distribution:\vspace{-1.5mm}
  \begin{align*}
  \mu &= np
      &\sigma& = \sqrt{np(1 - p)}
  \end{align*}
\end{onebox}

The normal approximation may be used when computing
the range of many possible successes.
For instance, we may apply the normal distribution to
the setting of Example~\ref{exactBinomSmokerExSetup}.

\D{\newpage}

\begin{examplewrap}
\begin{nexample}{How can we use the normal approximation
    to estimate the probability of observing \smokex{} or
    fewer smokers in a sample of \smoken{}, if the true
    proportion of smokers is $p = \smokeprop{}$?}
  \label{approxNormalForSmokerBinomEx}
  Showing that the binomial model is reasonable was a
  suggested exercise in Example~\ref{exactBinomSmokerExSetup}.
  We also verify that both $np$ and $n(1-p)$ are at least 10:
  \begin{align*}
  np &= \smoken{} \times \smokeprop{} = \smokemean{}
  &n (1 - p) &= \smoken{} \times \smokepropcomp{}
      = \smokemeancomp{}
  \end{align*}
  With these conditions checked, we may use the normal
  approximation in place of the binomial distribution
  using the mean and standard deviation from the binomial
  model:
  \begin{align*}
  \mu &= np = \smokemean{}
  &\sigma &= \sqrt{np(1 - p)} = \smokesd{}
  \end{align*}
  We want to find the probability of observing
  \smokex{} or fewer smokers using this model.
\end{nexample}
\end{examplewrap}

\begin{exercisewrap}
\begin{nexercise}
Use the normal model $N(\mu = \smokemean{}, \sigma = \smokesd{})$
to estimate the probability of observing \smokex{} or fewer
smokers.
Your answer should be approximately equal to the solution
of Example~\ref{exactBinomSmokerExSetup}:%
~\smokelowertailbinom{}.~\footnotemark{}
\end{nexercise}
\end{exercisewrap}
\footnotetext{Compute the Z-score first:
  $Z = \frac{\smokex{} - \smokemean{}}{\smokesd{}} = \smokez{}$.
  The corresponding left tail area is \smokelowertailnormal{}.}



\newpage


\subsection{The normal approximation breaks down on small intervals}

The normal approximation to the binomial distribution tends to perform poorly when estimating the probability of a small range of counts, even when the conditions are met.

\newcommand{\smokeA}{49}
\newcommand{\smokeB}{50}
\newcommand{\smokeC}{51}
\newcommand{\smokeABCBinom}{0.0649}
\newcommand{\smokeABCNormal}{0.0421}
\newcommand{\smokeABCNormalFixed}{0.0633}

Suppose we wanted to compute the probability of observing
\smokeA{}, \smokeB{}, or \smokeC{} smokers in \smoken{}
when $p = \smokeprop{}$.
With such a large sample, we might be tempted to apply
the normal approximation and use the range \smokeA{} to \smokeC{}.
However, we would find that the binomial solution and the normal
approximation notably differ:
\begin{align*}
\text{Binomial:}&\ \smokeABCBinom{}
&\text{Normal:}&\ \smokeABCNormal{}
\end{align*}
We can identify the cause of this discrepancy using
Figure~\ref{normApproxToBinomFail}, which shows the areas
representing the binomial probability (outlined) and normal
approximation (shaded).
Notice that the width of the area under the normal
distribution is 0.5 units too slim on both sides of
the interval.

\begin{figure}[h]
  \centering
  \Figure{1.0}{normApproxToBinomFail}
  \caption{A normal curve with the area between
      \smokeA{} and \smokeC{} shaded.
      The outlined area represents the exact binomial
      probability.}
  \label{normApproxToBinomFail}
\end{figure}

\begin{onebox}{Improving the normal approximation
    for the binomial distribution}
  The normal approximation to the binomial distribution
  for intervals of values is usually improved if cutoff
  values are modified slightly.
  The cutoff values for the lower end of a shaded region
  should be reduced by 0.5, and the cutoff value for the
  upper end should be increased by 0.5.
\end{onebox}

The tip to add extra area when applying the normal
approximation is most often useful when examining
a range of observations.
In the example above, the revised normal distribution
estimate is \smokeABCNormalFixed{}, much closer to the
exact value of \smokeABCBinom{}.
While it is possible to also apply this correction when
computing a tail area, the benefit of the modification
usually disappears since the total interval is typically
quite wide.

\index{distribution!binomial!normal approximation|)}
\index{distribution!binomial|)}


{


%_______________
\newpage\subsection*{Exercises} % Binomial distribution

% 1

\eoce{\qt{Underage drinking, Part I\label{underage_drinking_intro}}
Data collected by the Substance Abuse and Mental Health
Services Administration (SAMSHA) suggests that 69.7\% of
18-20 year olds consumed alcoholic beverages in any given
year.\footfullcite{webpage:alcohol}
\begin{parts}
\item Suppose a random sample of ten 18-20 year olds is taken. Is the use 
of the binomial distribution appropriate for calculating the probability that 
exactly six consumed alcoholic beverages? Explain.
\item Calculate the probability that exactly 6 out of 10 randomly sampled 18-
20 year olds consumed an alcoholic drink.
\item What is the probability that exactly four out of ten 18-20 year 
olds have \textit{not} consumed an alcoholic beverage?
\item What is the probability that at most 2 out of 5 randomly sampled 18-20 
year olds have consumed alcoholic beverages?
\item What is the probability that at least 1 out of 5 randomly sampled 18-20 
year olds have consumed alcoholic beverages?
\end{parts}
}{}

% 2

\eoce{\qt{Chicken pox, Part I\label{chicken_pox_intro}} The National Vaccine 
Information Center estimates that 90\% of Americans have had chickenpox by 
the time they reach adulthood.  \footfullcite{webpage:chickenpox}
\begin{parts}
\item Suppose we take a random sample of 100 American adults. Is the use of 
the binomial distribution appropriate for calculating the probability that exactly 97 
out of 100 randomly sampled American adults had chickenpox during childhood? Explain.
\item Calculate the probability that exactly 97 out of 100 randomly sampled 
American adults had chickenpox during childhood.
\item What is the probability that exactly 3 out of a new sample of 100 
American adults have \textit{not} had chickenpox in their childhood?
\item What is the probability that at least 1 out of 10 randomly sampled 
American adults have had chickenpox?
\item What is the probability that at most 3 out of 10 randomly sampled 
American adults have \textit{not} had chickenpox?
\end{parts}
}{}

% 3

\eoce{\qt{Underage drinking, Part II\label{underage_drinking_normal_approx}}
We learned in Exercise~\ref{underage_drinking_intro}
that about 70\% of 18-20 year olds consumed alcoholic
beverages in any given year. We now consider a random 
sample of fifty 18-20 year olds.
\begin{parts}
\item How many people would you expect to have consumed alcoholic beverages? 
And with what standard deviation?
\item Would you be surprised if there were 45 or more people who have 
consumed alcoholic beverages?
\item What is the probability that 45 or more people in this sample have 
consumed alcoholic beverages? How does this probability relate to your answer 
to part (b)?
\end{parts}
}{}

% 4

\eoce{\qt{Chickenpox, Part II\label{chicken_pox_normal_approx}} We learned in 
Exercise~\ref{chicken_pox_intro} that about 90\% of American adults had 
chickenpox before adulthood. We now consider a random sample of 120 American 
adults.
\begin{parts}
\item How many people in this sample would you expect to have had chickenpox 
in their childhood? And with what standard deviation?
\item Would you be surprised if there were 105 people who have had chickenpox 
in their childhood?
\item What is the probability that 105 or fewer people in this sample have 
had chickenpox in their childhood? How does this probability relate to your 
answer to part (b)?
\end{parts}
}{}

% 5

\eoce{\qt{Game of dreidel\label{dreidel}} A dreidel is a four-sided spinning top 
with the Hebrew letters \textit{nun}, \textit{gimel}, \textit{hei}, and 
\textit{shin}, one on each side. Each side is equally likely to come up in a 
single spin of the dreidel. Suppose you spin a dreidel three times. Calculate 
the probability of getting

\noindent\begin{minipage}[c]{0.38\textwidth}
\begin{parts}
\item at least one \textit{nun}? 
\item exactly 2 \textit{nun}s? 
\item exactly 1 \textit{hei}? 
\item at most 2 \textit{gimel}s? \vspace{3mm}
\end{parts}
\end{minipage}%
\begin{minipage}[c]{0.32\textwidth}
\includegraphics[width=0.9\textwidth]{ch_distributions/figures/eoce/dreidel/dreidel.jpg}
\end{minipage}%
\begin{minipage}[c]{0.28\textwidth}%
{\footnotesize Photo by Staccabees, cropped \\
  (\oiRedirect{textbook-flickr_staccabees_dreidels}{http://flic.kr/p/7gLZTf}) \\
  \oiRedirect{textbook-CC_BY_2}{CC~BY~2.0~license}} \\
\end{minipage}
}{}

% 6

\eoce{\qt{Arachnophobia\label{arachnophobia}}
A Gallup Poll found that 7\% of teenagers (ages 13 to 17)
suffer from arachnophobia and are extremely afraid of spiders.
At a summer camp there are 10 teenagers sleeping in each tent.
Assume that these 10 teenagers are independent of each other.%
\footfullcite{webpage:spiders}
\begin{parts}
\item Calculate the probability that at least one of them suffers from 
arachnophobia.
\item Calculate the probability that exactly 2 of them suffer from 
arachnophobia.
\item Calculate the probability that at most 1 of them suffers from 
arachnophobia. 
\item If the camp counselor wants to make sure no more than 1 teenager in 
each tent is afraid of spiders, does it seem reasonable for him to randomly 
assign teenagers to tents?
\end{parts}
}{}

% 7

\eoce{\qt{Eye color, Part II\label{eye_color_binomial}} 
Exercise~\ref{eye_color_geometric} introduces a husband and wife with brown 
eyes who have 0.75 probability of having children with brown eyes, 0.125 
probability of having children with blue eyes, and 0.125 probability of 
having children with green eyes.
\begin{parts}
\item What is the probability that their first child will have green eyes and 
the second will not?
\item What is the probability that exactly one of their two children will 
have green eyes?
\item If they have six children, what is the probability that exactly two 
will have green eyes?
\item If they have six children, what is the probability that at least one 
will have green eyes?
\item What is the probability that the first green eyed child will be the 
$4^{th}$ child? 
\item Would it be considered unusual if only 2 out of their 6 children had 
brown eyes?
\end{parts}
}{}

% 8

\eoce{\qt{Sickle cell anemia\label{sickle_cell_anemia}} Sickle cell anemia is a 
genetic blood disorder where red blood cells lose their flexibility and 
assume an abnormal, rigid, ``sickle" shape, which results in a risk of 
various complications. If both parents are carriers of the disease, then a 
child has a 25\% chance of having the disease, 50\% chance of being a 
carrier, and 25\% chance of neither having the disease nor being a carrier. 
If two parents who are carriers of the disease have 3 children, what is the 
probability that 
\begin{parts}
\item two will have the disease?
\item none will have the disease?
\item at least one will neither have the disease nor be a carrier?
\item the first child with the disease will the be $3^{rd}$ child?
\end{parts}
}{}

% 9

\eoce{\qt{Exploring permutations\label{explore_combinations}} The formula for the 
number of ways to arrange $n$ objects is $n! = n\times(n-1)\times \cdots 
\times 2 \times 1$. This exercise walks you through the derivation of this 
formula for a couple of special cases.

\indent A small company has five employees: Anna, Ben, Carl, Damian, and 
Eddy. There are five parking spots in a row at the company, none of which are 
assigned, and each day the employees pull into a random parking spot. That 
is, all possible orderings of the cars in the row of spots are equally likely.
\begin{parts}
\item On a given day, what is the probability that the employees park in 
alphabetical order?
\item If the alphabetical order has an equal chance of occurring relative to 
all other possible orderings, how many ways must there be to arrange the five 
cars?
\item Now consider a sample of 8 employees instead. How many possible ways 
are there to order these 8 employees' cars?
\end{parts}
}{}

% 10

\eoce{\qt{Male children\label{male_children}} While it is often assumed that the 
probabilities of having a boy or a girl are the same, the actual probability 
of having a boy is slightly higher at 0.51. Suppose a couple plans to have 3 
kids. 
\begin{parts}
\item Use the binomial model to calculate the probability that two of them 
will be boys.
\item Write out all possible orderings of 3 children, 2 of whom are boys. Use 
these scenarios to calculate the same probability from part (a) but using the 
addition rule for disjoint outcomes. Confirm that your answers from parts (a) 
and (b) match.
\item If we wanted to calculate the probability that a couple who plans to 
have 8 kids will have 3 boys, briefly describe why the approach from part (b) 
would be more tedious than the approach from part (a).
\end{parts}
}{}
}




%_________________
\section{Negative binomial distribution}
\label{negativeBinomial}

\index{distribution!negative binomial|(}

The geometric distribution describes the probability of observing the first success on the $n^{th}$ trial. The \termsub{negative binomial distribution}{distribution!negative binomial} is more general: it describes the probability of observing the $k^{th}$ success on the $n^{th}$ trial.

\begin{examplewrap}
\begin{nexample}{Each day a high school football coach tells his star kicker, Brian, that he can go home after he successfully kicks four 35 yard field goals. Suppose we say each kick has a probability $p$ of being successful. If $p$ is small -- e.g. close to 0.1 -- would we expect Brian to need many attempts before he successfully kicks his fourth field goal?}
We are waiting for the fourth success ($k=4$). If the probability of a success ($p$) is small, then the number of attempts ($n$) will probably be large. This means that Brian is more likely to need many attempts before he gets $k=4$ successes. To put this another way, the probability of $n$ being small is low.
\end{nexample}
\end{examplewrap}

To identify a negative binomial case, we check 4 conditions. The first three are common to the binomial distribution.

\begin{onebox}{Is it negative binomial? Four conditions to check}
(1) The trials are independent. \\
(2) Each trial outcome can be classified as a success or failure. \\
(3) The probability of a success ($p$) is the same for each trial. \\
(4) The last trial must be a success.
\end{onebox}

\begin{exercisewrap}
\begin{nexercise}
Suppose Brian is very diligent in his attempts and he makes each 35 yard field goal with probability $p=0.8$. Take a guess at how many attempts he would need before making his fourth kick.\footnotemark
\end{nexercise}
\end{exercisewrap}
\footnotetext{One possible answer: since he is likely to make each field goal attempt, it will take him at least 4 attempts but probably not more than 6 or 7.}

\begin{examplewrap}
\begin{nexample}{In yesterday's practice, it took Brian only 6 tries to get his fourth field goal. Write out each of the possible sequence of kicks.} \label{eachSeqOfSixTriesToGetFourSuccesses}
Because it took Brian six tries to get the fourth success, we know the last kick must have been a success. That leaves three successful kicks and two unsuccessful kicks (we label these as failures) that make up the first five attempts. There are ten possible sequences of these first five kicks, which are shown in Figure~\ref{successFailureOrdersForBriansFieldGoals}. If Brian achieved his fourth success ($k=4$) on his sixth attempt ($n=6$), then his order of successes and failures must be one of these ten possible sequences.
\end{nexample}
\end{examplewrap}

\begin{figure}[ht]
\newcommand{\succObs}[1]{{\color{oiB}$\stackrel{#1}{S}$}}
\centering
\begin{tabular}{c|c ccc cl | r}
\multicolumn{8}{c}{\hspace{10mm}Kick Attempt} \\
& & 1 & 2 & 3 & 4 & \multicolumn{2}{l}{5\hfill6} \\
\hline
1&& $F$ & $F$ & \succObs{1} & \succObs{2} & \succObs{3} & \succObs{4} \\
2&& $F$ & \succObs{1} & $F$ & \succObs{2} & \succObs{3} & \succObs{4} \\
3&& $F$ & \succObs{1} & \succObs{2} & $F$ & \succObs{3} & \succObs{4} \\
4&& $F$ & \succObs{1} & \succObs{2} & \succObs{3} & $F$ & \succObs{4} \\
5&& \succObs{1} & $F$ & $F$ & \succObs{2} & \succObs{3} & \succObs{4} \\
6&& \succObs{1} & $F$ & \succObs{2} & $F$ & \succObs{3} & \succObs{4} \\
7&& \succObs{1} & $F$ & \succObs{2} & \succObs{3} & $F$ & \succObs{4} \\
8&& \succObs{1} & \succObs{2} & $F$ & $F$ & \succObs{3} & \succObs{4} \\
9&& \succObs{1} & \succObs{2} & $F$ & \succObs{3} & $F$ & \succObs{4} \\
10&& \succObs{1} & \succObs{2} & \succObs{3} & $F$ & $F$ & \succObs{4} \\
\end{tabular}
\caption{The ten possible sequences when the fourth successful kick is on the sixth attempt.}
\label{successFailureOrdersForBriansFieldGoals}
\end{figure}

\begin{exercisewrap}
\begin{nexercise} \label{probOfEachSeqOfSixTriesToGetFourSuccesses}
Each sequence in Figure~\ref{successFailureOrdersForBriansFieldGoals} has exactly two failures and four successes with the last attempt always being a success. If the probability of a success is $p=0.8$, find the probability of the first sequence.\footnotemark
\end{nexercise}
\end{exercisewrap}
\footnotetext{The first sequence:
  $0.2 \times 0.2 \times 0.8 \times
      0.8 \times 0.8 \times 0.8
    = 0.0164$.}

\D{\newpage}

If the probability Brian kicks a 35 yard field goal is $p=0.8$, what is the probability it takes Brian exactly six tries to get his fourth successful kick? We can write this as
{\small\begin{align*}
&P(\text{it takes Brian six tries to make four field goals}) \\
& \quad = P(\text{Brian makes three of his first five field goals, and he makes the sixth one}) \\
& \quad = P(\text{$1^{st}$ sequence OR $2^{nd}$ sequence OR ... OR $10^{th}$ sequence})
\end{align*}
}where the sequences are from Figure~\ref{successFailureOrdersForBriansFieldGoals}. We can break down this last probability into the sum of ten disjoint possibilities:
{\small\begin{align*}
&P(\text{$1^{st}$ sequence OR $2^{nd}$ sequence OR ... OR $10^{th}$ sequence}) \\
&\quad = P(\text{$1^{st}$ sequence}) + P(\text{$2^{nd}$ sequence}) + \cdots + P(\text{$10^{th}$ sequence})
\end{align*}
}The probability of the first sequence was identified in Guided Practice~\ref{probOfEachSeqOfSixTriesToGetFourSuccesses} as 0.0164, and each of the other sequences have the same probability. Since each of the ten sequence has the same probability, the total probability is ten times that of any individual sequence.

The way to compute this negative binomial probability is similar to how the binomial problems were solved in Section~\ref{binomialModel}. The probability is broken into two pieces:
\begin{align*}
&P(\text{it takes Brian six tries to make four field goals}) \\
&= [\text{Number of possible sequences}] \times P(\text{Single sequence})
\end{align*}
Each part is examined separately, then we multiply to get the final result.

We first identify the probability of a single sequence. One particular case is to first observe all the failures ($n-k$ of them) followed by the $k$ successes:
\begin{align*}
&P(\text{Single sequence}) \\
&= P(\text{$n-k$ failures and then $k$ successes}) \\
&= (1-p)^{n-k} p^{k}
\end{align*}

\D{\newpage}

We must also identify the number of sequences for the general case. Above, ten sequences were identified where the fourth success came on the sixth attempt. These sequences were identified by fixing the last observation as a success and looking for all the ways to arrange the other observations. In other words, how many ways could we arrange $k-1$ successes in $n-1$ trials? This can be found using the $n$ choose $k$ coefficient but for $n-1$ and $k-1$ instead:
\begin{align*}
{n-1 \choose k-1} = \frac{(n-1)!}{(k-1)! \left((n-1) - (k-1)\right)!} = \frac{(n-1)!}{(k-1)! \left(n - k\right)!}
\end{align*}
This is the number of different ways we can order $k-1$ successes and $n-k$ failures in $n-1$ trials. If the factorial notation (the exclamation point) is unfamiliar, see page~\pageref{factorial_defined}.

\begin{onebox}{Negative binomial distribution}
  The negative binomial distribution describes the
  probability of observing the $k^{th}$ success on
  the $n^{th}$ trial, where all trials are independent:
  \begin{align*}
  P(\text{the $k^{th}$ success on the $n^{th}$ trial})
      = {n-1 \choose k-1} p^{k}(1-p)^{n-k}
  \end{align*}
  The value $p$ represents the probability that
  an individual trial is a success.
\end{onebox}

\begin{examplewrap}
\begin{nexample}{Show using the formula for the negative binomial distribution that the probability Brian kicks his fourth successful field goal on the sixth attempt is 0.164.}
The probability of a single success is $p=0.8$, the number of successes is $k=4$, and the number of necessary attempts under this scenario is $n=6$.
\begin{align*}
{n-1 \choose k-1}p^k(1-p)^{n-k}\ 
	=\ \frac{5!}{3!2!} (0.8)^4 (0.2)^2\ 
	=\ 10\times 0.0164\ 
	=\ 0.164
\end{align*}
\end{nexample}
\end{examplewrap}

\begin{exercisewrap}
\begin{nexercise}
The negative binomial distribution requires that each kick attempt by Brian is independent. Do you think it is reasonable to suggest that each of Brian's kick attempts are independent?\footnotemark
\end{nexercise}
\end{exercisewrap}
\footnotetext{Answers may vary. We cannot conclusively say they are or are not independent. However, many statistical reviews of athletic performance suggests such attempts are very nearly independent.}

\begin{exercisewrap}
\begin{nexercise}
Assume Brian's kick attempts are independent. What is the probability that Brian will kick his fourth field goal within 5 attempts?\footnotemark
\end{nexercise}
\end{exercisewrap}
\footnotetext{If his fourth field goal ($k=4$) is within five attempts, it either took him four or five tries ($n=4$ or $n=5$). We have $p=0.8$ from earlier. Use the negative binomial distribution to compute the probability of $n = 4$ tries and $n=5$ tries, then add those probabilities together:
\begin{align*}
& P(n=4\text{ OR }n=5) = P(n=4) + P(n=5) \\
&\quad = {4-1 \choose 4-1} 0.8^4 + {5-1 \choose 4-1} (0.8)^4(1-0.8) = 1\times 0.41 + 4\times 0.082 = 0.41 + 0.33 = 0.74
\end{align*}}

\D{\newpage}

\begin{onebox}{Binomial versus negative binomial}
  In the binomial case, we typically have a fixed number
  of trials and instead consider the number of successes.
  In the negative binomial case, we examine how many trials
  it takes to observe a fixed number of successes and
  require that the last observation be a success.
\end{onebox}

\begin{exercisewrap}
\begin{nexercise}
On 70\% of days, a hospital admits at least one heart attack patient. On 30\% of the days, no heart attack patients are admitted. Identify each case below as a binomial or negative binomial case, and compute the probability.\footnotemark
\begin{enumerate}[(a)]
\setlength{\itemsep}{0mm}
\item What is the probability the hospital will admit
    a heart attack patient on exactly three days this week?

\item What is the probability the second day with a heart
    attack patient will be the fourth day of the week?

\item What is the probability the fifth day of next month
    will be the first day with a heart attack patient?
\end{enumerate}
\end{nexercise}
\end{exercisewrap}
\footnotetext{In each part, $p=0.7$. (a) The number of days is fixed, so this is binomial. The parameters are $k=3$ and $n=7$: 0.097. (b) The last ``success'' (admitting a heart attack patient) is fixed to the last day, so we should apply the negative binomial distribution. The parameters are $k=2$, $n=4$: 0.132. (c) This problem is negative binomial with $k=1$ and $n=5$: 0.006. Note that the negative binomial case when $k=1$ is the same as using the geometric distribution.}

\index{distribution!negative binomial|)}


{\exercisesheader{}

% 25

\eoce{\qt{Rolling a die\label{roll_die}} Calculate the 
following probabilities and indicate which probability distribution model is 
appropriate in each case. You roll a fair die 5 times. What is the 
probability of rolling
\begin{parts}
\item the first 6 on the fifth roll?
\item exactly three 6s?
\item the third 6 on the fifth roll?
\end{parts}
}{}

% 26

\eoce{\qt{Playing darts\label{play_darts}} Calculate the following probabilities 
and indicate which probability distribution model is appropriate in each 
case. A very good darts player can hit the bull's eye (red circle in the 
center of the dart board) 65\% of the time. What is the probability that he
\begin{parts}
\item hits the bullseye for the $10^{th}$ time on the $15^{th}$ try?
\item hits the bullseye 10 times in 15 tries?
\item hits the first bullseye on the third try?
\end{parts}
}{}

% 27

\eoce{\qt{Sampling at school\label{sampling_at_school}} For a sociology class 
project you are asked to conduct a survey on 20 students at your school. You 
decide to stand outside of your dorm's cafeteria and conduct the survey on a 
random sample of 20 students leaving the cafeteria after dinner one evening. 
Your dorm is comprised of 45\% males and 55\% females.
\begin{parts}
\item Which probability model is most appropriate for calculating the 
probability that the $4^{th}$ person you survey is the $2^{nd}$ female? 
Explain.
\item Compute the probability from part (a).
\item The three possible scenarios that lead to $4^{th}$ person you survey 
being the $2^{nd}$ female are
\[ \{M, M, F, F\}, \{M, F, M, F\}, \{F, M, M, F\} \]
One common feature among these scenarios is that the last trial is always 
female. In the first three trials there are 2 males and 1 female. Use the 
binomial coefficient to confirm that there are 3 ways of ordering 2 males and 
1 female. 
\item Use the findings presented in part (c) to explain why the formula for 
the coefficient for the negative binomial is ${n-1 \choose k-1}$ while the 
formula for the binomial coefficient is ${n \choose k}$.
\end{parts}
}{}

% 28

\eoce{\qt{Serving in volleyball\label{serving_volleyball}} A not-so-skilled 
volleyball player has a 15\% chance of making the serve, which involves 
hitting the ball so it passes over the net on a trajectory such that it will 
land in the opposing team's court. Suppose that her serves are independent of 
each other.
\begin{parts}
\item What is the probability that on the $10^{th}$ try she will make her 
$3^{rd}$ successful serve?
\item Suppose she has made two successful serves in nine attempts. What is 
the probability that her $10^{th}$ serve will be successful?
\item Even though parts (a) and (b) discuss the same scenario, the 
probabilities you calculated should be different. Can you explain the reason 
for this discrepancy?
\end{parts}
}{}
}





%_________________
\section{Poisson distribution}
\label{poisson}

\index{distribution!Poisson|(}

\begin{examplewrap}
\begin{nexample}{There are about 8 million individuals
    in New York City.
    How many individuals might we expect to be hospitalized
    for acute myocardial infarction (AMI), i.e. a heart attack,
    each day?
    According to historical records, the average number is
    about 4.4 individuals.
    However, we would also like to know the approximate
    distribution of counts.
    What would a histogram of the number of AMI occurrences
    each day look like if we recorded the daily counts over
    an entire year?}
  \label{amiIncidencesEachDayOver1YearInNYCExample}%
  A histogram of the number of occurrences of AMI on 365 days
  for NYC is shown in
  Figure~\ref{amiIncidencesOver100Days}.\footnotemark{}
  The sample mean (4.38) is similar to the historical average
  of~4.4.
  The sample standard deviation is about 2, and the histogram
  indicates that about 70\% of the data fall between 2.4 and~6.4.
  The distribution's shape is unimodal and skewed to the right.
\end{nexample}
\end{examplewrap}
\footnotetext{These data are simulated. In practice, we should check for an association between successive days.}

\begin{figure}[h]
  \centering
  \Figure{0.6}{amiIncidencesOver100Days}
  \caption{A histogram of the number of occurrences
      of AMI on 365 separate days in NYC.}
  \label{amiIncidencesOver100Days}
\end{figure}

The \termsub{Poisson distribution}{distribution!Poisson} is often useful for estimating the number of events in a large population over a unit of time. For instance, consider each of the following events:
\begin{itemize}
\setlength{\itemsep}{0mm}
\item having a heart attack,
\item getting married, and
\item getting struck by lightning.
\end{itemize}
The Poisson distribution helps us describe the number of such events that will occur in a day for a fixed population if the individuals within the population are independent. The Poisson distribution could also be used over another unit of time, such as an hour or a~week.

The histogram in Figure~\ref{amiIncidencesOver100Days} approximates a Poisson distribution with rate equal to 4.4. The \term{rate} for a Poisson distribution is the average number of occurrences in a mostly-fixed population per unit of time. In Example~\ref{amiIncidencesEachDayOver1YearInNYCExample}, the time unit is a day, the population is all New York City residents, and the historical rate is 4.4. The parameter in the Poisson distribution is the rate -- or how many events we expect to observe -- and it is typically denoted by $\lambda$\index{Greek!lambda@lambda ($\lambda$)}
(the Greek letter \emph{lambda})  or $\mu$. Using the rate, we can describe the probability of observing exactly $k$ events in a single unit of time.

\D{\newpage}

\begin{onebox}{Poisson distribution}
  Suppose we are watching for events and the number
  of observed events follows a Poisson distribution
  with rate $\lambda$.
  Then
  \begin{align*}
  P(\text{observe $k$ events})
      = \frac{\lambda^{k} e^{-\lambda}}{k!}
  \end{align*}
  where $k$ may take a value 0, 1, 2, and so on,
  and $k!$ represents $k$-factorial, as described on
  page~\pageref{factorial_defined}.
  The letter $e\approx2.718$ is the base of the natural
  logarithm.
  The mean and standard deviation of this distribution
  are $\lambda$ and $\sqrt{\lambda}$, respectively.
\end{onebox}

We will leave a rigorous set of conditions for the Poisson distribution to a later course. However, we offer a few simple guidelines that can be used for an initial evaluation of whether the Poisson model would be appropriate.

A random variable may follow a Poisson distribution if we are looking for the number of events, the population that generates such events is large, and the events occur independently of each other.

Even when events are not really independent --
for instance, Saturdays and Sundays are especially
popular for weddings --
a Poisson model may sometimes still be reasonable
if we allow it to have a different rate for different
times.
In the wedding example, the rate would be modeled as
higher on weekends than on weekdays.
The idea of modeling rates for a Poisson distribution
against a second variable such as the day of week forms
the foundation of some more advanced methods that fall
in the realm of \termsub{generalized linear models}
    {generalized linear model}.
In Chapters~\ref{linRegrForTwoVar}
and~\ref{multipleAndLogisticRegression},
we will discuss a foundation of linear models.

\index{distribution!Poisson|)}


{


%_______________
\newpage\subsection*{Exercises} % Poisson distribution

% 1

\eoce{\qt{Customers at a coffee shop\label{coffee_shop_customers}} A coffee shop 
serves an average of 75 customers per hour during the morning rush.
\begin{parts}
\item Which distribution we have studied is most appropriate for calculating 
the probability of a given number of customers arriving within one hour 
during this time of day?
\item What are the mean and the standard deviation of the number of customers 
this coffee shop serves in one hour during this time of day?
\item Would it be considered unusually low if only 60 customers showed up to 
this coffee shop in one hour during this time of day?
\item Calculate the probability that this coffee shop serves 70 customers in 
one hour during this time of day?
\end{parts}
}{}

% 2

\eoce{\qt{Stenographer's typos\label{stenographer_typos}} A very skilled 
court stenographer makes one typographical error (typo) per hour on average.
\begin{parts}
\item What probability distribution is most appropriate for calculating the 
probability of a given number of typos this stenographer makes in an hour?
\item What are the mean and the standard deviation of the number of typos 
this stenographer makes?
\item Would it be considered unusual if this stenographer made 4 typos in a 
given hour? 
\item Calculate the probability that this stenographer makes at most 2 typos 
in a given hour.
\end{parts}
}{}
}
