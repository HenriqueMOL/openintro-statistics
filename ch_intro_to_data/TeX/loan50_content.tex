Examine the \var{interest\_\hspace{0.3mm}rate}, \var{term},
\var{home\_\hspace{0.3mm}ownership}, and \var{grade} variables
in the \data{loan50} data set.
Each of these variables is inherently different from the
other three yet some of them share certain characteristics.

First consider \var{interest\_\hspace{0.3mm}rate},
which is said to be a \term{numerical} variable since
it's always recorded as a number, and importantly, it is
sensible to add, subtract, or take averages with those values.
On the other hand, we would not classify a variable
reporting telephone area codes as numerical since the
average, sum, or difference of these values would not have
a clear meaning.

\Comment{This is a little hacky for discrete, since a loan
  could in theory be set on a weekly, daily, or other basis
  than whole months.}

The \var{term} variable represents the length of the loan,
in months.
This variable is also numerical,
although it seems to be a little different than
\var{interest\_\hspace{0.3mm}rate}.
The loan term is always given in a whole number of months
(e.g. \resp{1}, \resp{2}, \resp{3}, ...).
For this reason, the term is said to be \term{discrete}
since it can only take numerical values with jumps.
Often times, discrete numerical variables are counts.
On the other hand, the interest rate variable is said to
be \term{continuous}.

The variable \var{home\_\hspace{0.3mm}ownership} can take
one of three values: \resp{MORTGAGE}, \resp{OWN}, or \resp{RENT}.
Because the responses themselves are categories,
\var{home\_\hspace{0.3mm}ownership} is called
a \term{categorical} variable,
and the possible values are called the variable's \term{levels}.









%Many analyses are motivated by a researcher looking for
%a relationship between two or more variables.
%Someone evaluating a loan may like to answer
%some of the following questions:
%\begin{enumerate}
%\setlength{\itemsep}{0mm}
%\item[(1)]\label{loan_size_vs_total_income}
%    What is the relationship between loan amount and total income?
%\item[(2)]\label{interest_rate_vs_income}
%    If someone's income is above the average, will their
%    interest rate tend to be above or below the average?
%%\item[(2)]\label{interest_rate_vs_loan_size}
%%    Does the interest rate tend to be lower or higher
%%    for smaller loans?
%\item[(3)]\label{interest_rate_vs_rent-mortgage_question}
%    Who has a higher interest rate: those who rent
%    or those with a mortgage?
%%\item[(1)]\label{fedSpendingPovertyQuestion} Is federal spending, on average, higher or lower in counties with high rates of poverty?
%%\item[(2)]\label{ownershipMultiUnitQuestion} If homeownership is lower than the national average in one county, will the percent of multi-unit structures in that county likely be above or below the national average?
%%\item[(3)]\label{isAverageIncomeAssociatedWithSmokingBans} Which counties have a higher average income: those that enact one or more smoking bans or those that do not?
%\end{enumerate}
%
%To answer these questions, data must be collected, such as
%the \data{loan50} data set shown in Figure~\ref{loan50DF}.
%Examining summary statistics \index{summary statistic} could
%provide insights for each of the three questions about counties.
%Additionally, graphs can be used to visually summarize data
%and are useful for answering such questions as well.
%
%\indexthis{Scatterplots}{scatterplot} are one type of graph
%used to study the relationship between two numerical variables.
%Figure~\ref{loan_amount_vs_income} compares the variables
%\var{interest\_\hspace{0.3mm}rate}
%and \var{total\_\hspace{0.3mm}income}.
%Each point on the plot represents a single loan.
%For instance, the highlighted dot corresponds to
%Loan~35, which represents a borrower who had an
%income of \$45,000 and received a loan of \$9,000.
%Other details of the loan interest rate was 15.04\%,
%the borrower owns their home,
%and the loan received a grade of C.

\begin{figure}
\centering
\includegraphics[width=0.8\textwidth]{ch_intro_to_data/figures/loan_amount_vs_income/loan_amount_vs_income}
\caption{A scatterplot showing \var{loan\_\hspace{0.3mm}amount}
  against \var{total\_\hspace{0.3mm}income}. Loan~35 is highlighted,
  where the borrower had a total income of \$45,000 and received
  a loan of \$9,000.}
\label{loan_amount_vs_income}
\end{figure}

\begin{exercise}
Examine the variables in the \data{loan50} data set,
which are described in Table~\vref{loan50Variables}.
Create two additional questions about the relationships
between these variables that are of interest to
you.\footnote{Two more example questions:
  (1) How closely tied is the loan grade to the interest rate?
  (2) What is the relationship between interest rate and the
      income of the person requesting the loan?}
\end{exercise}

Figure~\ref{loan_amount_vs_income}
shows a positive trend, where the loan amount
tends to be larger for people with larger incomes.
When two variables show some connection with one another,
they are called \term{associated} variables.
Associated variables can also be called \term{dependent}
variables and vice-versa.
When the variables increase together,
as they do in Figure~\ref{loan_amount_vs_income},
they are said to be \term{positively associated}.
When the trend in the scatterplot goes down to the right,
then they are described as \term{negatively correlated}.

While we may find it interesting to consider the relationship
between two variables such as those in the scatterplot,
the relationship between those variables can be more complex.
For example, interest rates on loans tend to be chosen based
on the riskiness of the loan, i.e. how likely it is to be
paid back, and that is likely to depend on a variety of
details, such as what the loan is for, the person's
creditworthiness, whether their income is verified, etc.
We will begin exploring some of these more complex relationships
in graphs in Chapter~\ref{ch_summarizing_data} and beyond.
\Comment{Revise if we don't add these more rich plots...}

\begin{example}{Figure~\ref{interest_rate_vs_loan_amount}
    features a scatterplot of interest rate against loan amount.
    Are these variables associated?}
  There isn't an evident trend in the data,
  so we would say these two variables are not associated.
\end{example}

\begin{figure}
   \centering
   \includegraphics[width=0.8\textwidth]{ch_intro_to_data/figures/interest_rate_vs_loan_amount/interest_rate_vs_loan_amount}
   \caption{A scatterplot of interest rate against loan amount.}
   \label{interest_rate_vs_loan_amount}
\end{figure}

When two variables are not evidently associated,
as is the case between interest rate and loan amount,
they are said to be \term{independent}.
That is, two variables are independent if there is
no evident relationship between them.
