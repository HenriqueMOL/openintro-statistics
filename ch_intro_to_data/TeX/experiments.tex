\exercisesheader{}

% 29

\eoce{\qt{Light and exam performance\label{light_exam_performance}} A study is designed to 
test the effect of light level on exam performance of students. The researcher believes 
that light levels might have different effects on males and females, so wants to make 
sure both are equally represented in each treatment. The treatments are fluorescent 
overhead lighting, yellow overhead lighting, no overhead lighting (only desk lamps). 
\begin{parts}
\item What is the response variable?
\item What is the explanatory variable? What are its levels?
\item What is the blocking variable? What are its levels?
\end{parts}
}{}

% 30

\eoce{\qt{Vitamin supplements\label{vitamin_supplement}}
To assess the effectiveness of taking large doses
of vitamin C in reducing the duration of the common cold, 
researchers recruited 400 healthy volunteers from staff
and students at a university.
A~quarter of the patients were assigned a placebo,
and the rest were evenly divided between 1g Vitamin C,
3g Vitamin C, or 3g Vitamin C plus additives to be
taken at onset of a cold for the following two days.
All tablets had identical appearance and packaging.
The nurses who handed the prescribed pills to the
patients knew which patient received which treatment,
but the researchers assessing the patients when they
were sick did not. 
No significant differences were observed in any measure
of cold duration or severity between the four groups,
and the placebo group had the shortest duration of 
symptoms.\footfullcite{Audera:2001}
\begin{parts}
\item Was this an experiment or an observational study? Why?
\item What are the explanatory and response variables in this study?
\item Were the patients blinded to their treatment?
\item Was this study double-blind?
\item Participants are ultimately able to choose whether or not to use the pills 
prescribed to them. We might expect that not all of them will adhere and take their 
pills. Does this introduce a confounding variable to the study? Explain your reasoning.
\end{parts}
}{}

% 31

\eoce{\qt{Light, noise, and exam performance\label{light_noise_exam_performance}} A study is 
designed to test the effect of light level and noise level on exam performance of 
students. The researcher believes that light and noise levels might have different 
effects on males and females, so wants to make sure both are equally represented in each 
treatment. The light treatments considered are fluorescent overhead lighting, yellow 
overhead lighting, no overhead lighting (only desk lamps). The noise treatments 
considered are no noise,  construction noise, and human chatter noise.
\begin{parts}
\item What type of study is this?
\item How many factors are considered in this study? Identify them, and describe their 
levels.
\item What is the role of the sex variable in this study?
\end{parts}
}{}

% 32

\eoce{\qt{Music and learning\label{music_learning}} You would like to conduct an experiment in 
class to see if students learn better if they study without any music, with music that 
has no lyrics (instrumental), or with music that has lyrics. Briefly outline a design for 
this study.
}{}

% 33

\eoce{\qt{Soda preference\label{soda_preference}} You would like to conduct an experiment in 
class to see if your classmates prefer the taste of regular Coke or Diet Coke. Briefly 
outline a design for this study.
}{}

% 34

\eoce{\qt{Exercise and mental health\label{exercise_mental_health}} A researcher is interested 
in the effects of exercise on mental health and he proposes the following study: Use 
stratified random sampling to ensure representative proportions of 18-30, 31-40 and 41-
55 year olds from the population. Next, randomly assign half the subjects from each age 
group to exercise twice a week, and instruct the rest not to exercise. Conduct a mental 
health exam at the beginning and at the end of the study, and compare the results.
\begin{parts}
\item What type of study is this? 
\item What are the treatment and control groups in this study?
\item Does this study make use of blocking? If so, what is the blocking variable?
\item Does this study make use of blinding?
\item Comment on whether or not the results of the study can be used to establish a 
causal relationship between exercise and mental health, and indicate whether or not the 
conclusions can be generalized to the population at large.
\item Suppose you are given the task of determining if this proposed study should get 
funding. Would you have any reservations about the study proposal?
\end{parts}
}{}
