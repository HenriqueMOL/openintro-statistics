\exercisesheader{}

% 1

\eoce{\qt{Migraine and acupuncture\label{migraine_and_acupuncture_intro}} A migraine 
is a particularly painful type of headache, which patients sometimes wish to 
treat with acupuncture. To determine whether acupuncture relieves migraine 
pain, researchers conducted a randomized controlled study where 89 females 
diagnosed with migraine headaches were randomly assigned to one of two groups: 
treatment or control. 43 patients in the treatment group received acupuncture 
that is specifically designed to treat migraines. 46 patients in the control 
group received placebo acupuncture (needle insertion at non-acupoint 
locations). 24 hours after patients received acupuncture, they were asked 
if they were pain free. Results are summarized in the contingency table below. 
\footfullcite{Allais:2011}

\noindent\begin{minipage}[l]{0.4\textwidth}
\begin{tabular}{ll  cc c} 
			                         		&           & \multicolumn{2}{c}{\textit{Pain free}} \\
\cline{3-4}
			                        	 	&			& Yes 	& No 	                  & Total \\
\cline{2-5}
							& Treatment 	& 10	 	& 33		                  & 43 \\
\raisebox{1.5ex}[0pt]{\emph{Group}} & Control	 	& 2	 	& 44 	 	                  & 46 \\
\cline{2-5}
							& Total		& 12		& 77		                  & 89
\end{tabular}
\end{minipage}
\begin{minipage}[c]{0.05\textwidth}
\end{minipage}
\begin{minipage}[c]{0.27\textwidth}
\begin{center}
\includegraphics[width = 0.75\textwidth]{ch_intro_to_data/figures/eoce/migraine_and_acupuncture_intro/earacupuncture.pdf}
\end{center}
\end{minipage}
\begin{minipage}[c]{0.25\textwidth}
{\footnotesize Figure from the original paper displaying the appropriate area 
(M) versus the inappropriate area (S) used in the treatment of migraine attacks.}
\end{minipage}
\begin{parts}
\item What percent of patients in the treatment group were pain free 24 hours 
after receiving acupuncture? 
\item What percent were pain free in the control group?
\item In which group did a higher percent of patients become pain free 24 hours 
after receiving acupuncture?
\item Your findings so far might suggest that acupuncture is an effective treatment 
for migraines for all people who suffer from migraines. However this is not the 
only possible conclusion that can be drawn based on your findings so far. What is 
one other possible explanation for the observed difference between the percentages 
of patients that are pain free 24 hours after receiving acupuncture in the two groups?
\end{parts}
}{}

% 2

\eoce{\qt{Sinusitis and antibiotics\label{sinusitis_and_antibiotics_intro}} 
Researchers studying the effect of antibiotic treatment for acute sinusitis 
compared to symptomatic treatments randomly assigned 166 adults diagnosed 
with acute sinusitis to one of two groups: treatment or control. Study 
participants received either a 10-day course of amoxicillin (an antibiotic) 
or a placebo similar in appearance and taste. The placebo consisted of 
symptomatic treatments such as acetaminophen, nasal decongestants, etc. At the 
end of the 10-day period patients were asked if they experienced improvement 
in symptoms. The distribution of responses is summarized below. 
\footfullcite{Garbutt:2012}
\begin{center}
\begin{tabular}{ll  cc c} 
                                    			&			& \multicolumn{2}{c}{\textit{Self-reported improvement}} \\
                                    			&			& \multicolumn{2}{c}{\textit{in symptoms}} \\
\cline{3-4}
			                        		&			& Yes 	& No 	& Total \\
\cline{2-5}
							& Treatment 	& 66		& 19		& 85 \\
\raisebox{1.5ex}[0pt]{\emph{Group}}	& Control		& 65		& 16 		& 81 \\
\cline{2-5}
							& Total		& 131	& 35		& 166
\end{tabular}
\end{center}
\begin{parts}
\item What percent of patients in the treatment group experienced improvement 
in symptoms? 
\item What percent experienced improvement in symptoms in the 
control group?
\item In which group did a higher percentage of patients experience improvement
in symptoms?
\item Your findings so far might suggest that a real difference in effectiveness of 
antibiotic and placebo treatments for improving symptoms of sinusitis. However this is 
not the only possible conclusion that can be drawn based on your findings so far. What is 
one other possible explanation for the observed difference between the percentages 
of patients that are pain free 24 hours after receiving acupuncture in the two groups?
\end{parts}
}{}
