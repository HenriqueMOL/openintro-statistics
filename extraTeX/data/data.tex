\chapter{Data sets within the text}
\label{appendix_data}

%A foundational principle that supports quality statistical
%analysis is well-organized data.

Each data set within the text is described in this appendix,
and there is a corresponding page for each of these data sets at
\oiRedirect{data}
    {\color{black}\textbf{openintro.org/data}}.
This page also includes additional data sets that can be
used for honing your skills.
Each data set has its own page with the following information:
\begin{itemize}
\item
    Description of each data set.
\item
    Detailed overview of each data set's variables.
\item
    CSV download.
\item
    R object file download.
\end{itemize}
Over time we will also expand the information available
on these pages.

%\vspace{10mm}

\Comment{Redirects must be created for each link in this appendix.}



\section{\nameref{ch_intro_to_data}}
\label{ch_intro_to_data_data}

\begin{itemize}
\item[\ref{basicExampleOfStentsAndStrokes}]
    [\datalink{stent30}, \datalink{stent365}]
    The stent data is split across two data sets,
    one for the 0-30 day and one for the 0-365 day results. \\
    Chimowitz MI, Lynn MJ, Derdeyn CP, et al. 2011.
    Stenting versus Aggressive Medical Therapy for
    Intracranial Arterial Stenosis.
    New England Journal of Medicine 365:993-1003.
    \oiRedirect{textbook-nejm_stent_study}
        {www.nejm.org/doi/full/10.1056/NEJMoa1105335}. \\
    NY Times article:
    \oiRedirect{textbook-nytimes_stent_study}
        {www.nytimes.com/2011/09/08/health/research/08stent.html}.

\item[\ref{dataBasics}]
    [\datalink{loan50}, \datalink{loans\_full\_schema}]
    This data comes from Lending Club
    (\href{https://www.lendingclub.com/info/download-data.action}
        {lendingclub.com}),
    which provides a large set of data on the people who
    received loans through their platform.
    The data used in the textbook comes from a sample
    of the loans made in Q1 (Jan, Feb, March) 2018.
\item[\ref{dataBasics}]
    [\datalink{county}, \datalink{county\_complete}]
    These data come from several government sources.
    For those variables included in the
    county data set, only the most recent data is reported,
    as of what was available in late 2018.
    Data prior to 2011 is all from
    \href{http://census.gov}{census.gov},
    where the specific Quick Facts page providing the data
    is no longer online.
    The more recent data comes from
    \href{https://www.ers.usda.gov/data-products/county-level-data-sets/download-data/}
        {USDA (ers.usda.gov)},
    \href{https://www.bls.gov/lau/}
        {Bureau of Labor Statistics (bls.gov/lau)},
    \href{https://www.census.gov/did/www/saipe/}
        {SAIPE (census.gov/did/www/saipe)},
    and
    \href{https://www.census.gov/programs-surveys/acs/}
        {American Community Survey
            (census.gov/programs-surveys/acs)}.

\item[\ref{overviewOfDataCollectionPrinciples}]
    No data sets were described in this section.

\item[\ref{section_obs_data_sampling}]
    No data sets were described in this section.

\item[\ref{experimentsSection}]
    No data sets were described in this section.
\end{itemize}






\section{\nameref{ch_summarizing_data}}
\label{ch_summarizing_data_data}

\begin{itemize}
\item[\ref{numericalData}]
    [\datalink{loan50}, \datalink{county}]
    These data sets are described in
    Appendix~\ref{ch_intro_to_data_data}.

\item[\ref{categoricalData}]
    [\datalink{loan50}, \datalink{county}]
    These data sets are described in
    Appendix~\ref{ch_intro_to_data_data}.

\item[\ref{caseStudyMalariaVaccine}]
    [\datalink{malaria}]
    Lyke et al. 2017.
    PfSPZ vaccine induces strain-transcending T cells
    and durable protection against heterologous controlled
    human malaria infection.
    PNAS 114(10):2711-2716.
    \url{http://www.pnas.org/content/114/10/2711}
\end{itemize}









\section{\nameref{ch_probability}}
\label{ch_probability_data}

\begin{itemize}
\item[\ref{basicsOfProbability}]
    [\datalink{loan50}, \datalink{county}]
    These data sets are described in
    Appendix~\ref{ch_intro_to_data_data}.
\item[\ref{basicsOfProbability}]
    [\datalink{playing\_cards}]
    A table describing the 52 cards in a standard deck.

\item[\ref{conditionalProbabilitySection}]
    [\datalink{family\_college}]
    A simulated data set based on real population summaries at
    \oiRedirect{textbook-student_parent_college_2001}
        {nces.ed.gov/pubs2001/2001126.pdf}.
\item[\ref{conditionalProbabilitySection}]
    [\datalink{smallpox}]
    Fenner F. 1988.
    Smallpox and Its Eradication
    (History of International Public Health, No. 6).
    Geneva: World Health Organization. ISBN 92-4-156110-6.

\item[\ref{conditionalProbabilitySection}]
    [Mammogram screening, probabilities.]
    The probabilities reported were obtained using studies
    reported at
    \oiRedirect{textbook-breastCancerDotOrg_20090831b}
        {www.breastcancer.org}
    and \oiRedirect{textbook-ncbi_nih_breast_cancer}
        {www.ncbi.nlm.nih.gov/pmc/articles/PMC1173421}. 

\item[\ref{conditionalProbabilitySection}]
    [Jose campus visits, probabilities, no data link]
    This example was entirely made up.

\item[\ref{smallPop}]
    No data sets were described in this section.

\item[\ref{randomVariablesSection}]
    [Course material purchases, probabilities, no data link]
    The probabilities for course materials purchases were
    created as hypotheticals.
    \Comment{Should this even be mentioned?}

\item[\ref{randomVariablesSection}]
    [Auctions for TV and toaster, no data link]
    The statistics used were created as hypotheticals.
    \Comment{Should this even be mentioned?}

\item[\ref{randomVariablesSection}]
    [\datalink{stocks\_18}]
    Monthly returns for Caterpillar, Exxon Mobil Corp,
    and Google for November 2015 to October 2018.

\item[\ref{contDist}]
    [\datalink{fcid}]
    This sample can be considered a simple random sample
    from the US population.
    It relies on the USDA Food Commodity Intake Database.

\end{itemize}




\section{\nameref{ch_distributions}}
\label{ch_distributions_data}

\begin{itemize}
\item[SAT and ACT score distributions]
    \Comment{Would be nice to add something here...}
\item[Male heights]
    The distribution is based on the
    USDA Food Commodity Intake Database.
\item[\ref{normalDist}]
    [\datalink{possum}]
    The distribution parameters are based on a sample
    of possums from Australia and New Guinea.
    The original source of this data is as follows.
    Lindenmayer DB, et al. 1995.
    \emph{Morphological variation among columns of the
        mountain brushtail possum, Trichosurus caninus
        Ogilby (Phalangeridae: Marsupiala)}.
    Australian Journal of Zoology 43: 449-458.

\item[\ref{assessingNormal}]
    [\datalink{male\_heights\_fcid}]
    This sample can be considered a simple random sample
    from the US population.
    It relies on the USDA Food Commodity Intake Database.
\item[\ref{assessingNormal}]
    [\datalink{simulated\_normal}]
    These data were simulated from a standard normal distribution.
    This data set includes three different data sets.
\item[\ref{assessingNormal}]
    [\datalink{nba\_players\_19}]
    Summary information from the NBA players for the
    2018-2019 season.
    Data were retrieved from
    \oiRedirect{data-nba_players_19}{www.nba.com/players}.
\item[\ref{assessingNormal}]
    [\datalink{poker}]
    Poker winnings (and losses) for 50 days by a professional
    poker player, which represents their first 50 days trying
    to play for a living.
    Anonymity has been requested by the player.
\item[\ref{assessingNormal}]
    [\datalink{simulated\_dist}]
    Simulated data sets,
    not necessarily drawn from a normal distribution.
    This data set includes six different data sets.

\item[\ref{geomDist}]
    [\datalink{}]
    \Comment{Reference for the Milgram experiment rates.}
\item[\ref{geomDist}]
    [\datalink{}]
    \Comment{Reference previous section and Milgram
        experiment rates.}

\item[\ref{binomialModel}]
    The statistics referenced in Section~\ref{geomDist} were used.
    No new data sets were described in this section.
    
\item[\ref{negativeBinomial}]
    [\datalink{}]
    \Comment{Reference for the smoking rate in the US.}

\item[\ref{poisson}]
    [\datalink{ami\_occurrences}]
    This is a simulated data set but resembles actual
    AMI data for New York City based on typical AMI
    incidence rates.
\end{itemize}








\begin{itemize}
\item[\ref{}]
    [\datalink{}]
    

\item[\ref{}]
    [\datalink{}]
    

\item[\ref{}]
    [\datalink{}]
    

\item[\ref{}]
    [\datalink{}]
    

\item[\ref{}]
    [\datalink{}]
    

\item[\ref{}]
    [\datalink{}]
    

\end{itemize}

