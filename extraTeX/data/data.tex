\chapter{Data sets within the text}
\label{appendix_data}
\label{data_appendix}

%A foundational principle that supports quality statistical
%analysis is well-organized data.

Each data set within the text is described in this appendix,
and there is a corresponding page for each of these data sets at
\oiRedirect{data}
    {\color{black}\textbf{openintro.org/data}}.
This page also includes additional data sets that can be
used for honing your skills.
Each data set has its own page with the following information:
\begin{itemize}
\item
    Description of each data set.
\item
    Detailed overview of each data set's variables.
\item
    CSV download.
\item
    R object file download.
\end{itemize}
Over time we will also expand the information available
on these pages.

%\vspace{10mm}

\Comment{Redirects must be created for each link in this appendix.}



\section{\nameref{ch_intro_to_data}}
\label{ch_intro_to_data_data}

\begin{itemize}
\item[\ref{basicExampleOfStentsAndStrokes}]
    [\datalink{stent30}, \datalink{stent365}]
    The stent data is split across two data sets,
    one for the 0-30 day and one for the 0-365 day results. \\
    Chimowitz MI, Lynn MJ, Derdeyn CP, et al. 2011.
    Stenting versus Aggressive Medical Therapy for
    Intracranial Arterial Stenosis.
    New England Journal of Medicine 365:993-1003.
    \oiRedirect{textbook-nejm_stent_study}
        {www.nejm.org/doi/full/10.1056/NEJMoa1105335}. \\
    NY Times article:
    \oiRedirect{textbook-nytimes_stent_study}
        {www.nytimes.com/2011/09/08/health/research/08stent.html}.

\item[\ref{dataBasics}]
    [\datalink{loan50}, \datalink{loans\_full\_schema}]
    This data comes from Lending Club
    (\href{https://www.lendingclub.com/info/download-data.action}
        {lendingclub.com}),
    which provides a large set of data on the people who
    received loans through their platform.
    The data used in the textbook comes from a sample
    of the loans made in Q1 (Jan, Feb, March) 2018.
\item[\ref{dataBasics}]
    [\datalink{county}, \datalink{county\_complete}]
    These data come from several government sources.
    For those variables included in the
    county data set, only the most recent data is reported,
    as of what was available in late 2018.
    Data prior to 2011 is all from
    \href{http://census.gov}{census.gov},
    where the specific Quick Facts page providing the data
    is no longer available.
    The more recent data comes from
    \href{https://www.ers.usda.gov/data-products/county-level-data-sets/download-data/}
        {USDA (ers.usda.gov)},
    \href{https://www.bls.gov/lau/}
        {Bureau of Labor Statistics (bls.gov/lau)},
    \href{https://www.census.gov/did/www/saipe/}
        {SAIPE (census.gov/did/www/saipe)},
    and
    \href{https://www.census.gov/programs-surveys/acs/}
        {American Community Survey
            (census.gov/programs-surveys/acs)}.

\item[\ref{overviewOfDataCollectionPrinciples}]
    No data sets were described in this section.

\item[\ref{section_obs_data_sampling}]
    The Nurses' Health Study was mentioned.
    For more information on this data set, see \\
    \oiRedirect{textbook-channing_nurse_study}
        {www.channing.harvard.edu/nhs}

\item[\ref{experimentsSection}]
    The study we had in mind when discussing the
    simple randomization (no blocking) study was \\
    Anturane Reinfarction Trial Research Group. 1980.
    \emph{Sulfinpyrazone in the prevention of sudden
    death after myocardial infarction.}
    New England Journal of Medicine 302(5):250-256.
\end{itemize}






\section{\nameref{ch_summarizing_data}}
\label{ch_summarizing_data_data}

\begin{itemize}
\item[\ref{numericalData}]
    [\datalink{loan50}, \datalink{county}]
    These data sets are described in
    Data Appendix~\ref{ch_intro_to_data_data}.

\item[\ref{categoricalData}]
    [\datalink{loan50}, \datalink{county}]
    These data sets are described in
    Data Appendix~\ref{ch_intro_to_data_data}.

\item[\ref{caseStudyMalariaVaccine}]
    [\datalink{malaria}]
    Lyke et al. 2017.
    PfSPZ vaccine induces strain-transcending T cells
    and durable protection against heterologous controlled
    human malaria infection.
    PNAS 114(10):2711-2716.
    \url{http://www.pnas.org/content/114/10/2711}
\end{itemize}









\section{\nameref{ch_probability}}
\label{ch_probability_data}

\begin{itemize}
\item[\ref{basicsOfProbability}]
    [\datalink{loan50}, \datalink{county}]
    These data sets are described in
    Data Appendix~\ref{ch_intro_to_data_data}.
\item[\ref{basicsOfProbability}]
    [\datalink{playing\_cards}]
    A table describing the 52 cards in a standard deck.

\item[\ref{conditionalProbabilitySection}]
    [\datalink{family\_college}]
    A simulated data set based on real population summaries at
    \oiRedirect{textbook-student_parent_college_2001}
        {nces.ed.gov/pubs2001/2001126.pdf}.
\item[\ref{conditionalProbabilitySection}]
    [\datalink{smallpox}]
    Fenner F. 1988.
    Smallpox and Its Eradication
    (History of International Public Health, No. 6).
    Geneva: World Health Organization. ISBN 92-4-156110-6.

\item[\ref{conditionalProbabilitySection}]
    [Mammogram screening, probabilities.]
    The probabilities reported were obtained using studies
    reported at
    \oiRedirect{textbook-breastCancerDotOrg_20090831b}
        {www.breastcancer.org}
    and \oiRedirect{textbook-ncbi_nih_breast_cancer}
        {www.ncbi.nlm.nih.gov/pmc/articles/PMC1173421}. 

\item[\ref{conditionalProbabilitySection}]
    [Jose campus visits, probabilities, no data link]
    This example was entirely made up.

\item[\ref{smallPop}]
    No data sets were described in this section.

\item[\ref{randomVariablesSection}]
    [Course material purchases, probabilities, no data link]
    The probabilities for course materials purchases were
    created as hypotheticals.
    \Comment{Should this even be mentioned?}

\item[\ref{randomVariablesSection}]
    [Auctions for TV and toaster, no data link]
    The statistics used were created as hypotheticals.
    \Comment{Should this even be mentioned?}

\item[\ref{randomVariablesSection}]
    [\datalink{stocks\_18}]
    Monthly returns for Caterpillar, Exxon Mobil Corp,
    and Google for November 2015 to October 2018.

\item[\ref{contDist}]
    [\datalink{fcid}]
    This sample can be considered a simple random sample
    from the US population.
    It relies on the USDA Food Commodity Intake Database.

\end{itemize}




\section{\nameref{ch_distributions}}
\label{ch_distributions_data}

\begin{itemize}
\item[\ref{normalDist}]
    [SAT and ACT score distributions]
    The SAT score data comes from the 2018 distribution,
    which is provided at \\
    {\small
    \oiRedirect{textbook-collegeboard_sat_2018_score_distribution}
        {reports.collegeboard.org/pdf/2018-total-group-sat-suite-assessments-annual-report.pdf}} \\
    The ACT score data is available at \\
    {\footnotesize
    \oiRedirect{textbook-act_2018_score_distribution}
        {act.org/content/dam/act/unsecured/documents/cccr2018/P\_99\_999999\_N\_S\_N00\_ACT-GCPR\_National.pdf}} \\
    We also acknowledge that the actual ACT score distribution
    is \emph{not} nearly normal.
    However, since the topic is very accessible,
    we decided to keep the context and examples.
\item[\ref{normalDist}]
    [Male heights]
    The distribution is based on the
    USDA Food Commodity Intake Database.
\item[\ref{normalDist}]
    [\datalink{possum}]
    The distribution parameters are based on a sample
    of possums from Australia and New Guinea.
    The original source of this data is as follows.
    Lindenmayer DB, et al. 1995.
    \emph{Morphological variation among columns of the
        mountain brushtail possum, Trichosurus caninus
        Ogilby (Phalangeridae: Marsupiala)}.
    Australian Journal of Zoology 43: 449-458.

%\item[\ref{assessingNormal}]
%    [\datalink{male\_heights\_fcid}]
%    This sample can be considered a simple random sample
%    from the US population.
%    It relies on the USDA Food Commodity Intake Database.
%\item[\ref{assessingNormal}]
%    [\datalink{simulated\_normal}]
%    These data were simulated from a standard normal distribution.
%    This data set includes three different data sets.
%\item[\ref{assessingNormal}]
%    [\datalink{nba\_players\_19}]
%    Summary information from the NBA players for the
%    2018-2019 season.
%    Data were retrieved from
%    \oiRedirect{data-nba_players_19}{www.nba.com/players}.
%\item[\ref{assessingNormal}]
%    [\datalink{poker}]
%    Poker winnings (and losses) for 50 days by a professional
%    poker player, which represents their first 50 days trying
%    to play for a living.
%    Anonymity has been requested by the player.
%\item[\ref{assessingNormal}]
%    [\datalink{simulated\_dist}]
%    Simulated data sets,
%    not necessarily drawn from a normal distribution.
%    This data set includes six different data sets.

\item[\ref{geomDist}]
    [Exceeding insurance deductible]
    These statistics were made up but are possible
    values one might observe for low-deductible plans.

\item[\ref{binomialModel}]
    [Exceeding insurance deductible]
    These statistics were made up but are possible
    values one might observe for low-deductible plans.
\item[\ref{binomialModel}]
    [Smoking friends]
    Unfortunately, we don't currently have additional
    information on the source for the 30\% statistic,
    so don't consider this one as fact since we cannot
    verify it was from a reputable source.    
\item[\ref{binomialModel}]
    [US smoking rate]
    The 15\% smoking rate in the US figure is close to
    the value from the Centers for Disease Control and
    Prevention website, which reports a value of 14\%
    as of the 2017 estimate: \\
    \href{https://www.cdc.gov/tobacco/data_statistics/fact_sheets/adult_data/cig_smoking/index.htm}{cdc.gov/tobacco/data\_statistics/fact\_sheets/adult\_data/cig\_smoking/index.htm}

\item[\ref{negativeBinomial}]
    [Football kicker]
    This example was made up.
\item[\ref{negativeBinomial}]
    [Heart attack admissions]
    This example was made up, though the heart attack
    admissions are realistic for some hospitals.

\item[\ref{poisson}]
    [\datalink{ami\_occurrences}]
    This is a simulated data set but resembles actual
    AMI data for New York City based on typical AMI
    incidence rates.
\end{itemize}








\section{\nameref{ch_foundations_for_inf}}
\label{ch_foundations_for_inf_data}

\begin{itemize}
\item[\ref{pointEstimates}]
    [\datalink{pew\_energy\_2018}]
    The actual data has more observations than were referenced
    in this chapter.
    That is, we used a subsample since it helped smooth some
    of the examples to have a bit more variability.
    The \data{pew\us{}energy\us{}2018} data set represents
    the full data set for each of the different energy source
    questions, which covers solar, wind, offshore drilling,
    hydrolic fracturing, and nuclear energy.
    The statistics used to construct the data are from
    the following page:
    \begin{center}
    \oiRedirect{textbook-pew_2018_poll_on_solar_and_wind_expansion}
        {{\small{www.pewinternet.org/2018/05/14/majorities-see-government-efforts-to-protect-the-environment-as-insufficient/}}}
    \end{center}
    
\item[\ref{confidenceIntervals}]
    [\datalink{pew\_energy\_2018}]
    See the details for this data set above
    in the Section~\ref{pointEstimates} data section.
\item[\ref{confidenceIntervals}]
    [\datalink{ebola\_survey}]
    In New York City on October 23rd, 2014, a doctor who had
    recently been treating Ebola patients in Guinea went to
    the hospital with a slight fever and was subsequently
    diagnosed with Ebola.
    Soon thereafter, an NBC~4 New York/The Wall Street
    Journal/Marist Poll found that
    82\% of New Yorkers favored a
    ``mandatory 21-day quarantine for anyone who has come
    in contact with an Ebola patient''.
    This poll included responses of 1,042
    New York adults between Oct 26th and~28th, 2014.
    \oiRedirect{textbook-maristpoll_ebola_201410}
        {Poll ID NY141026 on maristpoll.marist.edu}.

\item[\ref{hypothesisTesting}]
    [\datalink{pew\_energy\_2018}]
    See the details for this data set above
    in the Section~\ref{pointEstimates} data section.
\item[\ref{hypothesisTesting}]
    [Rosling questions]
    We noted much smaller samples than the Roslings'
    describe in their book,
    \oiRedirect{amazon_factfulness}{Factfulness},
    The samples we describe are similar but not
    the same as the actual rates.
    The approximate rates for the correct answers for the
    two questions for (sometimes different) populations
    discussed in the book, as reported in
    \oiRedirect{amazon_factfulness}{Factfulness},
    are
    \begin{itemize}
    \item
        80\% of the world's 1 year olds have been vaccinated
        against some disease:
        13\% get this correct (17\% in the US).
        \oiRedirect{gapm-io-q9}{gapm.io/q9}
    \item
        Number of children in the world in 2100:
        9\% correct.
        \oiRedirect{gapm-io-q5}{gapm.io/q5}
    \end{itemize}
    Here are a few more questions and a rough percent
    of people who get them correct:
    \begin{itemize}
    \item
        In all low-income countries across the world today,
        how many girls finish primary school: 20\%, 40\%, or 60\%?
        Answer: 60\%.
        About 7\% of people get this question correct.
        \oiRedirect{gapm-io-q1}{gapm.io/q1}
    \item
        What is the life expectancy of the world today:
        50 years, 60 years, or 70 years?
        Answer: 70 years.
        In the US, about 43\% of people get this question correct.
        \oiRedirect{gapm-io-q4}{gapm.io/q4}
%    \item
%        How many of the world's 1 year old children today
%        have been vaccinated against some disease:
%        20\%, 50\%, or 80\%?
%        Answer: 80\%.
%        About 13\% of people get this question correct.
%        \oiRedirect{gapm-io-q9}{gapm.io/q9}
    \item
        In 1996, tigers, giant pandas, and black rhinos
        were all listed as endangered.
        How many of these three species are more
        critically endangered today:
        two of them,
        one of them,
        none of them?
        Answer: none of them.
        About 7\% of people get this question correct.
        \oiRedirect{gapm-io-q11}{gapm.io/q11}
    \item
        How many people in the world have some access
        to electricity? 20\%, 50\%, 80\%.
        Answer: 80\%.
        About 22\% of people get this correct.
        \oiRedirect{gapm-io-q12}{gapm.io/q12}
    \end{itemize}
    For more information, check out the book,
    \oiRedirect{amazon_factfulness}{Factfulness}.
\item[\ref{hypothesisTesting}]
    [\datalink{nuclear\_survey}]
    A simple random sample of 1,028 US adults in March 2013
    found that 56\% of US adults support nuclear arms reduction.
    \oiRedirect{textbook-nuclear_arms_reduction_201303}
        {www.gallup.com/poll/161198/favor-russian-nuclear-arms-reductions.aspx}
\item[\ref{hypothesisTesting}]
    [\datalink{stent30}, \datalink{stent365}]
    This data is described in
    Data Appendix~\ref{ch_intro_to_data_data}.

\end{itemize}







\section{\nameref{ch_inference_for_props}}
\label{ch_inference_for_props_data}

\begin{itemize}
\item[\ref{singleProportion}]
    [Payday loans]
    The statistics come from the following source: \\
    {\small\oiRedirect{pew-payday-loans-2017}
        {pewtrusts.org/-/media/assets/2017/04/payday-loan-customers-want-more-protections-methodology.pdf}}
\item[\ref{singleProportion}]
    [Tire factory]
    The statistics were created for the purposes
    of performing sample size calculations.

\item[\ref{differenceOfTwoProportions}]
    [\datalink{cpr}]
    B$\ddot{\text{o}}$ttiger et al.
    \emph{Efficacy and safety of thrombolytic therapy after
        initially unsuccessful cardiopulmonary resuscitation:
        a prospective clinical trial}.
        The Lancet, 2001.
\item[\ref{differenceOfTwoProportions}]
    [\datalink{fish\_oil\_18}]
    Manson JE, et al. 2018.
    Marine n-3 Fatty Acids and Prevention of
    Cardiovascular Disease and Cancer. NEJMoa1811403.
\item[\ref{differenceOfTwoProportions}]
    [\datalink{mammogram}]
    \oiRedirect{textbook-90k_mammogram_study_2014}
        {Miller AB. 2014.
            \emph{Twenty five year follow-up for breast cancer
            incidence and mortality of the Canadian National
            Breast Screening Study: randomised screening trial}.
            BMJ 2014;348:g366.}
\item[\ref{differenceOfTwoProportions}]
    [\datalink{drone\_blades}]
    The quality control data set for quadcopter drone blades
    is a made-up data set for an example.
    We provide the simulated data in the
    \data{drone\us{}blades} data set.

\item[\ref{oneWayChiSquare}]
    [\datalink{jury}]
    The jury data set for examining discrimination
    is a made-up data set an example.
    We provide the simulated data in the \data{jury} data set.
\item[\ref{oneWayChiSquare}]
    [\datalink{sp500\_1950\_2018}]
    Data is sourced from
    \oiRedirect{yahoo_finance}
        {finance.yahoo.com}.

\item[\ref{twoWayTablesAndChiSquare}]
    [\datalink{ask}]
    Minson JA, Ruedy NE, Schweitzer ME.
    \emph{There is such a thing as a stupid question:
    Question disclosure in strategic communication}. \\
    {\small\oiRedirect{minson_ruedy_data_source}
        {opim.wharton.upenn.edu/DPlab/papers/workingPapers/}}\\
    {\small\oiRedirect{minson_ruedy_data_source}
        {Minson\_working\_Ask\%20(the\%20Right\%20Way)\%20and\%20You\%20Shall\%20Receive.pdf}}

\item[\ref{twoWayTablesAndChiSquare}]
    [\datalink{diabetes2}]
    Zeitler P, et al. 2012.
    A Clinical Trial to Maintain Glycemic Control in Youth
    with Type~2 Diabetes.
    N Engl J Med.

\end{itemize}






\section{\nameref{ch_inference_for_means}}
\label{ch_inference_for_means_data}

\begin{itemize}
\item[\ref{oneSampleMeansWithTDistribution}]
    [Risso's dolphins]
    Endo T and Haraguchi K. 2009.
    High mercury levels in hair samples from residents of Taiji,
    a Japanese whaling town.
    Marine Pollution Bulletin 60(5):743-747.

    Taiji was featured in the movie
    \emph{The Cove}, and it is a significant source of dolphin
    and whale meat in Japan.
    Thousands of dolphins pass through the Taiji area annually,
    and we will assume these 19 dolphins represent a simple
    random sample from those dolphins.
\item[\ref{oneSampleMeansWithTDistribution}]
    [Croaker white fish]
    \oiRedirect{textbook-fda_mercury_in_fish_2010}
        {www.fda.gov/food/foodborneillnesscontaminants/metals/ucm115644.htm}
\item[\ref{oneSampleMeansWithTDistribution}]
    [\datalink{run17}]
    \oiRedirect{textbook-cherryblossom_org}{www.cherryblossom.org}

\item[\ref{pairedData}]
    [\datalink{textbooks}, \datalink{ucla\_textbooks\_f18}]
    Data were collected by OpenIntro staff in 2010 and again
    in 2018.
    For the 2018 sample, we sampled 201 UCLA courses.
    Of those, 68 required books that could be
    found on Amazon.
    The websites where information was retrieved: \\
    \oiRedirect{ucla_class_schedule}
        {sa.ucla.edu/ro/public/soc},
    \oiRedirect{ucla_verbacompare}{ucla.verbacompare.com},
    and \oiRedirect{amazon}{amazon.com}.

\item[\ref{differenceOfTwoMeans}]
    [\datalink{stem\_cells}]
    \href{menard-stem-cells-2005}%{https://www.thelancet.com/journals/lancet/article/PIIS0140-6736(05)67380-1/fulltext}
        {Menard C, et al. 2005.
            Transplantation of cardiac-committed mouse embryonic
            stem cells to infarcted sheep myocardium:
            a preclinical study.
            The Lancet: 366:9490, p1005-1012.}
\item[\ref{differenceOfTwoMeans}]
    [\datalink{ncbirths}]
    Birth records released by North Carolina in 2004.
    Unfortunately, we don't currently have additional
    information on the source for this data set.
\item[\ref{differenceOfTwoMeans}]
    [Exam versions]
    This is a made-up data set for an example.
    There isn't any explicit data set to analyze,
    only summary statistics.

\item[\ref{PowerForDifferenceOfTwoMeans}]
    [Blood pressure statistics]
    The blood pressure standard deviation for patients
    with blood pressure ranging from from 140 to 180 mmHg
    is guessed and may be a little (but likely not dramatically)
    imprecise from what we'd observe in actual data.

\item[\ref{anovaAndRegrWithCategoricalVariables}]
    [\datalink{toy\_anova}]
    Data used for Figure~\ref{toyANOVA}, which was fake data.
\item[\ref{anovaAndRegrWithCategoricalVariables}]
    [\datalink{mlb\_players\_18}]
    Data were retrieved from
    \oiRedirect{mlb-stats}{mlb.mlb.com/stats}.
    Only players with at least 100 at bats were considered
    during the analysis.
\item[\ref{anovaAndRegrWithCategoricalVariables}]
    [\datalink{class\_data}]
    These data are simulated.
    \Comment{Data set is currently called \data{classData}
      in the package.}

\end{itemize}






\section{\nameref{ch_regr_simple_linear}}
\label{ch_regr_simple_linear_data}

\begin{itemize}
\item[\ref{fitting_line_to_data_section}]
    [\datalink{simulated\_scatter}]
    Fake data used for the first three plots.
    The perfect linear plot uses group~4 data,
    where \var{group} variable in the data set
    (Figure~\ref{perfLinearModel}).
    The group of 3 imperfect linear plots use groups~1-3
    (Figure~\ref{imperfLinearModel}).
    The sinusoidal curve uses group~5 data
    (Figure~\ref{notGoodAtAllForALinearModel}).
    The curved plot of data with the curved band
    uses group~30 data
    (right panel of Figure~\ref{scattHeadLTotalLTube}).
    The group of 3 scatterplots with residual plots use groups~6-8
    (Figure~\ref{sampleLinesAndResPlots}).
    The correlation plots uses groups~9-19 data
    (Figures~\ref{posNegCorPlots} and~\ref{corForNonLinearPlots}).
\item[\ref{fitting_line_to_data_section}]
    [\datalink{possum}]
    This data is described in
    Data Appendix~\ref{ch_distributions_data}.

\item[\ref{fittingALineByLSR}]
    [\datalink{elmhurst}]
    These data were sampled from a table of data for all
    freshman from the 2011 class at Elmhurst College that
    accompanied an article titled
    \emph{What Students Really Pay to Go to College}
    published online by \emph{The~Chronicle of Higher Education}:
    \oiRedirect{textbook-chronicle_elmhurst_article}
        {chronicle.com/article/What-Students-Really-Pay-to-Go/131435}.
\item[\ref{fittingALineByLSR}]
    [\datalink{simulated\_scatter}]
    The plots for things that can go wrong uses groups 20-23
    (Figure~\ref{whatCanGoWrongWithLinearModel}).
\item[\ref{fittingALineByLSR}]
    [\datalink{mario\_kart}]
    \Comment{Confirm data set name change.}
    Auction data from Ebay (ebay.com) for the game Mario Kart
    for the Nintendo Wii.
    This data set was collected in early October, 2009.

\item[\ref{typesOfOutliersInLinearRegression}]
    [\datalink{simulated\_scatter}]
    The plots for types of outliers uses groups 24-29
    (Figure~\ref{outlierPlots}).

\item[\ref{inferenceForLinearRegression}]
    [\datalink{midterms\_house}]
    Data was retrieved from Wikipedia.

\end{itemize}







\section{\nameref{ch_regr_mult_and_log}}
\label{ch_regr_mult_and_log_data}

\begin{itemize}
\item[\ref{introductionToMultipleRegression}]
    [\datalink{loans\_full\_schema}]
    This data is described in
    Data Appendix~\ref{ch_intro_to_data_data}.

\item[\ref{model_selection_section}]
    [\datalink{loans\_full\_schema}]
    This data is described in
    Data Appendix~\ref{ch_intro_to_data_data}.

\item[\ref{multipleRegressionModelAssumptions}]
    [\datalink{loans\_full\_schema}]
    This data is described in
    Data Appendix~\ref{ch_intro_to_data_data}.

\item[\ref{mario_kart_case_study}]
    [\datalink{mario\_kart}]
    This data is describe in
    Data Appendix~\ref{ch_regr_simple_linear_data}.

\item[\ref{logisticRegression}]
    [\datalink{resume}]
    Bertrand M, Mullainathan S. 2004.
    \emph{Are Emily and Greg More Employable than Lakisha and Jamal?
    A Field Experiment on Labor Market Discrimination}.
    The American Economic Review 94:4 (991-1013).
    \oiRedirect{resume-data-2004}
        {www.nber.org/papers/w9873}

    We did omit discussion of some structure in
    the data for the analysis presented:
    the experiment design included blocking,
    where typically four resumes were sent to each job:
    one for each inferred race/sex combination
    (as inferred based on the first name).
    We did not worry about this blocking aspect,
    since accounting for the blocking would
    \emph{reduce} the standard error without notably
    changing the point estimates for the
    \var{race} and \var{sex} variables
    versus the analysis performed in the section.
    That is, the most interesting conclusions in the
    study would be unaffected even with a more
    sophisticated analysis.

\item[\ref{logisticRegression}]
    [\datalink{research\_reply}]
    Milkman KL, Akinola M, Chugh D. 2015.
    What Happens Before?
    A Field Experiment Exploring How Pay and
    Representation Differentially Shape Bias
    on the Pathway Into Organizations.
    Journal of Applied Psychology, 100:6, p1678-1712.

    This study highlights results where fictional students
    contacted faculty members.
    The outcome of interest was whether the faculty member
    would reply, and the variables of interest were the
    race and sex of the prospective student as well as
    demographics of the faculty member who received the message.
    The authors have made the data set publicly available,
    and we've put it into a CSV file that is friendly
    for downloading through the \data{research\_reply} data set.
    \Comment{Either get this data set in a sharable form
      or remove this reference.}

\end{itemize}

