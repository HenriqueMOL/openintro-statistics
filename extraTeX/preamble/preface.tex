\chapter*{Preface}
%\chaptertext{}
%\sectiontext{}

This book may be downloaded as a free PDF at
\oiRedirect{os}
    {\color{black}\textbf{openintro.org/os}}. \vspace{3mm}

\noindent We hope readers will take away three ideas from
this book in addition to forming a foundation of statistical
thinking and methods.\vspace{-1mm}
\begin{enumerate}
\setlength{\itemsep}{0mm}
\item[(1)] Statistics is an applied field with a wide range
    of practical applications.
\item[(2)] You don't have to be a math guru to learn
    from real, interesting data.
\item[(3)] Data are messy, and statistical tools are imperfect.
    But, when you understand the strengths and weaknesses of
    these tools, you can use them to learn about the world.
\end{enumerate}


\subsection*{Textbook overview}

\noindent%
The chapters of this book are as follows:%\vspace{2mm}
\begin{description}
\setlength{\itemsep}{0mm}
\item[1. Introduction to data.]
    Data structures, variables,
    and basic data collection techniques.
\item[2. Summarizing data.]
    Data summaries, graphics,
    and a teaser of inference using randomization.
\item[3. Probability.]
    Basic principles of probability.
    %This chapter is not required for the later chapters.
\item[4. Distributions of random variables.]
    The normal model and other key distributions.
\item[5. Foundations for inference.]
    %Introduction to uncertainty in point estimates,
    %confidence intervals, and hypothesis tests.
    General ideas for statistical inference in the context
    of estimating the population proportion.
\item[6. Inference for categorical data.]
    Inference for proportions and tables using the normal
    and chi-square distributions.
\item[7. Inference for numerical data.]
    Inference for one or two sample means using the
    \mbox{$t$-distribution},
    statistical power for comparing two groups,
    and also comparisons of many
    means using ANOVA.
\item[8. Introduction to linear regression.]
    Regression for a numerical outcome with a single predictor.
    Most of this chapter could be covered after
    Chapter~\ref{introductionToData}.
\item[9. Multiple and logistic regression.]
    Regression for numerical and categorical data
    using many predictors. %for an accelerated course.
\end{description}

\noindent%
\emph{OpenIntro Statistics} supports flexibility
in choosing and ordering topics.
If the main goal is to reach multiple regression
(Chapter~\ref{ch_regr_mult_and_log})
as quickly as possible, then consider the
following path:
\begin{itemize}
\item Chapter~\ref{ch_intro_to_data},
    Sections~\ref{numericalData},
    and Section~\ref{categoricalData} for a solid
    introduction to data structures and statistical
    summaries that are used throughout the book.
\item Section~\ref{normalDist}
    for a solid understanding of the normal distribution.
\item Chapter~\ref{ch_foundations_for_inf}
    to establish the core set of inference tools.
%\item Section~\ref{oneSampleMeansWithTDistribution}
%    and Chapter~\ref{ch_regr_simple_linear}
%    provide required for multiple regression with a numerical
%    outcome.
%    For the remaining chapters, they could be tackled in
%    almost any order, with the exception that
%    
%    
%    come before Chapter~\ref{ch_regr_mult_and_log}.
\item Section~\ref{oneSampleMeansWithTDistribution}
    to give a foundation for the $t$-distribution
\item Chapter~\ref{ch_regr_simple_linear}
    for establishing ideas and principles for single
    predictor regression.
%    introduce the 
%    which introduces the $t$-distribution, should come before
%    Section~\ref{oneSampleMeansWithTDistribution}
%Chapters~\ref{ch_inference_for_props}-\ref{ch_regr_mult_and_log},
%    could be tackled in
%    almost any order, with the exception that
%    Section~\ref{oneSampleMeansWithTDistribution}
%    and Chapter~\ref{ch_regr_simple_linear}
%    come before Chapter~\ref{ch_regr_mult_and_log}.
%\item Sections~\ref{ch_inference_for_props}
%    and~\ref{} are recommended before logistic regression.
\end{itemize}
%One conspicuously missing topic from the list above is the
%chapter on Probability.
%While useful for a deeper understanding of the calculations,
%especially for anyone looking to take a second course in
%statistics, it is not required reading when the focus is on
%applied data analysis.


\subsection*{Examples and exercises}
%, and appendices}

\noindent%
Examples and Guided Practice throughout the textbook may be
identified as follows:

\begin{examplewrap}
\begin{nexample}{This is an example.
    When a question is asked here, where can the answer be found?}
  The answer can be found here, in the solution section
  to the example!
\end{nexample}
\end{examplewrap}

\begin{exercisewrap}
\begin{nexercise}
When we think the reader should be ready to try something out
or answer a question, we frame it in a Guided Practice problem.
Of course, the reader may not know the answer,
so we provide the full solution in a footnote.\footnotemark{}
%Readers are strongly encouraged to attempt these practice problems.
\end{nexercise}
\end{exercisewrap}
\footnotetext{Guided Practice solutions are always located
    in a footnote.}

A large number of exercises are also provided at the end
of each chapter for practice or homework assignments.
Solutions are given for odd-numbered exercise in
Appendix~\ref{eoceSolutions}.
%Probability tables for the normal, $t$,
%and chi-square distributions are in
%Appendix~\ref{distributionTables}.


\subsection*{Additional resources}

OpenIntro is focused on increasing access to education.
\emph{OpenIntro Statistics} is intended for introductory
statistics courses at the college level.
We~offer another title,
\oiRedirect{textbook-books}
    {\emph{Advanced High School Statistics}},
that is intended for high school courses.

Video overviews, slides, statistical software labs,
data sets used in the textbook,
and much more are readily available at\\[-5mm]
\begin{center}
\oiRedirect{os}
    {\color{black}\textbf{openintro.org/os}}
\end{center}
%Data sets for this textbook are available on the website
%and in a companion R package.\footnote{Diez DM,
%    Barr CD, \c{C}etinkaya-Rundel M. 2015.
%    \texttt{openintro}: OpenIntro data sets and supplement
%    functions.
%    \oiRedirect{textbook-github_openintro}
%        {github.com/OpenIntroOrg/openintro-r-package}.}
%All of these resources are free and may be used with
%or without this textbook as a companion.
We also have improved the ability to access data in this book
through the addition of Appendix~\ref{data_appendix},
which provides a guide to each of the data sets used in the
main text.
Online guides to each of these data sets are also provided at
\oiRedirect{data}
    {\color{black}\textbf{openintro.org/data}}.

We appreciate all feedback as well as reports of any
typos through the website.
A short-link to report a new typo or review known typos is
\oiRedirect{os_typos}
    {\color{black}\textbf{openintro.org/os/typos}}.
\Comment{Make link work.}
%You may also learn more about OpenIntro or find our
%contact information at
%\oiRedirect{textbook-openintro_about}
%    {\color{black}\textbf{openintro.org/about}}.



\subsection*{Acknowledgements}

This project would not be possible without the passion
and dedication of all those involved.
The authors would like to thank the
\oiRedirect{textbook-openintro_about}{OpenIntro Staff}
for their involvement and ongoing contributions.
We~are also very grateful to the hundreds of students
and instructors who have provided us with valuable feedback
since we first started working on this book in~2009.

We also want to thank the many teachers who helped review
this edition, including
Laura Acion,
Jon C. New,
Mario Orsi,
Jesse Mostipak,
Jo Hardin,
\Comment{Complete the list.}