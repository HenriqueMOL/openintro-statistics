\chapter{Preface}
%\chaptertext{}
%\sectiontext{}

This book may be downloaded as a free PDF at
\oiRedirect{textbook-openintro}
    {\color{black}\textbf{openintro.org/os}}. \vspace{3mm}

\noindent We hope readers will take away three ideas from
this book in addition to forming a foundation of statistical
thinking and methods.\vspace{-1mm}
\begin{enumerate}
\setlength{\itemsep}{0mm}
\item[(1)] Statistics is an applied field with a wide range
    of practical applications.
\item[(2)] You don't have to be a math guru to learn
    from real, interesting data.
\item[(3)] Data are messy, and statistical tools are imperfect.
    But, when you understand the strengths and weaknesses of
    these tools, you can use them to learn about the world.
\end{enumerate}


\subsection*{Textbook overview}

\noindent%
The chapters of this book are as follows:%\vspace{2mm}
\begin{description}
\setlength{\itemsep}{0mm}
\item[1. Introduction to data.]
    Data structures, variables,
    and basic data collection techniques.
\item[2. Summarizing data.]
    Data summaries, graphics,
    and a teaser of inference using randomization.
\item[3. Probability.]
    The basic principles of probability.
    An understanding of this chapter is not required for
    the main content in
    Chapters~\ref{modeling}-\ref{multipleAndLogisticRegression}.
\item[4. Distributions of random variables.]
    Introduction to the normal model and other key
    distributions.
\item[5. Foundations for inference.]
    General ideas for statistical inference in the context
    of estimating the population proportion.
\item[6. Inference for categorical data.]
    Inference for proportions and tables using the normal
    and chi-square distributions.
\item[7. Inference for numerical data.]
    Inference for one or two sample means using the
    \mbox{$t$-distribution}, and also comparisons of many
    means using ANOVA.
\item[8. Introduction to linear regression.]
    An introduction to regression with two variables.
    Most of this chapter could be covered after
    Chapter~\ref{introductionToData}.
\item[9. Multiple and logistic regression.]
    A light introduction to multiple regression
    and logistic regression for an accelerated course.
\end{description}

\noindent%
\emph{OpenIntro Statistics} was written to allow flexibility
in choosing and ordering course topics.
The text has been structured to allow exploration of
several special topics, including chi-square, and ANOVA.
If the main goal is to move as fast as possible through
the material to get to later chapters, then consider the
following to be an outline of the required topics:
\begin{itemize}
\item Chapter~\ref{ch_intro_to_data} provides basic data
    considerations that are required, and
    Sections~\ref{numericalData}
    and~\ref{categoricalData}
    cover basic data summaries that are used throughout the book.
\item Obtain a basic foundation in the normal distribution
    in Section~\ref{normalDist}, which will be used through
    the inference sections.
\item Chapter~\ref{ch_foundations_for_inf} is all required
    to understand the topic of inference for the remaining
    chapters.
\item For the remaining chapters, they could be tackled in
    almost any order, with the exception that
    Section~\ref{oneSampleMeansWithTDistribution}
    and Chapter~\ref{ch_regr_simple_linear}
    come before Chapter~\ref{ch_regr_mult_and_log}.
\end{itemize}
One conspicuously missing topic from the list above is the
chapter on Probability.
While useful for a deeper understanding of the calculations,
especially for anyone looking to take a second course in
statistics, it is not required reading when the focus is on
applied data analysis.


\subsection*{Examples, exercises, and appendices}

Examples and Guided Practice throughout the textbook may be
identified as follows:

\begin{examplewrap}
\begin{nexample}{This is the example setting.}
  Full solutions to examples are provided and may include
  an accompanying table or figure.
\end{nexample}
\end{examplewrap}

\begin{exercisewrap}
\begin{nexercise}
We list a Guided Practice problem when we think the reader
should be ready to try something out.
This does not guarantee the reader will (or should)
know the answer, so we provide solutions for all Guided
Practice in footnotes.\footnotemark
\end{nexercise}
\end{exercisewrap}
\footnotetext{Full solutions are located down here
    in the footnote!}

There are exercises at the end of each chapter for practice
or homework assignments.
Odd-numbered exercise solutions are in
Appendix~\ref{eoceSolutions}.
Probability tables for the normal, $t$,
and chi-square distributions are in
Appendix~\ref{distributionTables}.

\subsection*{OpenIntro, online resources, and getting involved}

OpenIntro is an organization focused on developing
free and affordable education materials.
\emph{OpenIntro Statistics} is intended for introductory
statistics courses at the college level.
We~offer another title,
\oiRedirect{textbook-books}
    {\emph{Advanced High School Statistics}},
for high school courses.

Video overviews, slides, statistical software labs,
and much more is readily available at\\[-4mm]
\begin{center}
\oiRedirect{textbook-openintro_videos}{\color{black}\textbf{openintro.org/os}}
\end{center}
Data sets for this textbook are available on the website
and through a companion R package.\footnote{Diez DM,
    Barr CD, \c{C}etinkaya-Rundel M. 2015.
    \texttt{openintro}: OpenIntro data sets and supplement
    functions.
    \oiRedirect{textbook-github_openintro}
        {github.com/OpenIntroOrg/openintro-r-package}.}
All of these resources are free and may be used with
or without this textbook as a companion.

We value your feedback.
If there is a particular component of the project you
especially like or think needs improvement,
we want to hear from you.
You may find our contact information on the title page
of this book or at
\oiRedirect{textbook-openintro_about}
    {\color{black}\textbf{openintro.org/about}}.

\subsection*{Acknowledgements}

This project would not be possible without the passion
and dedication of all those involved.
The authors would like to thank the
\oiRedirect{textbook-openintro_about}{OpenIntro Staff}
for their involvement and ongoing contributions.
We~are also very grateful to the hundreds of students
and instructors who have provided us with valuable feedback
since we first started working on OpenIntro in~2009.

We also want to thank the many teachers who helped review
the new edition, including
Laura Acion,
\Comment{Complete the list.}