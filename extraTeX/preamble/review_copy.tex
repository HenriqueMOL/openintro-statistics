\chapter*{Feedback Instructions}
%\chaptertext{}
%\sectiontext{}

This is a review copy of an unfinished version of the
Fourth Edition of OpenIntro Statistics.
Please read this before reviewing the content.


\subsection*{What *not* to watch for}

\noindent%
There are several components that you should ignore.
\begin{enumerate}
\setlength{\itemsep}{0mm}
\item
    \textbf{End-of-section/chapter exercises
    and odd-numbered solutions will be included
    in the final version.}
    The newer exercises are not yet ready for sharing,
    so we've omitted exercises from this review copy
    to avoid any confusion.
\item
    \Comment{This is comment text that we are using to
    call out items that we know requires special attention
    and may be useful to you as FYIs.}
    For example, you'll see a big dot in the margin that
    ensures we spot and remove all these comments before
    finalizing the book.
\item
    There are plenty of awkward page breaks or footnotes
    on the incorrect page.
    These will be fixed during final textbook formatting.
\item
    There may be broken references (e.g. Figure~\ref{}).
    We will ensure we catch all of these that have
    a question mark.
    However, if you notice any that are simply pointed
    to the wrong thing entirely (we don't expect this
    to happen), please let us know.
\end{enumerate}


\subsection*{Where we're looking for feedback}

\noindent%
If you are browsing through the book and think,
``Hey, they should add / do / change / etc [thing]'',
please let us know.
You can submit feedback at
\begin{center}
\url{http://www.openintro.org/os4}
\end{center}
or by emailing \url{admin@openintro.org}. \\

\noindent%
Below are a few specific topics where you may
want to voice your thoughts:
\begin{enumerate}
\item
    Suggestions around where we might add more
    callouts around multivariate considerations.
    For example, if you are reading an example
    or case study and think that there's an
    interesting comment that might be made on
    confounding variables or on what a multivariate
    analysis would be like, please let us know.
\item
    If you read the fully-revised
    \emph{Foundations for Inference} chapter,
    what do you think about it?
    Do you like, dislike, or not care that we
    now introduce inference using proportions
    before means?
\item
    We have also reversed the ordering of the two chapters
    covering inference for proportions / means.
    Do we move too quickly or slowly in spots for either
    section?
    Which spots require more explanation or examples?
\item
    The new case study for logistic regression
    covers a sensitive yet important topic:
    racial discrimination.
    If you read this section, do you think the
    topic was presented in an appropriately respectful
    and responsible way in how it was discussed?

    Regardless of answers here, we will be getting
    a thorough review by subject-matter experts for this
    section.
\item
    In newer examples, we more strongly suggest software
    over using tables for finding tail areas.
    We are planning to do further changes around wording
    in existing examples and would like feedback on this
    direction.
%\item
%    The 3rd Edition launched with only black-and-white
%    paperbacks, and a year after launch we made
%    full color hardcovers available.
%    How important is it to you that we offer
%    (1) full-color books available and/or
%    (2) hardcover textbooks available?
%    (Our tentative plan is to launch with
%    a black-and-white paperback and also
%    a full-color paperback, where the expected
%    prices are \$20 and \$35, respectively.)
%\item
    
\end{enumerate}


\subsection*{Large changes already implemented}

\noindent%
The following sections contained notable updates
in content or examples:
\begin{itemize}
\item 1.2,
\item 1.3.4,
\item all of Chapter~\ref{ch_summarizing_data},
\item scattered loan data examples in 3.1,
\item stock return examples in 3.4,
\item nothing notable in Chapter~\ref{ch_distributions},
\item all of Chapter~\ref{ch_foundations_for_inf},
\item 6.1.2,
\item 6.1.3,
\item 6.3.5,
\item 6.4,
\item 7.1.5,
\item 7.2,
\item 7.5 (updated MLB data),
\item 8.4 (updated election data),
\item 9.4
\end{itemize}


\noindent%
Here are some special callouts for changes made:
\begin{description}
\item[Stylistic.]
    There is a new style for sections and subsections.
    Each section will also start at the top of the
    page and now be much easier to spot with the accent
    lines that have been added.
    (There are some bugs with spacing that we are still
    working out.)
    
    Video and slide icons / links have also been removed,
    since these will be presented in a different way
    in the Fourth Edition.
\item[Graphics and statistical summaries to their own chapter.]
    The first chapter of the Third Edition has been
    broken into two chapters in the Fourth Edition.
\item[Inference: proportions before means.]
    We introduce inference using proportion before means
    in the Fourth Edition.
\item[Simulation and randomization.]
    Two sections on inference for proportions
    in small sample situations has been removed from
    the Fourth Edition and will become online extras
    in April 2019.
    The randomization case study near the start of the
    textbook was retained and has a new case study.
\item[Lots of new examples.]
    We completed an audit of the data sets in the textbook.
    We have replaced or updated older or less interesting
    data sets with new case studies to make the book more
    engaging for both students and teachers.
    (A few lingering instances remain that will be resolved
    before the Fourth Edition is complete.)
    If any data sets strike you as outdated or uninteresting,
    please make a note and we'll consider making an update.
\item
    
\end{description}


\subsection*{Planned changes}

\noindent%
Below are tentative changes, and we welcome
feedback and suggestions on these plans.
\begin{enumerate}
\item%[Data references.]
    We are moving all data references into an appendix
    and out of footnotes in the text (in progress).
    Our goals with this change are to
    (1) simplify reading for the large majority of readers,
    and (2) provide a place where we can provide a complete
    list of all data sets in the text.
    The appendix will also include links (in the PDF)
    to pages dedicated to each data set
    and a CSV download link.
\item
    We are also tentatively planning to place exercises
    at the end of each section.
    We would also include a handful of exercises at the
    end of each chapter that would be more comprehensive.
\item
    Create a page that more strongly stands out for the
    start of a new chapter.
    Designs have been drawn up but are not yet implemented
    in the \LaTeX{} source files.
\item
    Replace the Mario Kart auction data in
    Chapter~9 with a new data set that is to-be-determined.
\item
    We are cutting out the
    \emph{sample size needs to be $\leq 10\%$ of the
    population size} condition.
    It will be mentioned briefly as a consideration
    but no longer included as a condition.
%    We've received several cases of feedback that this
%    is confusing (often asked: why is collecting more data bad?),
%    or that it is not practically relevant except
%    in very rare cases.
%    If you are concerned about this change,
%    please let us know.
\item
    The discussion of statistical vs practical significance
    is not in the new \emph{Foundations for Inference} chapter.
    However, it will be added back into the book before the
    Fourth Edition is released.
\item
    We are considering adding in a new section on graphics
    that would follow the sections on summarizing numerical
    and categorical data.
\item
    We may include some blank pages in the Fourth Edition
    launch if we plan to add specific types of new content.
    This strategy would allow us to add extra (non-critical)
    content later without affecting page numbering of
    textbooks already purchased or downloaded.
\item
    Do a thorough review of the inference chapters
    to ensure they read well in their new order.
\item
    You'll also find several comments throughout the book
    that callout additional items.
\end{enumerate}







