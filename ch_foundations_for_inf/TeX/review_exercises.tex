


%_______________
\section{Review exercises}

% 1

\eoce{\qt{Relaxing after work\label{relax_after_work}} The General Social Survey asked the question:
``After an average work day, about how many hours do you have to relax or pursue 
activities that you enjoy?" to a random sample of 1,155 Americans.\footfullcite{data:gss} A 95\% confidence interval for the mean number of hours spent 
relaxing or pursuing activities they enjoy was (1.38, 1.92).
\begin{parts}
\item Interpret this interval in context of the data.
\item Suppose another set of researchers reported a confidence interval with a 
larger margin of error based on the same sample of 1,155 Americans. How does 
their confidence level compare to the confidence level of the interval stated 
above?
\item Suppose next year a new survey asking the same question is conducted, and 
this time the sample size is 2,500. Assuming that the population 
characteristics, with respect to how much time people spend relaxing after work, 
have not changed much within a year. How will the margin of error of the 95\% 
confidence interval constructed based on data from the new survey compare to the 
margin of error of the interval stated above?
\end{parts}
}{}

% 2

\eoce{\qt{Minimum wage, Part 2\label{minimum_wage_prop_2}}
In Exercise~\ref{minimum_wage_prop_1},
we learned that a Rasmussen Reports survey
of 1,000 US adults found that 42\% believe
raising the minimum wage will help the economy.
Construct a 99\% confidence interval for the
true proportion of US adults who believe this.
}{}

% 3

\eoce{\qt{Testing for food safety\label{errors_food_safety}} A food safety inspector 
is called upon to investigate a restaurant with a few customer reports of poor 
sanitation practices. The food safety inspector uses a hypothesis testing 
framework to evaluate whether regulations are not being met. If he decides 
the restaurant is in gross violation, its license to serve food will be revoked.
\begin{parts}
\item Write the hypotheses in words.
\item What is a Type~1 Error in this context?
\item What is a Type~2 Error in this context?
\item Which error is more problematic for the restaurant owner? Why?
\item Which error is more problematic for the diners? Why?
\item As a diner, would you prefer that the food safety inspector requires 
strong evidence or very strong evidence of health concerns before revoking a 
restaurant's license? Explain your reasoning.
\end{parts}
}{}

% 4

\eoce{\qt{True or false\label{tf_found_inf_prop_friendly}}
Determine if the following statements are true or false, and 
explain your reasoning. If false, state how it could be corrected.
\begin{parts}
\item If a given value (for example, the null hypothesized value of a parameter) 
is within a 95\% confidence interval, it will also be within a 99\% confidence 
interval.
\item Decreasing the significance level ($\alpha$) will increase the probability 
of making a Type~1 Error.
\item Suppose the null hypothesis is $p = 0.5$ and we fail to reject $H_0$. 
Under this scenario, the true population proportion is 0.5.
\item With large sample sizes, even small differences between the null value and 
the observed point estimate, a difference often called the
effect size\index{effect size}, will be identified as statistically significant.
\end{parts}
}{}

% 5

\eoce{\qt{Unemployment and relationship problems\label{unemployment_relationship}} 
A USA Today/Gallup poll asked a group of
unemployed and underemployed Americans if they have
had major problems in their  relationships with their
spouse or another close family member as a result of
not having a job (if unemployed) or not having
a full-time job (if underemployed).
27\%~of the 1,145 unemployed respondents and
25\%~of the 675 underemployed respondents said they had
major problems in relationships as a  result of their
employment status.
\begin{parts}
\item
    What are the hypotheses for evaluating if the proportions
    of unemployed and underemployed people who had relationship
    problems were different?
\item
    The p-value for this hypothesis test is approximately 0.35.
    Explain what this means in context of the hypothesis test
    and the data.
\end{parts}
}{}

% 6

\eoce{\qt{Nearsighted\label{nearsighted_updated}}
It is believed that nearsightedness affects about 8\% of 
all children.
In a random sample of 194 children, 21 are nearsighted.
Conduct a hypothesis test for the following question:
do these data provide evidence that the 8\% value is inaccurate?
}{}

% 7

\eoce{\qt{Nutrition labels\label{nutrition_labels}}
The nutrition label on a bag of potato chips says
that a one ounce (28~gram) serving of potato chips
has 130 calories and contains ten grams of fat,
with three grams of saturated fat.
A~random sample of 35 bags yielded
a confidence interval for the number of calories
per bag of 128.2 to 139.8 calories.
Is there evidence that the nutrition label does not 
provide an accurate measure of calories in the bags
of potato chips?
}{}

% 8

\eoce{\qt{CLT for proportions\label{CLT_prop}}
Define the term ``sampling distribution" of the sample proportion,
and describe how the shape, center, and spread of the sampling
distribution change as the sample size increases when $p = 0.1$.
}{}

% 9

\eoce{\qt{Practical vs. statistical significance\label{prac_stat_sig}}
Determine whether the following statement is true
or false, and explain your reasoning:
``With large sample sizes, even small differences
between the null value and the observed point
estimate can be statistically significant.''
}{}

% 10

\eoce{\qt{Same observation, different sample size\label{same_obs_diff_n}} Suppose you 
conduct a hypothesis test based on a sample where the sample size is $n = 50$, 
and arrive at a p-value of 0.08. You then refer back to your notes and discover 
that you made a careless mistake, the sample size should have been $n = 500$. 
Will your p-value increase, decrease, or stay the same? Explain.
}{}

% 11

\eoce{\qt{Gender pay gap in medicine\label{gender_pay_gap_medicine}}
A study examined the average pay for men and women
entering the workforce as doctors for 21 different
positions.\footfullcite{LoSassoMedicineGenderPayGap}
\begin{parts}
\item\label{gender_pay_gap_medicine_hypotheses}
    If each gender was equally paid, then we would expect
    about half of those positions to have men paid more
    than women and women would be paid more than men in
    the other half of positions.
    Write appropriate hypotheses to test this scenario.
\item
    Men were, on average, paid more in 19 of those
    21 positions.
    Complete a hypothesis test using your hypotheses
    from part~(\ref{gender_pay_gap_medicine_hypotheses}).
\end{parts}
}{}
