\exercisesheader{}

% 15

\eoce{\qt{Possum classification, Part I\label{possum_classification_model_select}} 
The common brushtail possum of the Australia region is a bit cuter than its 
distant cousin, the American opossum (see Figure~\vref{brushtail_possum}). We 
consider 104 brushtail possums from two regions in Australia, where the possums 
may be considered a random sample from the population. The first region is 
Victoria, which is in the eastern half of Australia and traverses the southern 
coast. The second region consists of New South Wales and Queensland, which make 
up eastern and northeastern Australia.
We use logistic regression to differentiate between possums in these two 
regions. The outcome variable, called \var{population}, takes value 1 when a 
possum is from Victoria and 0 when it is from New South Wales or Queensland. We 
consider five predictors: \var{sex\_\hspace{0.3mm}male} (an indicator for a 
possum being male), \var{head\_\hspace{0.3mm}length}, \var{skull\_\hspace{0.3mm}
width}, \var{total\_\hspace{0.3mm}length}, and \var{tail\_\hspace{0.3mm}length}. 
Each variable is summarized in a histogram. The full logistic regression model 
and a reduced model after variable selection are summarized in the table.
\begin{center}
\includegraphics[width=\textwidth]{ch_regr_mult_and_log/figures/eoce/possum_classification_model_select/possum_variables.pdf} 
\end{center}
\begin{center}\footnotesize
\begin{tabular}{r rrrr r rrrr}
                            & \multicolumn{4}{c}{\emph{Full Model}} &
                            & \multicolumn{4}{c}{\emph{Reduced Model}}  \\
  \cline{2-5}\cline{7-10}
\vspace{-3.1mm} \\
                            & Estimate & SE & Z & Pr($>$$|$Z$|$) &
                            & Estimate & SE & Z & Pr($>$$|$Z$|$) \\ 
  \hline
\vspace{-3.1mm} \\
(Intercept)                 & 39.2349 & 11.5368 & 3.40  & 0.0007 &
                            & 33.5095 & 9.9053  & 3.38  & 0.0007 \\ 
sex\_\hspace{0.3mm}male     & -1.2376 & 0.6662  & -1.86 & 0.0632 &
                            & -1.4207 & 0.6457  & -2.20 & 0.0278 \\ 
head\_\hspace{0.3mm}length  & -0.1601 & 0.1386  & -1.16 & 0.2480 \\ 
skull\_\hspace{0.3mm}width  & -0.2012 & 0.1327  & -1.52 & 0.1294 &
                            & -0.2787 & 0.1226  & -2.27 & 0.0231 \\ 
total\_\hspace{0.3mm}length & 0.6488  & 0.1531  & 4.24  & 0.0000 &
                            & 0.5687  & 0.1322  & 4.30  & 0.0000 \\ 
tail\_\hspace{0.3mm}length  & -1.8708 & 0.3741  & -5.00 & 0.0000 &
                            & -1.8057 & 0.3599  & -5.02 & 0.0000 \\ 
  \hline
\end{tabular}
\end{center}
\begin{parts}
\item Examine each of the predictors. Are there any outliers that are likely to 
have a very large influence on the logistic regression model?
\item The summary table for the full model indicates that at least one variable 
should be eliminated when using the p-value approach for variable selection: 
\var{head\_\hspace{0.3mm}length}. The second component of the table summarizes 
the reduced model following variable selection. Explain why the remaining estimates 
change between the two models.
\end{parts}
}{}

% 16

\eoce{\qt{Challenger disaster, Part I\label{challenger_disaster_model_select}} 
On January 28, 1986, a routine launch was anticipated for the Challenger space 
shuttle. Seventy-three seconds into the flight, disaster happened: the shuttle 
broke apart, killing all seven crew members on board. An investigation into the 
cause of the disaster focused on a critical seal called an O-ring, and it is 
believed that damage to these O-rings during a shuttle launch may be related to 
the ambient temperature during the launch. The table below summarizes 
observational data on O-rings for 23 shuttle missions, where the mission order 
is based on the temperature at the time of the launch. \emph{Temp} gives the 
temperature in Fahrenheit, \emph{Damaged} represents the number of damaged O-
rings, and \emph{Undamaged} represents the number of O-rings that were not 
damaged.
\begin{center}
\begin{tabular}{l rrrrr rrrrr rrrrr rrrrr rrr}
\hline
\vspace{-3.1mm} \\
Shuttle Mission   & 1  & 2 & 3 & 4 & 5 & 6 & 7 & 8 & 9 & 10 & 11 & 12 \\
\hline
\vspace{-3.1mm} \\
Temperature       & 53 & 57 & 58 & 63 & 66 & 67 & 67 & 67 & 68 & 69 & 70 & 70  \\
Damaged           & 5  & 1 & 1 & 1 & 0 & 0 & 0 & 0 & 0 & 0 & 1 & 0 \\
Undamaged         & 1  & 5 & 5 & 5 & 6 & 6 & 6 & 6 & 6 & 6 & 5 & 6 \\
\hline
\\ 
\cline{1-12}
\vspace{-3.1mm} \\
Shuttle Mission   & 13 & 14 & 15 & 16 & 17 & 18 & 19 & 20 & 21 & 22 & 23 \\
\cline{1-12}
\vspace{-3.1mm} \\
Temperature       & 70 & 70 & 72 & 73 & 75 & 75 & 76 & 76 & 78 & 79 & 81 \\
Damaged           & 1  & 0 & 0 & 0 & 0 & 1 & 0 & 0 & 0 & 0 & 0 \\
Undamaged         & 5  & 6 & 6 & 6 & 6 & 5 & 6 & 6 & 6 & 6 & 6 \\
\cline{1-12}
\end{tabular}
\end{center}
\begin{parts}
\item Each column of the table above represents a different shuttle mission. 
Examine these data and describe what you observe with respect to the 
relationship between temperatures and damaged O-rings.
\item Failures have been coded as 1 for a damaged O-ring and 0 for an undamaged 
O-ring, and a logistic regression model was fit to these data. A summary of this 
model is given below. Describe the key components of this summary table in words.
\begin{center}
\begin{tabular}{rrrrr}
  \hline
            & Estimate & Std. Error & z value   & Pr($>$$|$z$|$) \\ 
  \hline
(Intercept) & 11.6630  & 3.2963     & 3.54      & 0.0004 \\ 
Temperature & -0.2162  & 0.0532     & -4.07     & 0.0000 \\ 
  \hline
\end{tabular}
\end{center}
\item Write out the logistic model using the point estimates of the model 
parameters.
\item Based on the model, do you think concerns regarding O-rings are justified? 
Explain.
\end{parts}
}{}

% 17

\eoce{\qt{Possum classification, Part II\label{possum_classification_predict}} 
A logistic regression model was proposed for classifying common brushtail 
possums into their two regions in 
Exercise~\ref{possum_classification_model_select}. The outcome variable took 
value 1 if the possum was from Victoria and 0 otherwise.
\begin{center}
\begin{tabular}{r rrrr}
  \hline
\vspace{-3.1mm} \\
                            & Estimate  & SE      & Z     & Pr($>$$|$Z$|$) \\ 
  \hline
\vspace{-3.1mm} \\
(Intercept)                 & 33.5095   & 9.9053  & 3.38  & 0.0007 \\ 
sex\_\hspace{0.3mm}male     & -1.4207   & 0.6457  & -2.20 & 0.0278 \\ 
skull\_\hspace{0.3mm}width  & -0.2787   & 0.1226  & -2.27 & 0.0231 \\ 
total\_\hspace{0.3mm}length & 0.5687    & 0.1322  & 4.30  & 0.0000 \\ 
tail\_\hspace{0.3mm}length  & -1.8057   & 0.3599  & -5.02 & 0.0000 \\ 
  \hline
\end{tabular}
\end{center}
\begin{parts}
\item Write out the form of the model. Also identify which of the variables are 
positively associated when controlling for other variables.
\item Suppose we see a brushtail possum at a zoo in the US, and a sign says the 
possum had been captured in the wild in Australia, but it doesn't say which part 
of Australia. However, the sign does indicate that the possum is male, its skull 
is about 63 mm wide, its tail is 37 cm long, and its total length is 83 cm. What 
is the reduced model's computed probability that this possum is from Victoria? 
How confident are you in the model's accuracy of this probability calculation?
%logitp <- 33.5095 - 1.4207 - 0.2787*63 + 0.5687*83 - 1.8057*37; exp(logitp)/(1+exp(logitp))
\end{parts}
}{}

% 18

\eoce{\qt{Challenger disaster, Part II\label{challenger_disaster_predict}} 
Exercise~\ref{challenger_disaster_model_select} introduced us to O-rings that 
were identified as a plausible explanation for the breakup of the Challenger 
space shuttle 73 seconds into takeoff in 1986. The investigation found that the 
ambient temperature at the time of the shuttle launch was closely related to the 
damage of O-rings, which are a critical component of the shuttle. See this 
earlier exercise if you would like to browse the original data.
\begin{center}
\includegraphics[width=0.6\textwidth]{ch_regr_mult_and_log/figures/eoce/challenger_disaster_predict/challenger_disaster_damage_temp.pdf} 
\end{center}
\begin{parts}
\item The data provided in the previous exercise are shown in the plot. The logistic 
model fit to these data may be written as
\begin{align*}
\log\left( \frac{\hat{p}}{1 - \hat{p}} \right) = 11.6630 - 0.2162\times Temperature
\end{align*}
where $\hat{p}$ is the model-estimated probability that an O-ring will become 
damaged. Use the model to calculate the probability that an O-ring will become 
damaged at each of the following ambient temperatures: 51, 53, and 55 degrees 
Fahrenheit. The model-estimated probabilities for several additional ambient 
temperatures are provided below, where subscripts indicate the temperature:
\begin{align*}
&\hat{p}_{57} = 0.341
	&& \hat{p}_{59} = 0.251
	&& \hat{p}_{61} = 0.179
	&& \hat{p}_{63} = 0.124 \\
&\hat{p}_{65} = 0.084
	&& \hat{p}_{67} = 0.056
	&& \hat{p}_{69} = 0.037
	&& \hat{p}_{71} = 0.024
\end{align*}
\item Add the model-estimated probabilities from part~(a) on the plot, then 
connect these dots using a smooth curve to represent the model-estimated 
probabilities.
\item Describe any concerns you may have regarding applying logistic regression 
in this application, and note any assumptions that are required to accept the 
model's validity.
\end{parts}
}{}
