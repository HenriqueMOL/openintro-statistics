


%_______________
\section{Review exercises}

% 1

\eoce{\qt{Multiple regression fact checking\label{mult_regr_facts}}
\textbf{\color{red}PROOF READ and ADD SOLUTIONS.}
Determine which of the following statements are
true and false.
For those statements that are false,
explain why it is false.
\begin{parts}
\item
    If variables are collinear, then removing
    one variable will have no influence on the
    point estimate of another variable's coefficient.
\item
    Suppose a numerical variable $x$ has a coefficient of
    $b_1 = 2.5$ in the multiple regression model.
    Suppose also that the first observation has $x_1 = 7.2$
    the second observation has a value of $x_1 = 8.2$,
    and these two observations have the same values
    for all other predictors.
    Then the predicted value of the second observation
    will be 2.5 higher than the prediction of the first
    observation based on the multiple regression model.
\item
    If a regression model's first variable has
    a coefficient of $b_1 = 5.7$, then if we are
    able to modify an observation so that $x_1$
    takes a value of 1 more than it would otherwise,
    the value $y_1$ for this observation would
    increase by 5.7.
\item
    Suppose we fit a multiple regression model
    based on a data set of 472 observations.
    We also notice that the distribution of the
    residuals includes some skew but does not
    include any particularly extreme outliers.
    Because the residuals are not nearly normal,
    we should not use this model and require
    more advanced methods to model these data.
\end{parts}
}{}

% 2

\eoce{\qt{Logistic regression fact checking\label{log_regr_facts}}
\textbf{\color{red}PROOF READ and ADD SOLUTIONS.}
Determine which of the following statements are
true and false.
For those statements that are false,
explain why it is false.
\begin{parts}
\item
    Suppose we consider the first two observations
    based on a logistic regression model,
    where the first variable in observation~1
    takes a value of $x_1 = 6$ and observation~2
    has $x_1 = 4$.
%    Each observation has all the same values for the
%    other variables used in the model.
    Suppose we realized we made an error for these
    two observations, and the first observation
    was actually $x_1 = 7$ (instead of~6)
    and the second observation actually had
    $x_1 = 5$ (instead of~4).
    Then the predicted probability from the
    logistic regression model would increase
    the same amount for each observation after
    we correct these variables.
\item
    When using a logistic regression model,
    it is impossible for the model to predict
    a probability that is negative or a probability
    that is greater than 1.
\item
    Because logistic regression predicts probabilities
    of outcomes, observations used to build a logistic
    regression model need not be independent.
\item
    When fitting logistic regression,
    we typically complete model selection using
    adjusted $R^2$.
\end{parts}
}{}

% 3

\eoce{\qt{Title\label{mult_regr_ex}}
\textbf{\color{red}CREATE THIS QUESTION.}
\begin{parts}
\item Fact 1.
\item Fact 2.
\item Fact 3.
\item Fact 4.
\end{parts}
}{}

% 4

\eoce{\qt{Spam filtering, Part I\label{spam_filtering_model_sel}}
\textbf{\color{red}ADD SOLUTION.}
Spam filters are built on principles similar to those
used in logistic regression.
Each message gets a probability to represent the
model's prediction that it is spam or not spam.
We have a set of variables for a large number of
emails:
to\us{}multiple, cc, attach, dollar, winner,
inherit, password, format, re\us{}subj,
exclaim\us{}subj, and sent\us{}email.
We won't describe what each variable means
here for the sake of brevity, but each is
either a numerical variable or an indicator variable.
\begin{parts}
\item
    For variable selection,
    we fit the full model, which includes all
    variables, and then we fit each model where
    we've dropped exactly one of the variables
    and identified the AIC value for each such
    model.
    Based on the results below, which variable,
    if any, should we drop as part of model
    selection?
    Explain.
    \begin{center}
    \begin{tabular}{lc}
      \hline
      Variable Dropped & AIC \\ 
      \hline
      Keep All & 1863.50 \\ 
      to\us{}multiple & 2023.50 \\ 
      cc & 1863.18 \\ 
      attach & 1871.89 \\ 
      dollar & 1879.70 \\ 
      winner & 1885.03 \\ 
      inherit & 1865.55 \\ 
      password & 1879.31 \\ 
      format & 2008.85 \\ 
      re\us{}subj & 1904.60 \\ 
      exclaim\us{}subj & 1862.76 \\ 
      sent\us{}email & 1958.18 \\ 
      \hline
    \end{tabular}
    \end{center}

\item
    Consider the following version of the model.
    Here again we've computed the AIC
    for each leave-one-variable-out model.
    Based on the results, which variable,
    if any, should we drop as part of model
    selection?
    Explain.
    \begin{center}
    \begin{tabular}{lc}
      \hline
      Variable Dropped & AIC \\ 
      \hline
      Keep All & 1862.41 \\ 
      to\us{}multiple & 2019.55 \\ 
      attach & 1871.17 \\ 
      dollar & 1877.73 \\ 
      winner & 1884.95 \\ 
      inherit & 1864.52 \\ 
      password & 1878.19 \\ 
      format & 2007.45 \\ 
      re\us{}subj & 1902.94 \\ 
      sent\us{}email & 1957.56 \\ 
      \hline
    \end{tabular}
    \end{center}
\end{parts}
}{}

% 5

\eoce{\qt{Spam filtering, Part 2\label{spam_filtering_predict}}
\textbf{\color{red}ADD SOLUTION.}
In Exercise~\ref{spam_filtering_model_sel}
we encountered a data set where we were applying
logistic regression to aid in spam classification
for individual emails.
In this exercise, we've taken a small set of these
variables and fit a formal model with the following
output:
\begin{center}
\begin{tabular}{rrrrr}
  \hline
  & Estimate & Std. Error & z value & Pr($>$$|$z$|$) \\ 
  \hline
  (Intercept) & -0.8124 & 0.0870 & -9.34 & 0.0000 \\ 
  to\us{}multiple & -2.6351 & 0.3036 & -8.68 & 0.0000 \\ 
  winner & 1.6272 & 0.3185 & 5.11 & 0.0000 \\ 
  format & -1.5881 & 0.1196 & -13.28 & 0.0000 \\ 
  re\us{}subj & -3.0467 & 0.3625 & -8.40 & 0.0000 \\ 
  \hline
\end{tabular}
\end{center}
\begin{parts}
\item
    Write down the model using the coefficients
    from the model fit.
    (Remember: this is logistic regression,
    so the output is not simply ``$y_i$''!)

\item
    Suppose we have an observation where
    $\var{to\us{}multiple} = 0$,
    $\var{winner} = 1$,
    $\var{format} = 0$, and
    $\var{re\us{}subj} = 0$.
    What is the predicted probability that this message
    is spam?

\item
    Put yourself in the shoes of a data scientist
    working on a spam filter.
    For a given message, how high must the probability
    a message is spam be before you think it would be
    reasonable to put it in a \emph{spambox}
    (which the user is unlikely to check)?
    What tradeoffs might you consider?
    Any ideas about how you might make your spam-filtering
    system even better from the perspective of someone
    using your email service?
\end{parts}
}{}
